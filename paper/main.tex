\documentclass{report}
\usepackage{import}

\import{./}{preamble.tex}

% Language and typesetting notes:
%
% - Color, not Colour (American English)

\author{Adrian Siwiec}
\date{\today{}}
\begin{document}

\begin{titlepage}
	\begin{center}
        
		\large
		\textbf{Jagiellonian University}\\
		Department of Theoretical Computer Science\\

		\vspace{1.5cm}

		\Large
		\textbf{Adrian Siwiec}

		\vspace*{2cm}

		\textbf{\LARGE Perfect Graph Recognition and Coloring}
		
		\vspace{0.5cm}
		\large
		
		\vfill
		\Large
		Master Thesis

		\vfill
		\Large
		Supervisor: dr in\.z. Krzysztof Turowski
		
		\vspace{0.8cm}
		
		September 2020
		
\end{center}
\end{titlepage}
\pagebreak

\begin{abstract}
	TODOa
\end{abstract}

\listoftheorems[ignoreall,show={defn}]
\tableofcontents

\pagebreak

\chapter*{Definitions}
We begin by recalling the basic notions of graph theory. We use standard definitions, sourced from the book by \citeauthor{BB98} \citetitle*{BB98}, modified and extended as needed.

\begin{defn}[graph]
  A \emph{graph} $G$ is an ordered pair of disjoint sets $(V, E)$ such that $E$ is the subset of the set $V \choose 2$, that is, of unordered pairs of $V$.
\end{defn}

\begin{wrapfigure}{r}{0.4\textwidth}
  \centering\begin{tikzpicture}[scale=.7, simplegraph]
    \node(a) at (0, 0) {$v_1$};
    \node(b) at (2, 0) {$v_2$};
    \node(c) at (4, 0) {$v_3$};
    \node(d) at (6, 0) {$v_4$};

    \draw(a) to (b);
    \draw(b) to (c);
    \draw(c) to (d);

    \draw[dashed](a) to[in=90, out=90] (d);
    \draw[dashed](a) to[in=90, out=90] (c);
    \draw[dashed](b) to[in=90, out=90] (d);
  \end{tikzpicture}
  \caption{An example graph $G_0$}
  \label{fig:examplePath}
\end{wrapfigure}

We will only consider finite graphs, that is, $V$ and $E$ are always finite. If $G$ is a graph, then $V = V(G)$ is the \emph{vertex set} of $G$, and $E = E(G)$ is the \emph{edge set}. When the context of $G$ is clear we will use $V$ and $E$ to denote its vertex and edge set.

An edge $\{x, y\}$ is said to \emph{join}, or \emph{be between} vertices $x$ and $y$ and is denoted by $xy$. Thus $xy$ and $yx$ denote the same edge (all our graphs are \emph{undirected}). If $xy \in E(G)$ then $x$ and $y$ are \emph{adjacent}, \emph{connected} or \emph{neighboring}. By $N(x)$ we will denote the \emph{neighborhood} of $x$, that is, all vertices $y$ such that $xy$ is an edge. Similarly, for $X \subseteq V(G)$, by $N(X)$ we will denote the \emph{neighborhood} of $X$, meaning all vertices of $v \in V(G) \setminus X$, so that there is a $x \in X$, that $xv$ is an edge in $G$. If $xy \notin E(G)$ then $xy$ is a \emph{nonedge} and $x$ and $y$ are \emph{anticonnected}.

\Cref{fig:examplePath} shows an example graph $G_0 = (V, E)$ with $V = \{v_1, v_2, v_3, v_4\}$ and $E = \{v_1v_2, v_2v_3, v_3v_4\}$. We will mark edges as solid lines on figures. Nonedges significant to the ongoing reasoning will be marked as dashed lines.

\begin{defn}[subgraph]
  $G' = (V', E')$ is a \emph{subgraph} of $G = (V, E)$ if and only if $V' \subseteq V$ and $E' \subseteq E$.
\end{defn}

\begin{defn}[induced subgraph]
  If $G' = (V', E')$ is a subgraph of $G$ and it contains \emph{all edges} of $G$ that join two vertices in $V'$, then $G'$ is said to be \emph{induced subgraph} of $G$ and is denoted $G[V']$.
\end{defn}

Given a graph $G = (V, E)$ and a set $X \subseteq V$ by $G\setminus X$ we will denote a induced subgraph $G[V\setminus X]$.

For example $(\{v_1, v_2, v_3\}, \{v_1v_2\})$ is \emph{not} an induced subgraph of the example graph $G_0$, while $(\{v_1, v_2, v_3\}, \{v_1v_2, v_2v_3\}) = G_0[\{v_1, v_2, v_3\}] = G_0 \setminus \{v_0\}$ is.

\begin{defn}[$X$-completeness]
  Given a graph $G = (V, E)$ and a set $X \subseteq V$, vertex $v \in V(G) \setminus X$ is \emph{$X$-complete} if and only if it is adjacent to every vertex $x \in X$. A set $Y \subseteq V$ is $X$-complete if and only if $X \cap Y = \emptyset$ and every vertex $y \in Y$ is $X$-complete.
\end{defn}

For example, for $X = \{v_2\}$, the set $\{v_1, v_3\}$ is $X$-complete in $G$, while the set $\{v_3, v_4\}$ is not.

\begin{defn}[path]
  A \emph{path} is a graph $P$ of the form
  \[ V(P) = \{x_1, x_2, \ldots, x_l\},\quad E(P) = \{x_1x_2, x_2x_3, \ldots, x_{l-1}x_l\} \]
\end{defn}
This path $P$ is usually denoted by $x_1x_2\ldots x_l$ or $x_1$-$x_2$-$\ldots$-$x_l$. The vertices $x_1$ and $x_l$ are the \emph{endvertices} and ${l-1} = |E(P)|$ is the \emph{length} of the path P. $\{x_2, \ldots x_{l-1}\}$ is the \emph{inside} of the path $P$, denoted as $P^*$. Notice that we don't allow any edges other than the ones between consecutive vertices for a graph to be called a path.

Graph $G_0$ is a path of length 3, with the inside $G_0^* = \{v_2, v_3\}$. If we  added any edge to $G_0$ it would stop being a path.


\begin{defn}[connected graph, subset]
  A graph $G$ is \emph{connected} if and only if for every pair $\{x, y\} \subseteq V(G)$ of distinct vertices, there is a path from $x$ to $y$.
  A subset $X \subseteq V(G)$ is connected if and only if the graph $G[X]$ is connected.
\end{defn}

\begin{defn}[component]
  A \emph{component} of a graph $G$ is its maximal connected induced subgraph.
\end{defn}


\begin{defn}[cycle]
  A \emph{cycle} is a graph $C$ of the form
  \[ V(C) = \{x_1, x_2, \ldots, x_l\},\quad E(C) = \{x_1x_2, x_2x_3, \ldots, x_{l-1}x_l, x_lx_1\} \]
\end{defn}

This cycle $C$ is usually denoted by $x_1x_2\ldots x_lx_1$ or $x_1 - x_2 - \ldots - x_l - x_1$. $l = |E(C)|$ is the \emph{length} of the cycle $C$. Sometimes we will denote the cycle of length $l$ as $C_l$.


\begin{defn}[hole]
  A \emph{hole} is a cycle of length at least four.
\end{defn}

If a path, a cycle or a hole has an odd length, it will be called \emph{odd}. Otherwise, it will be called \emph{even}. Notice that if we add an edge $v_1v_4$ to the path $G_0$ it becomes an even cycle $C_4$.
\begin{defn}[complement]
  A \emph{complement} of a graph $G = (V, E)$ is a graph $\overline{G} = (V, {V \choose 2} \setminus E)$, that is, two vertices $x, y$ are adjacent in $\overline{G}$ if and only if they are not adjacent in $G$.
\end{defn}

% Sometimes, we will call a complement of a member of a class $\Gamma$ an \emph{anti-
% $\Gamma$}, e.g. graph $G = (\{v_1, v_2, v_3, v_4\}, \{v_1v_3, v_2v_4\})$ is an anticycle.

\begin{defn}[anticonnected graph, subset]
  A graph $G$ is \emph{anticonnected} if and only if $\overline{G}$ is connected.
  A subset $X$ is \emph{anticonnected} if and only if $\overline{G}[X]$ is connected.
\end{defn}

\begin{defn}[anticomponent]
  An \emph{anticomponent} of a graph $G$ is an induced subgraph whose complement is a component in $\overline{G}$.
\end{defn}

\begin{defn}[antihole]
  An \emph{antihole} is a subgraph of $G$ whose complement is a hole in $G$.
\end{defn}

\begin{defn}[clique]
  A \emph{complete graph} or a \emph{clique} is a graph of the form $G = (V, {V \choose 2})$, that is, every two vertices are connected. We will denote a clique on $n$ vertices as $K_n$.
\end{defn}

\begin{defn}[clique number]
  A \emph{clique number} of a graph $G$, denoted as $\omega(G)$, is a cardinality of its largest induced clique.
\end{defn}

\begin{defn}[anticlique]
  An \emph{independent set} or an \emph{anticlique} is a graph of the form $G = (V, \emptyset)$, that is no vertices are connected.
\end{defn}
In a similar fashion, given a graph $G = (V, E)$, a subset of its vertices $V' \subseteq V$ will be called \emph{independent} (in the context of $G$) if and only if $G[V']$ is an anticlique.

\begin{defn}[stability number]
  A \emph{stability number} of a graph $G$, denoted as $\alpha(G)$, is a cardinality of its largest induced stable set.
\end{defn}

\begin{defn}[coloring]
  Given a graph $G$, its \emph{coloring} is a function $c: V(G) \rightarrow \mathbb{N}^+$, such that for every edge $xy \in E(G)$, $c(x) \neq c(y)$ . A $k$-coloring of $G$ (if exists) is a coloring, such that $c(x) \leq k$ for all vertices $x \in V(G)$.
\end{defn}

\begin{defn}[chromatic number]
  A \emph{chromatic number} of a graph $G$, denoted as $\chi(G)$, is a smallest natural number $k$, for which there exists a $k$-coloring of $G$.
\end{defn}

\begin{defn}[line graph]
  The \emph{line graph} of a graph $G = (V, E)$ is the graph $L(G)$ with $V(L(G)) = E$ and $E(L(G)) = \{e_1 e_2: e_1, e_2 \in E, e_1 \cap e_2 \neq \emptyset\}$, that is, $e_1, e_2 \in E$ are adjacent if and only if they share an endpoint in $G$.
\end{defn}




\chapter{Perfect Graphs}
Given a graph $G$, let us consider a problem of coloring it using as few colors as possible. If $G$ contains a clique $K$ as a subgraph, we must use at least $|V(K)|$ colors to color it. This gives us a lower bound for a chromatic number $\chi(G)$ -- it is always greater or equal to the cardinality of the largest clique $\omega(G)$. The reverse is not always true, in fact we can construct a graph with no triangle and requiring arbitrarily large numbers of colors (e.g. construction by Mycielski \cite{Mycielski1955}).

Do graphs that admit coloring using only $\omega(G)$ color are "simpler" to further analyze? Not necessarily so. Given a graph $G = (V, E)$, $|V| = n$, let us construct a graph $G'$ as the union of $G$ and a clique $K_n$. We can see that indeed $\chi(G') = \omega(G') = n$, but it gives us no indication of the structure of $G$ or $G'$. This leads us to the hereditary definition of \emph{perfect graphs}.

\begin{defn}[perfect graph]
	\label{def:perfectGraph}
	A graph $G$ is \emph{perfect} if and only if for its every induced subgraph $H$ we have $\chi(H) = \omega(H)$.
\end{defn}

The notion of perfect graphs was first introduced by Berge in 1961 \cite{CB61} and it indeed captures some of the idea of graph being ''simple'' -- in all perfect graphs the coloring problem, maximum (weighted) clique problem, and maximum (weighted) independent set problem can be solved in polynomial time \cite{grotschel1993}. Other classical NP-complete problems are still NP-complete in perfect graphs e.g. Hamiltonian path \cite{Mller1996}, maximum cut problem \cite{Bodlaender1994} or dominating set problem \cite{Dewdney81}. There are many subclasses of perfect graphs, we take a look and benchmark our implementations on some of them in \Cref{sec:experiments}

The most fundamental problem -- the problem of recognizing perfect graphs -- was open since its posing in 1961 until recently. Its solution, a polynomial algorithm recognizing perfect graphs is a union of the strong perfect graph theorem (\cref{sec:SPGT}) stating that a graph is perfect if and only if it is Berge and an algorithm for recognizing Berge graphs in polynomial time (\cref{sec:recognizingBerge}).

\begin{defn}[Berge graph]
	\label{def:bergeGraph}
	A graph $G$ is \emph{Berge} if and only if both $G$ and $\overline{G}$ have no odd hole as an induced subgraph.
\end{defn}

Perfect graphs are interesting not only because of their theoretical properties, but they are also used in other areas of study e.g. integrality of polyhedra \cite{Chvtal1975, Chudnovsky2003}, radio channel assignment problem \cite{McDiarmid99, McDiarmid2000} and appear in the theory of Shannon capacity of a graph \cite{Lovasz1979}. Also, as pointed out in \cite{alfonsinPerfect2001, Chudnovsky2003} algorithms to solve semi-definite programs grew out of the theory of perfect graphs. We will take a look at semi-definite programs and at perfect graph's relation to Shannon capacity in \cref{sec:ShannonCapacity}.

\section{Strong Perfect Graph Theorem}
\label{sec:SPGT}

The first step to solve the problem of recognizing perfect graphs was the \emph{(weak) perfect graph theorem} first conjured by Berge in 1961~\cite{CB61} and then proven by Lovász in 1972~\cite{LL72}.

\begin{theorem}[Perfect graph theorem]
	\label{thm:pgt}
	A graph is perfect if and only if its complement graph is also perfect.
\end{theorem}

This theorem is a consequence of a stronger result proven by Lovász:
\begin{theorem}
	\label{thm:omegaalpha}
	A graph $G$ is perfect if and only if for every induced subgraph $H$, the number of vertices of $H$ is at most $\alpha(H)\omega(H)$. \todo{proof!}
\end{theorem}

Then, since $\alpha(H) = \omega(\overline{H})$ and $\omega(H) = \alpha(\overline{H})$ \cref{thm:omegaalpha} implies \cref{thm:pgt}.

Odd holes are not perfect, since their chromatic number is 3 and their largest cliques are of size 2. It is also easy to see, that an odd antihole of size $n$ has a chromatic number of $\frac{n+1}{2}$ and largest cliques of size $\frac{n-1}{2}$. A graph with no odd hole and no odd antihole is called \emph{Berge} (\cref{def:bergeGraph}) after Claude Berge who studied perfect graphs.

In 1961 Berge conjured that a graph is perfect if and only if it contains no odd hole and no odd antihole in what has become known as a strong perfect graph conjecture. In 2001 Chudnovsky et al. have proven it and published the proof in an over 150 pages long paper \citetitle{MC06} \cite{MC06}.

\begin{theorem}[Strong perfect graph theorem]
	\label{thm:spgt}
	A graph is perfect if and only if it is Berge.
\end{theorem}

The proof is long and complicated. Moreover, it has little noticeable connection to the algorithm of recognizing Berge graphs we discuss later. Therefore we will discuss it very briefly following the overview by Cornuéjols \cite{GC03}.

\subsubsection{Basic classes of perfect graphs}
Bipartite graphs are perfect, since we can color them with two colors. From the theorem of König we get that line graphs of bipartite graphs are also perfect \cite{Knig1916, GC03}. From the perfect graph theorem (\cref{thm:pgt}) it follows that complements of bipartite graphs and complement of line graphs of bipartite graphs are also perfect. We will call these four classes \emph{basic}.

\subsubsection{2-join, Homogeneous Pair and Skew Partition}
A graph $G$ has a \emph{2-join} if its vertices can be partitioned into sets $V_1$, $V_2$, each of size at least three, and there are nonempty disjoint subsets $A_1, B_1 \subseteq V_1$ and $A_2, B_2 \subseteq V_2$, such that all vertices of $A_1$ are adjacent to all vertices of $A_2$, all vertices of $B_1$ are adjacent to all vertices of $B_2$, and these are the only edges between $V_1$ and $V_2$. When a graph $G$ has a 2-join, it can be decomposed onto two smaller graphs $G_1$, $G_2$, so that $G$ is perfect if and only if $G_1$ and $G_2$ are perfect \cite{Cornujols1985}.

A graph $G$ has a \emph{homogeneous pair} if $V(G)$ can be partitioned into subsets $A_1$, $A_2$, $B$, such that $|A_1|+|A_2| \geq 3$, $|B| \geq 2$ and if a vertex $v \in B$ is adjacent to a vertex from $A_i$, then it is adjacent to all vertices from $A_i$. Chvátal and Sbihi proved that no minimally imperfect graph has a homogeneous pair \cite{Chvtal1987}.

A graph $G$ has a \emph{skew partition} if $V(G)$ can be partitioned into nonempty subsets $A, B, C, D$ such that there are all possible edges between $A$ and $B$ and no edges between $C$ and $D$. Chudnovsky et at. proved that no minimally imperfect graph has a skew partition.

The proof of \cref{thm:spgt} is a consequence of the \emph{Decomposition theorem}:
\begin{theorem}[Decomposition theorem]
	\label{thm:decomposition}
	Every Berge graph $G$ is basic or has a skew partition or a homogeneous pair, or either $G$ or $\overline{G}$ has a 2-join.
\end{theorem}

See \cite{MC06} for the proof of \cref{thm:spgt} and \cref{thm:decomposition}.

\section{Recognizing Berge Graphs}
\label{sec:recognizingBerge}

The following is based on the paper by Maria \citeauthor{MC05} \citetitle*[]{MC05} \cite{MC05}. We will not provide full proof of its correctness, but will aim to show the intuition behind the algorithm.

Berge graph recognition algorithm (later called CCLSV from the names of its authors) could be divided into two parts: first we check if either $G$ or $\overline{G}$ contain any of a number of simple forbidden structures (\cref{SimpleStructures}). If they do, we output that graph is not Berge and stop. Else, we check if there is a near-cleaner for a shortest odd hole (\cref{AmenableHoles}).

\subsection{Simple forbidden structures}
\label{SimpleStructures}

\subsubsection{Pyramids}

\begin{wrapfigure}{r}{0.35\textwidth}
	\centering\begin{tikzpicture}[scale=0.7,simplegraph]
  \node(a) at (0,0) {$a$};
  \node(b1) at (-2,-6) {$b_1$};
  \node(b2) at (0,-4.268) {$b_2$};
  \node(b3) at (2,-6) {$b_3$};

  \node(P12) at (-2/4, -6/4) {\small$P_{12}$};
  \node(P13) at (-4/4, -12/4) {\small$P_{13}$};
  \node(P14) at (-6/4, -18/4) {\small$P_{14}$};

  \node(P32) at (2/3, -6/3) {\small$P_{32}$};
  \node(P33) at (4/3, -12/3) {\small$P_{33}$};

  \draw (b1) to (b3);
  \draw (b1) to (b2);
  \draw (b2) to (b3);

  \draw (a) to (P12);
  \draw (P12) to (P13);
  \draw (P13) to (P14);
  \draw (P14) to (b1);

  \draw (a) to (P32);
  \draw (P32) to (P33);
  \draw (P33) to (b3);

  \draw (a) to (b2);
\end{tikzpicture}%
	\caption{An example  of a pyramid.}%
	\vspace{-1.2cm}
\end{wrapfigure}


A \emph{pyramid} in G is an induced subgraph formed by the union of a triangle \footnote{A triangle is a clique $K_3$.} $\{b_1,b_2,b_3\}$, three paths $\{P_1, P_2, P_3\}$ and another vertex $a$, so that:
\begin{itemize}
	\item $\forall_{1\leq i \leq 3}$ $P_i$ is a path between $a$ and $b_i$,
	\item $\forall_{1\leq i < j \leq 3}$ $a$ is the only vertex in both $P_i$ and $P_j$ and $b_ib_j$ is the only edge between $V(P_i)\setminus\{a\}$ and $V(P_j)\setminus\{a\}$,
	\item $a$ is adjacent to at most one of $\{b_1, b_2, b_3\}$.
\end{itemize}

We will say that $a$ can be \emph{linked onto} the triangle $\{b_1, b_2, b_3\}$ \emph{via} the paths $P_1$, $P_2$, $P_3$. Let us notice, that a pyramid is uniquely determined by its paths $P_1$, $P_2$, $P_3$.

It is easy to see that every graph containing a pyramid contains an odd hole -- at least two of the paths $P_1$, $P_2$, $P_3$ will have the same parity.

\paragraph{Finding Pyramids}


\begin{alg}[Test if $G$ contains a Pyramid]
	\label{alg:testPyramid}
	Input: A graph $G$.

	\noindent Output: Returns whether $G$ contains a pyramid as an induced subgraph.
\end{alg}
\begin{algorithmic}[1]
	\mProcedure{Contains-Pyramid}{$G$}
	\mForEach{triangle $b_1, b_2, b_3$} \label{line:pyramidTriangle}
		\mForEach{$s_1, s_2, s_3$, such that for $1 \leq i < j \leq 3$, $\{b_i, s_i\}$ is disjoint
			\lsx  from $\{b_j, s_j\}$ and $b_ib_j$ is the only edge between them} \label{line:pyramidTriple}
			\mIf{there is a vertex $a$, adjacent to all of $s_1, s_2, s_3$, and to at most
				\lsx one of $b_1, b_2, b_3$, such that if $a$ is adjacent to $b_i$, then $b_i = s_i$ \lsx}\label{line:a}
				\ls $M \gets V(G) \setminus \{b_1, b_2, b_3, s_1, s_2, s_3\}$
				\mForEach{$m \in M$} \label{line:pyramidSStart}
					\ls $S_1(m) \gets$ the shortest path between $s_1$ and $m$ such that
					\lsx $s_2, s_3, b_2, b_3$ have no neighbors in its interior, if such a 
					\lsx path exists.
					\ls calculate $S_2(m), S_3(m)$ similarly
					\ls $T_1(m) \gets$ the shortest path between $m$ and $b_1$, such that
					\lsx $s_2, s_3, b_2, b_3$ have no neighbors in its interior, if such a
					\lsx path exists
					\ls calculate $T_2(m), T_3(m)$ similarly
				\mEndFor \label{line:pyramidSEnd}
				 
					\mIf{$s_1 = b_1$} \Comment{calculate all possible $P_1$ paths}
						\ls $P_1(b_1) \gets$ the one-vertex path with vertex $b_1$
						\ls \algorithmicforeach ~$m \in M$ \algorithmicdo~ $P_1(m) \gets$ \textsc{undefined}
					\mElse
						\ls $P_1(b_1) \gets $ \textsc{undefined}
						\mForEach{$m \in M$}
							\mIf{ $m$ is nonadjacent to all of $b_2, b_3, s_2, s_3$ \AND
								\lsx $S_1(m)$ and $T_1(m)$ both exist \AND
								\lsx $V(S_1(m) \cap T_1(m)) = \{m\}$ \AND
								\lsx there are no edges between $V(S_1(m) \setminus m)$ 
								\lsx and $V(T_1(m) \setminus m)$ 
								\lsx
							}
								\ls $P_1(m) \gets s_1-S_1(m)-m-T_1(m)-b_1$
							\mElse
								\ls $P_1(m) \gets$ \textsc{undefined}
							\mEndIf
						\mEndFor
					\mEndIf
				
				\ls assign $P_2$ and $P_3$ in a similar manner
				\ls $good\_pairs_{1,2} \gets \emptyset$ \Comment{see below for definition}
				\mForEach{$m_1 \in M \cup \{b_1\}$} \Comment{calculate good $(1, 2)$-pairs}
					\mIf{$P_1(m_1) \neq \UND$}
						\ls color black the vertices of $M$ that either belong to \label{line:pyramidColor}
						\lsx $P_1(m_1)$ or have a neighbor in $P_1(m_1)$
						\lsx color all other vertices white.
						\mForEach{$m_2 \in M \cup \{b_2\}$}
							\mIf{$P_2(m_2)$ exists and contains no black vertices \lsx} \label{line:pyramidColor2}
								\ls add $(m_1, m_2)$ to $good\_pairs_{1,2}$
							\mEndIf
						\mEndFor
					\mEndIf
				\mEndFor
				\ls calculate $good\_pairs_{1,3}$ and $good\_pairs_{2,3}$ in similar way
				\mForEach{triple $m_1, m_2, m_3$ such that $m_i \in M \cup \{b_i\}$} \label{line:m1m2m3}
					\mIf{$\forall_{1\leq i < j \leq 3}$: $(m_i, m_j)$ is a good $(i, j)$-pair}
						\ls \RETURN \TRUE \label{line:pyramidTrue}
					\mEndIf
				\mEndFor
			\mEndIf \label{line:pyramidEnd}
		\mEndFor
	\mEndFor
	\ls \RETURN \FALSE
	\mEndProcedure
\end{algorithmic}

With definitions as above, for $1 \leq i < j \leq 3$, we say that $(m_i, m_j)$ is a \emph{good $(i, j)$-pair}, if and only if $m_i \in M \cup \{b_i\}$, $m_j \in M \cup \{b_j\}$, $P_i(m_i)$ and $P_j(m_j)$ both exist, and the sets $V(P_i(m_i))$,$V(P_j(m_j))$ are both disjoint and $b_ib_j$ is the only edge between them. In line \ref{line:pyramidColor} we color vertices of $P_1(m_1)$ black, so that for each $m_2$ we can check if paths $P_1(m_1)$ and $P_2(m_2)$ are disjoint in $O(|V|)$ time.

It is easy to see, that if \textsc{Contains-Pyramid($G$)} outputs that $G$ contains a pyramid, it indeed does -- when we return in line \ref{line:pyramidTrue} the vertex $a$ found in line \ref{line:a} can be linked into a triangle $b_1, b_2, b_3$ via paths $P_1(m_1), P_2(m_2), P_3(m_3)$ for $m_1$, $m_2$, $m_3$ from line \ref{line:m1m2m3}. The proof of the converse is rather technical and we refer to \todo{dokładny nr twierdzenia} \cite{MC05} for it.

Now we will proove the time complexity.

\begin{theorem}
	Procedure \textsc{Contains-Pyramid($G$)} works in $O(|V|^9)$ time.
\end{theorem}
\begin{proof}
	There are $O(|V|^3)$ triangles (line \ref{line:pyramidTriangle}) and $O(|V|^3)$ triples $s_1$, $s_2$, $s_3$ (line \ref{line:pyramidTriple}), so lines \ref{line:a}-\ref{line:pyramidEnd} are executed at most $O(|V|^6)$ times.

	Checking if there exists an appropriate $a$ takes linear time (line \ref{line:a}). Calculating paths $S_i$ and $T_i$ (lines \ref{line:pyramidSStart}-\ref{line:pyramidSEnd}) takes $O(|V|^2)$ time for each $m \in M$ and $O(|V|^3)$ in total. Similarly, it takes $O(|V|^3)$ time to calculate all $P_i$ paths.

	Then, for each $m_1$ we do at most $O(|V|)$ work in line \ref{line:pyramidColor} and for each $(m_1, m_2)$ we do at most $O(|V|)$ work in line \ref{line:pyramidColor2}.

	Finally, there are $O(|V|^3)$ pairs $m_1, m_2, m_3$ and checking each takes $O(1)$ time.
\end{proof}

\subsubsection{Jewels}

% \begin{wrapfigure}[7]{r}{0.35\textwidth}
\begin{wrapfigure}{r}{0.35\textwidth}
	\centering\begin{tikzpicture}[scale=0.7,simplegraph]
  \def\ngon{5}
  \node[regular polygon,regular polygon sides=\ngon,minimum size=3cm, draw=none] (p) {};
  \foreach\x in {1,...,\ngon}{\node[] (p\x) at (p.corner \x){$v_\the\numexpr\intcalcMod{\x+3}{5}+1$};}
  %p1 - v5, p2 - v1 ...
  \draw (p1) to (p2);
  \draw (p2) to (p3);
  \draw (p3) to (p4);
  \draw (p4) to (p5);
  \draw (p1) to (p5);

  \draw[dashed] (p2) to (p5);
  \draw[dashed] (p2) to (p4);
  \draw[dashed] (p3) to (p5);

  \tikzset{decoration={snake,amplitude=.6mm,segment length=6mm,
        post length=0mm,pre length=0mm}}
  \draw[decorate] (p2) to [out=-100, in=180] (0, -5) to [out=0, in=-80] (p5);
  \node[draw=none] at (0, -4.7) {$P$};
\end{tikzpicture}
	\caption{An example of a jewel.}
	\vspace{-1.5cm}
\end{wrapfigure}


Five vertices $v_1, \ldots, v_5$ and a path $P$ form a \emph{jewel} if and only if:

\begin{itemize}
	\item $v_1, \ldots, v_5$ are distinct vertices,
	\item $v_1v_2, v_2v_3, v_3v_4, v_4v_5, v_5v_1$ are edges,
	\item $v_1v_3, v_2v_4, v_1,v_4$ are nonedges,
	\item $P$ is a path between $v_1$ and $v_4$, such that $v_2, v_3, v_5$ have no neighbors in its inside.
\end{itemize}

Most obvious way to find a jewel would be to enumerate all (possibly chordal) cycles of length 5 as $v_1, \ldots v_5$, check if it has all required nonedges and if it does, try to find a path $P$ as required. This gives us a time of $O(|V|^7)$. We could speed it up to $O(|V|^6)$ with more careful algorithm, but since whole Berge recognition algorithm takes time $O(|V|^9)$ and our testing showed that time it takes to test for jewels is negligible we decided against it.

\subsubsection{Configurations of type $\T_1$}

A configuration of type $\T_1$ is a hole of length 5. To find it, we simply iterate over all paths of length 4 and check if there exists a fifth vertex to complete the hole. See \cref{sec:usesGeneration} for more implementation details.

\subsubsection{Configurations of type $\T_2$}

% \begin{wrapfigure}[15]{r}{0.35\textwidth}
\begin{wrapfigure}{r}{0.35\textwidth}
	\centering\begin{tikzpicture}[scale=0.7,simplegraph]
  \node(v1) at (0,0) {$v_1$};
  \node(v2) at (1.5,0) {$v_2$};
  \node(v3) at (3,0) {$v_3$};
  \node(v4) at (4.5,0) {$v_4$};

  \draw (v1) to (v2);
  \draw (v2) to (v3);
  \draw (v3) to (v4);

  \tikzset{decoration={snake,amplitude=.6mm,segment length=6mm,
        post length=0mm,pre length=0mm}}
  \draw[decorate] (v1) to [out=90, in=180] (4.5/2, 3) to [out=0, in=90] (v4);
  \node[draw=none] at (4.5/2, 3.5) {$P$};

  \draw[dashed] (v2) to (.7,2.5);
  \draw[dashed] (v3) to (.7,2.5);
  \draw[dashed] (v2) to (3.8,2.5);
  \draw[dashed] (v3) to (3.8,2.5);

  \node[minimum size=4mm](x1) at (1.2, -2){};
  \node[minimum size=4mm](x2) at (3.3, -2){};
  \node[draw,dotted,inner sep=3pt, circle,yscale=.5, fit={(x1) (x2)},label=below:{$X$}] {};

  \draw[dashed] (x1) to (x2);
  \draw(v1) to (x1);
  \draw(v2) to (x1);
  \draw(v4) to (x1);
  \draw(v1) to (x2);
  \draw(v2) to (x2);
  \draw(v4) to (x2);

  % \draw (a) to (b2);
\end{tikzpicture}%
	\caption{An example of a $\T_2$.}%
	\vspace{-1.5cm}
\end{wrapfigure}

A configuration of type $\T_2$ is a tuple $(v_1, v_2, v_3, v_4, P, X)$, such that:
\begin{itemize}
	\item $v_1v_2v_3v_4$ is a path in $G$.
	\item $X$ is an anticomponent of the set of all $\{v_1, v_2, v_4\}$-complete vertices.
	\item $P$ is a path in $G\setminus(X \cup \{v_2, v_3\})$ between $v_1$ and $v_4$ and no vertex in $P^*$ is $X$-complete or adjacent to $v_2$ or adjacent to $v_3$.
\end{itemize}

Checking if configuration of type $\T_2$ exists in our graph is straightforward: we enumerate all paths $v_1\ldots v_4$, calculate set of all $\{v_1, v_2, v_4\}$-complete vertices and its anticomponents. Then, for each anticomponent $X$ we check if required path $P$ exists.

To prove that existence of $\T_2$ configuration implies that the graph is not Berge, we will need the following Roussel-Rubio lemma:

\begin{lemma}[Roussel-Rubio Lemma \cite{RR01,MC05}]\label{lem:Roussel-Rubio}
	Let $G$ be Berge, $X$ be an anticonnected subset of $V(G)$, $P$ be an odd path $p_1\ldots p_n$ in $G\setminus X$ with length at least 3, such that $p_1$ and $p_n$ are $X$-complete and $p_2, \ldots, p_{n-1}$ are not. Then:
	\begin{itemize}
		\item $P$ is of length at least 5 and there exist nonadjacent $x, y \in X$, such that there are exactly two edges between $x, y$ and $P^*$, namely $xp_2$ and $yp_{n-1}$,
		\item or $P$ is of length 3 and there is an odd antipath joining internal vertices of $P$ with interior in $X$.
	\end{itemize}
	\TODO{We may use this lemma quite often, might want to provide proof if so.}
\end{lemma}

Now, we shall prove the following:

\begin{theorem}
	If $G$ contains configuration of type $\T_2$ then $G$ is not Berge.
\end{theorem}
\begin{proof}
	Let $(v_1, v_2, v_3, v_4, P, X)$ be a configuration of type $\T_2$. Let us assume that $G$ is not Berge and consider the following:
	\begin{itemize}
		\item If $P$ is even, then $v_1, v_2, v_3, v_4, P, v_1$ is an odd hole,
		\item If $P$ is of length 3. \todo{I merged a couple of proofs from \cite{MC06}, check in the morning if this is correct.} Let us name its vertices $v_1, p_2, p_3, v_4$. It follows from \cref{lem:Roussel-Rubio}, that there exists an odd antipath between $p_2$ and $p_3$ with interior in $X$. We can complete it with $v_2p_2$ and $v_2p_3$ into an odd antihole.
		\item If $P$ is odd with the length of at least 5 \todo{check in the morning}, it follows from \cref{lem:Roussel-Rubio} that we have $x, y \in X$ with only two edges to $P$ being $xp_2$ and $yp_{n-1}$. This gives us an odd hole: $v_2, x, p_2, \ldots, p_{n-1}, y, v_2$.
	\end{itemize}
\end{proof}

\subsubsection{Configurations of type $\T_3$}

\begin{wrapfigure}{r}{0.35\textwidth}
	\centering\begin{tikzpicture}[scale=.7,simplegraph]
  \def\ngon{6}
  \node[regular polygon,regular polygon sides=\ngon,minimum size=3cm, draw=none] (p) {};
  \foreach\x in {1,...,\ngon}{\node[] (p\x) at (p.corner \x){$v_\the\numexpr\intcalcMod{\x+3}{6}+1$};}

  \draw (p3) to (p4);
  \draw (p5) to (p6);
  \draw (p3) to (p6);
  \draw (p4) to (p5);
  \draw (p5) to (p1);
  \draw (p6) to (p2);

  \draw[dashed] (p3) to (p5);
  \draw[dashed] (p4) to (p6);
  \draw[dashed] (p3) to (p1);
  \draw[dashed] (p4) to (p1);
  \draw[dashed] (p3) to (p2);
  \draw[dashed] (p4) to (p2);
  \draw[dashed] (p6) to (p1);

  \node[minimum size=4mm](x1) at (-1, -4.5){};
  \node[minimum size=4mm](x2) at (1, -4.5){};
  \node[draw,dotted,inner sep=3pt, circle,yscale=.5, fit={(x1) (x2)},label=below:{$X$}] {};

  \draw[dashed] (x1) to (x2);

  \draw (p3) to[out=-90] (x1);
  \draw (p3) to[out=-90, in=150] (x2);

  \draw (p4) to (x1);
  \draw (p4) to (x2);

  \draw (p1) to[out=0, in=0] (x1);
  \draw (p1) to[out=0, in=0] (x2);

  \draw[dashed] (p5) to (x1);
  \draw[dashed] (p6) to (x2);

  \tikzset{decoration={snake,amplitude=.6mm,segment length=4mm,
        post length=0mm,pre length=0mm}}
  \draw[decorate] (p1) to [out=90, in=0] (0, 4) to [out=180, in=90] (p2);
  \node[draw=none] at (0, 4.5) {$P$};

  \draw[dashed] (p4) to (0, 4.1);
  \draw[dashed] (x2) [out=100, in=-90] to (0, 4.1);
\end{tikzpicture}%
	\caption{An example of a $\T_3$.}%
	\vspace{-1.5cm}
\end{wrapfigure}

A configuration of type $\T_3$ is a sequence $v_1, \ldots, v_6$, $P$, $X$, such that:
\begin{itemize}
	\item $v_1, \ldots v_6$ are distinct vertices.
	\item $v_1v_2$, $v_3v_4$, $v_1v_4$, $v_2v_3$, $v_3v_5$, $v_4v_6$ are edges, and $v_1v_3$, $v_2v_4$, $v_1v_5$, $v_2v_5$, $v_1v_6$, $v_2v_6$, $v_4v_5$ are nonedges.
	\item $X$ is an anticomponent of the set of all $\{v_1, v_2, v_5\}$-complete vertices, and $v_3$, $v_4$ are not $X$-complete.
	\item $P$ is a path of $G \setminus ( X \cup \{v_1, v_2, v_3, v_4\} )$ between $v_5$ and $v_6$ and no vertex in $P*$ is $X$-complete or adjacent to $v_1$ or adjacent to $v_2$.
	\item If $v_5v_6$ is an edge, then $v_6$ is not $X$-complete.
\end{itemize}

The following algorithm with running time of $O(|V|^6)$ checks whether $G$ contains a configuration of type $\T_3$:

\begin{alg}
	\label{alg:t3}
	Input: A graph $G$.

	\noindent Output: Returns whether $G$ contains a configuration of type $T_3$ as an induced subgraph.
\end{alg}

\begin{algorithmic}[1]
	\mProcedure{Contains-T3}{$G$}
	\mForEach{$v_1, v_2, v_5 \in V(G)$, so that $v_1v_2$ is an edge \AND
		\lsx $v_1v_5, v_2v_5$ are nonedges}
		\ls $Y \gets$ the set of all $\{v_1, v_2, v_5\}$-complete vertices.
		\mForEach{$X$ -- an anticomponent of $Y$}
			\ls $F' \gets$ maximal connected subset containing $v_5$, such that $v_1, v_2$
			\lsx have no neighbors in $F'$ and no vertex of $F'\setminus\{v_5\}$ is $X$-complete.
			\ls $F'' \gets$ the set of all $X$-complete vertices that have a neighbor in 
			\lsx $F'$ and are nonadjacent to all of $v_1, v_2$ and $v_5$
			\ls $F \gets F' \cup F''$
			\mForEach{$v_4 \in V(G) \setminus\{v_1, v_2, v_5\}$, such that $v_4$ is adjacent to $v_1$
			\lsx and not to $v_2$ and $v_5$}
				\mIf{$v_4$ has a neighbor in $F$ and a nonneibhbor in $X$}
					\ls $v_6 \gets$ a neighbor of $v_4$ in $F$
					\mForEach{$v_3 \in V(G) \setminus\{v_1, v_2, v_4, v_5, v_6\}$}
						\mIf{$v_3$ is adjacent to $v_2, v_4, v_5$ and not adjacent to $v_1$ \lsx}
							\ls $P \gets$ a path from $v_6$ to $v_5$ with interior in $F'$
							\ls \RETURN \TRUE \Comment{$v_1, \ldots v_6, P, X$ is a $\T_3$}
						\mEndIf
					\mEndFor
				\mEndIf
			\mEndFor
		\mEndFor
	\mEndFor
	\ls \RETURN \FALSE
	\mEndProcedure
\end{algorithmic}

We will skip the proof that each graph containing $\T_3$ is not Berge, as it is quite technical. See section 6.7 of \cite{MC05} for the proof.

\begin{theorem}
	
\end{theorem}

To see that the \cref{alg:t3} has a running time of $O(|V|^6)$, let us note that for each triple $v_1, v_2, v_5$ we examine, of which there are $O(|V|^3)$, there are linear many choices of $X$, each taking $O(|V|^2)$ time to process and generating a linear many choices of $v_4$ which take a linear time to process in turn. This gives us the total running time of $O(|V|^6)$.

.

\subsection{Amenable holes.}
\label{AmenableHoles}

First, let us introduce a few new definitions.

\begin{defn}[C-major vertices]
  Given a shortest odd hole $C$ in $G$, a node $v \in V(G) \setminus V(C)$ is $C$-major if the set of its neighbors in $C$ is not contained in any 3-node path of $C$.
  \TODO{a picture of this, clean odd hole, amenable hole}
\end{defn}

\begin{defn}[clean odd hole]
  An odd hole $C$ of $G$ is \emph{clean} if no vertex in $G$ is $C$-major.
\end{defn}

\begin{defn}[cleaner]
  Given a shortest odd hole $C$ in $G$, a subset $X \subseteq V(G)$ is a \emph{cleaner for $C$} if $X \cap V(C) = \emptyset$ and every $C$-major vertex belongs to $X$.
\end{defn}

Let us notice, that if $X$ is a cleaner for $C$ then $C$ is a clean hole in $G \setminus X$.

\begin{defn}[near-cleaner]
  Given a shortest odd hole $C$ in $G$, a subset $X \subseteq V(G)$ is a \emph{near-cleaner for $C$} if $X$ contains all $C$-major vertices and $X \cap V(C)$ is a subset of vertex set of some 3-node path of $C$.
\end{defn}

\begin{defn}[amenable odd hole]
  An odd hole $C$ of $G$ is \emph{amenable} if it is a shortest off hole in $G$, it is of length at least 7 and for every anticonnected set $X$ of $C$-major vertices there is a $X$-complete edge in $C$.
\end{defn}

\begin{theorem}
	\label{thm:amenableHoles}
	Let $G$ be a graph, such that $G$ and $\overline{G}$ contain no Pyramid, no Jewel and no configuration of types $\T_1, \T_2$ or $\T_3$. Then every shortest hole in $G$ is amenable.
	% \TODO{Any ideas on what else to say here? List all 9 steps?}
	% \TODO{Dobre pytanie. Te 9 kroków to da się zawrzeć w 1-2 zdaniach każde? Jeśli nie, to tylko napisałbym ogólnikowo, że "jest sekwencja kroków, która to robi, tu jest odnośnik"}
\end{theorem}

The proof of this theorem is quite technical and we will not discuss it here. See section 8 of \cite{MC05} for the proof.

With \Cref{thm:amenableHoles} we can describe the rest of the algorithm.

\begin{alg}[List possible near cleaners, 9.2 of \cite{MC05}]
	\label{alg:listNearCleaners}
	Input: A graph $G$.

	\noindent Output: $O(|V|^5)$ subsets of $V(G)$, such that if $C$ is an amenable hole in $G$, then one of the subsets is a near-cleaner for $C$.
\end{alg}
\begin{algtext2}
	We will call a triple $(a, b, c)$ of vertices \emph{relevant} if $a \neq b$ (but possibly $c \in \{a, b\}$) and $G[\{a,b,c\}]$ is an independent set.

	Given a relevant triple $(a, b, c)$ we can compute the following:
	\begin{itemize}
		\item $r(a,b,c) \leftarrow$~the cardinality of the largest anticomponent of $N(a, b)$, that contains a nonneibhbor of $c$, or 0, if $c$ is $N(a, b)$-complete.
		\item $Y(a,b,c) \leftarrow$~the union of all anticomponents of $N(a, b)$ that have cardinality strictly greater than $r(a, b, c)$.
		\item $W(a, b, c) \leftarrow$~the anticomponent of $N(a,b) \cup \{c\}$ that contains $c$.
		\item $Z(a, b, c) \leftarrow$~the set of all $Y(a, b, c) \cup W(a,b,c)$-complete vertices.
		\item $X(a, b, c) \leftarrow$~$Y(a,b,c) \cup Z(a,b,c)$.
	\end{itemize}

	For every two adjacent vertices $u, v$ compute the set $N(u, v)$ and list all such sets.
	For each relevant triple $(a,b,c)$ compute the set $X(a,b,c)$ and list all such sets.

	Output all subsets of $V(G)$ that are the union of a set from the first list and a set from the second list.
\end{algtext2}

To prove the correctness of \cref{alg:listNearCleaners} we will need the following theorem.

\begin{theorem}[9.1 of \cite{MC05}]
	\label{thm:91}
	Let $C$ be a shortest odd hole in $G$, with length at least 7. Then there is a relevant triple $(a, b, c)$ of vertices such that
	\begin{itemize}
		\item the set of all $C$-major vertices not in $X(a, b, c)$ is anticonnected
		\item $X(a, b, c) \cap V(C)$ is a subset of the vertex set of some 3-vertex path of $C$.
	\end{itemize}
\end{theorem}

Let us suppose that $C$ is an amenable hole in $G$. By \cref{thm:91}, there is a relevant triple $(a, b, c)$ satisfying that theorem. Let $T$ be the set of all $C$-major vertices not in $X(a,b,c)$. From \cref{thm:91} we get that $T$ is anticonnected. Since $C$ is amenable, there is an edge $uv$ of $C$ that is $T$-complete, and therefore $T \subseteq N(u, v)$. But then $N(u, v) \cup X(a, b, c)$ is a near-cleaner for $C$. Therefore the output of the \cref{alg:listNearCleaners} is correct.

\begin{alg}[Test possible near cleaner, 5.1 of \cite{MC05}]
	\label{alg:testNearCleaner}
	Input: A graph $G$ containing no simple forbidden structure, and a subset $X \subseteq V(G)$.

	\noindent Output: Determines one of the following:
	\begin{itemize}
		\item $G$ has an odd hole
		\item There is no shortest odd hole $C$ such that $X$ is a near-cleaner for $C$.
	\end{itemize}
\end{alg}
\begin{algtext2}
	For every pair $x, y \in V(G)$ of distinct vertices find shortest path $R(x, y)$ between $x, y$ with no internal vertex in $X$. If there is one, let $r(x, y)$ be its length, if not, let $r(x, y) = \infty$.

	For all $y_1 \in V(G)\setminus X$ and all 3-vertex paths $x_1-x_2-x_3$ of $G\setminus y_1$ we check the following:
	\begin{itemize}
		\item $R(x_1, y_1), R(x_2, y_2)$ both exist -- define $y_2$ as the neighbor of $y_1$ in $R(x_2, y_1)$.
		\item $r(x_2, y_1) = r(x_1, y_1) + 1 = r(x_1, y_2)$ ($=n$ say) 
		\item $r(x_3, y_1), r(x_3, y_2) \geq n$
	\end{itemize}
	\TODO{moze raczej $min(r(x_3, y_1), r(x_3, y_2)) = r(x_2, y_1) = r(x_1, y_2) = r(x_1, y_1) + 1$?}

	If there is such a choice of $x_1$, $x_2$, $x_3$, $y_1$ then we output that there is an odd hole. If not, we report that there is no shortest odd hole $C$ such that $X$ is a near-cleaner for $C$.
\end{algtext2}

\begin{wrapfigure}{r}{0.35\textwidth}
	\centering\begin{tikzpicture}[scale=0.7,simplegraph]
  \node(x1) at (0,0) {$x_1$};
  \node(x3) at (1.5,0) {$x_3$};
  \node(x2) at (3,0) {$x_2$};

  \node (y1) at (.6, 3) {$y_1$};
  \node (y2) at (2.4, 3) {$y_2$};

  \draw (x1) to (x3);
  \draw (x2) to (x3);

  \tikzset{decoration={snake,amplitude=.6mm,segment length=4mm,
        post length=0mm,pre length=0mm}}
  \draw[decorate, color=c1] (x1) to (y1);
  \draw[decorate, color=c2] (x2) to (y2);
  \draw[color=c2] (y1) to (y2);

  \node[draw=none, color=c2] at (3.8, 2.5) {$R(x_2, y_1)$};
  \node[draw=none, color=c1] at (1.4, 1.5) {\small$R(x_1, y_1)$};

  \node[draw=none, minimum size=4mm](inx1) at (.8, -1.1){};
  \node[draw=none, minimum size=4mm](inx2) at (2.2, -1.1){};
  \node[draw,inner sep=3pt, circle,yscale=.5, fit={(inx1) (inx2)},label=below:{$X$}] {};
  \node[draw,dotted,inner sep=5pt, circle,yscale=.5, fit={(inx1) (inx2) (x3) (x1) (x2)}] {};

  % \draw[dashed] (x1) to (x2);
  % \draw(v1) to (x1);
  % \draw(v2) to (x1);
  % \draw(v4) to (x1);
  % \draw(v1) to (x2);
  % \draw(v2) to (x2);
  % \draw(v4) to (x2);

  % \draw (a) to (b2);
\end{tikzpicture}%
	\caption{An odd hole is found}%
	\vspace{-0.5cm}
\end{wrapfigure}

Below we will prove that if the \cref{alg:testNearCleaner} reports an odd hole in $G$, there indeed is one. The proof of the correctness of the other possible result is more complicated, see section 4 and theorem 5.1 of \cite{MC05}.

\begin{proof}
	Let us suppose that there is a choice of $x_1$, $x_2$, $x_3$, $y_1$ satisfying the three conditions in the \cref{alg:testNearCleaner} and let $y_2$ and $n$ be defined as in there. We claim that $G$ contains an odd hole.

	Let $R(x_1, y_1) = p_1-\ldots -p_n$, and let $R(x_2, y_1) = q_1-\ldots -q_{n+1}$, where $p_1 = x_1$, $p_n = q_{n+1} = y_1$, $q_1 = x_2$ and $q_n = y_2$. From the definition of $R(x_1, y_1)$ and $R(x_2, y_1)$, none of $p_2, \ldots, p_{n-1}, q_2, \ldots, q_n$ belong to $X$. Also, from the definition of $y_1$, $y_1 \notin X$.

	Since $r(x_1, y_1) = r(x_2, y_1) - 1$ and since $x_1$, $x_2$ are nonadjacent it follows that $x_2$ does not belong to $R(x_1, y_1)$ and $x_1$ does not belong to $R(x_2, y_1)$. Since $r(x_3, y_1), r(x_3, y_2) \geq n (= r(x_1, y_2))$ it follows that $x_3$ does not belong to $R(x_1, y_1)$ or to $R(x_2, y_1)$, and has no neighbors in $R(x_1, y_1) \cup R(x_2, y_1)$ other than $x_1$, $x_2$. Since $r(x_1, y_2) = n$ we get that $y_2$ does not belong to $R(x_1, y_1)$.

	We claim first that the insides of paths $R(x_1, y_1)$ and $R(x_2, y_1)$ have no common vertices. For suppose that there are $2 \leq i \leq n-1$ and $2 \leq j \leq n$ that $p_i = q_j$. Then the subpaths of these two paths between $p_i$, $y_1$ are both subpaths of the shortest paths, and therefore have the same length, that is $j=i+1$. So $p_1-\ldots-p_1-q_{j+1}-\ldots-q_n$ contains a path between $x_1$, $y_2$ of length $\leq n-2$, contradicting that $r(x_1, y_2) = n$. So $R(x_1, y_1)$ and $R(x_2, y_1)$ have no common vertex except $y_1$.

	If there are no edges between $R(x_1, y_1) \setminus y_1$ and $R(x_2, y_1) \setminus y_1$ then the union of these two paths and a path $x_1-x_3-x_2$ form an odd hole, so the answer is correct.

	Suppose that $p_iq_j$ is an edge for some $1\leq i \leq n-1$ and $1 \leq j \leq n$. We claim $i \geq j$. If $j = 1$ this is clear so let us assume $j > 1$. Then there is a path between $x_1$, $y_2$ within $\{p_1, \ldots , p_i, q_j, \ldots q_n\}$ which has length $\leq n-j+1$ and has no internal vertex in $X$ (since $j > 1$); and since $r(x_1, y_2) = n$, it follows that $n-j+i \geq n$, that is, $i \geq j$ as claimed.

	Now, since $x_1$, $x_2$ are nonadjacent $i \geq 2$. But also $r(x_2, y_1) \geq n$ and so $j+n-i \geq n$, that is $j \geq i$. So we get $i=j$. Let us choose $i$ minimum, then $x_3-x_1-\ldots-p_i-q_i-\ldots-x_2-x_3$ is an odd hole, which was what we wanted.
\end{proof}

\subsection{Summary}

\TODO{todo?}
A natural problem for perfect graphs is a problem of coloring them. In 1988 Gr\"otschel et al. published an ellipsoid-method-based polynomial algorithm for coloring perfect graphs \cite{Grtschel1993}. We consider it in \cref{sec:coloringEllipsoid}. However due to its use of the ellipsoid method this algorithm has been usually considered unpractical \cite{coloringSquareFree,Chudnovsky2003, coloringArtemis}.

There has been much progress on the quest of finding a more classical algorithm coloring perfect graphs, without the use of ellipsoid method (see \cref{sec:classicalColoring}), however there is still no known polynomial combinatorial algorithm to do this. \todo{better wording of this paragraph}

\section{Information theory background}
\label{sec:InformationTheory}

\Cref{sec:InformationTheory} and \cref{sec:computingTheta} are based on lecture notes by Lovász \cite{Lovasz95}.

The polynomial technique of coloring perfect graphs known so far arose in the field of semidefinite programming. Semidefinite programs are linear programs over the cones of semi-definite matrices. The connection of coloring graphs and the cones of semi-definite matrices might be surprising, so let us take a brief digression into the field of information theory, where we will see the connection more clearly. Also, this was exactly the background which motivated Berge to introduce perfect graphs \cite{Chudnovsky2003}.

\subsection{Shannon Capacity of a graph}
\label{sec:ShannonCapacity}

\begin{wrapfigure}{r}{0.35\textwidth}
  \centering\begin{tikzpicture}[scale=.7,simplegraph]
  \def\ngon{5}
  \node[regular polygon,regular polygon sides=\ngon,minimum size=3cm, draw=none] (p) {};
  \foreach\x in {1,...,\ngon}{\node[] (p\x) at (p.corner \x){$\the\numexpr\intcalcMod{\x}{5}$};}

  \draw (p1) to (p2);
  \draw (p2) to (p3);
  \draw (p3) to (p4);
  \draw (p4) to (p5);
  \draw (p5) to (p1);

\end{tikzpicture}%
  \caption{An example of a noisy channel}%
  \label{fig:c5}
  % \vspace{-0.5cm}
\end{wrapfigure}

Suppose we have a noisy communication channel in which certain signal values can be confused with others. For instance, suppose our channel has five discrete signal values, represented as 0, 1, 2, 3, 4. However, each value of $i$ when sent across the channel can be confused with value $(i \pm 1)$ mod $5$. This situation can be modeled by a graph $C_5$ (\cref{fig:c5}) in which vertices correspond to signal values and two vertices are connected if and only if values they represent can be confused.

We are interested in transmission without possibility of confusion. For this example it is possible for two values to be transmitted without ambiguity e.g. values 1 and 4, which allows us to send $2^n$ non-confoundable messages in $n$ steps. But we could do better, for example we could communicate five two-step codewords e.g. "00", "12", "24", "43", "31". Each pair of these codewords includes at least one position where its values differ by two or more modulo 5, which allows the recipient to distinguish them without confusion.  This allows us to send $5^{n / 2}$ non-confoundable messages in $n$ steps.

Let us be more precise. Given a graph $G$ modeling a communication channel and a number $k \geq 1$ we say that two messages $v_1v_2\ldots v_k$, $w_1w_2\ldots w_k \in V(G)^k$ of length $k$ are non-confoundable if and only if there is $1 \leq i \leq k$ such that $v_i$, $w_i$ are non-confoundable. We are interested in the maximum rate at which we can reliably transmit information (the \emph{Shannon capacity} of the channel defined by $G$).

For $k = 1$, maximum number of messages we can send without confusion in a single step is equal to $\alpha(G)$. To describe longer messages we use \emph{strong product} $G \cdot H$ of two graphs $G = (V, E)$, $H = (W, F)$ as the graph with $V(G \cdot H) = V \times W$, with $(i, u)(j, v) \in E(G \cdot H)$ if and only if $ij \in E$ and $uv \in F$, or $ij \in E$ and $u = v$, or $i = j$ and $uv \in F$. Given channel modeled by $G$ it is easy to see that the maximum number of distinguishable words of length 2 is equal to $\alpha(G \cdot G)$, and in general the number of distinguishable words of length $k$ is equal to $\alpha(G^k)$ -- which gives us $\sqrt[k]{\alpha(G^k)}$ as the number of distinguishable signals per single transmission. So, we can define the Shannon capacity of the channel defined by $G$ as $\Theta(G) = \sup\limits_k \sqrt[k]{\alpha(G^k)}$.

Unfortunately, it is not known whether $\Theta(G)$ can be computed for all graphs in finite time. If we could calculate $\alpha(G^k)$ for a first few values of $k$ (we will show how to do it in \cref{alg:maxStableSet}) we could have a lower bound on $\Theta(G)$. Let us now turn into search for some usable upper bound.

\subsection{Lovász number}

For a channel defined by a graph $C_5$, using five messages of length 2 to communicate gives us a lower bound on $\Theta(G)$ equal $\sqrt{5}$ (as does calculating $\alpha(C_5^2)$).

Consider an "umbrella" in $\mathbb{R}^3$ with the unit vector $e_1 = (1, 0, 0)$ as its "handle" and 5 "ribs" of unit length \todo{picture}. Open it up to the point where non-consecutive ribs are orthogonal, that is form an angle of 90$^\circ$. This way we get a representation of $C_5$ by 5 unit vectors $u_1, \ldots u_5$ so that each $u_i$ forms the same angle with $e_1$ and any two non-adjacent nodes are represented with orthogonal vectors. We can calculate $e_1^\intercal u_i = 5 ^ {-1/4}$.

It turns out, that we can obtain a similar representation of the nodes of $C_5^k$ by unit vectors $v_i \in \mathbb{R}^{3k}$, so that any two non-adjacent nodes are labeled with orthogonal vectors (this representation is sometimes called the \emph{orthogonal representation} \cite{Lovsz1989Orthogonal}). Moreover, we still get $e_1^\intercal v_i = 5^{-k/4}$ for every $i \in V(C_5^k)$ (the proof is quite technical and we omit it here).

If $S$ is any stable set in $C_5^k$, then $\{v_i, i \in S\}$ is a set of mutually orthogonal unit vectors so we get \todo{why?}
$$\sum\limits_{i\in S}(e_1^\intercal v_i)^2 \leq |e_1|^2 = 1$$
(if $v_i$ formed a basis then this inequality would be an equality).

On the other hand each term on the left hand side is $5^{-1/4}$, so the left hand side is equal to $|S|5^{-k/2}$, and so $|S| \leq 5^{k/2}$. Since $|S|$ was an arbitrary stable set, we get $\alpha(C_5^k) \leq 5 ^{k/2}$ and $\Theta(C_5) = \sqrt{5}$.

It turns out that this method extends to any graph $G$ in place of $C_5$. All we have to do is find a orthogonal representation that will give us the best bound. So, we can define the \emph{Lovász number} of a graph $G$ as \todo{this equation does not really follow from the thought process above, it is a slightly different definition}:
$$\vartheta(G) = \min\limits_{c,U} \max\limits_{i\in V} \frac{1}{(c^\intercal u_i)^2},$$

\noindent where $c$ is a unit vector in $\mathbb{R}^{|V(G)|}$ and $U$ is a orthogonal representation of $G$.

Contrary to Lovász's first hope \cite{Lovasz1979} $\vartheta(G)$ does not always equal $\Theta(G)$, it is only an upper bound on it. However, these two are equal for some graphs, including all perfect graphs, as is demonstrated in the Lovász "sandwich theorem".

\begin{theorem}[Lovász "sandwich theorem" \cite{Knuth1994}]
  \label{thm:sandwich}
  For any graph $G$:
  $$ \omega(G) \leq \vartheta(\overline{G}) \leq \chi(G) $$
\end{theorem}

Because in perfect graphs $\omega(G) = \chi(G)$, we get $\omega(G) = \vartheta(\overline{G}) = \chi(G)$. Therefore, if for any perfect graph $G$, we could calculate $\vartheta(G)$ and $\vartheta(\overline{G})$, we would get $\omega(G)$, $\chi(G)$ and $\alpha(G)$.

But how can we construct an optimum (or even good) orthogonal representation? It turns out that it can be computed in polynomial time using semidefinite optimization.

\section{Computing \boldmath$\vartheta$}
\label{sec:computingTheta}

First, let us recall some definitions, with \cite{gilbertstrang2020} as a reference for linear algebra.

\begin{defn}[eigenvector, eigenvalue]
  Let $A$ be an $n \times n$ real matrix. An \emph{eigenvector} of $A$ is a vector $x$ such that $Ax$ is parallel to $x$. In other words, there is a real or complex number $\lambda$, such that $Ax = \lambda x$. This $\lambda$ is called the \emph{eigenvalue} of $A$ belonging to eigenvector $x$.
\end{defn}

If a matrix $A$ is symmetric\footnote{Matrix $A$ is symmetric if and only if $A = A^\intercal$}, all the eigenvalues are real.

\begin{defn}[positive semidefinite matrix]
  Let $A$ be an $n \times n$ symmetric matrix. $A$ is \emph{positive semidefinite} if all of its eigenvalues are nonnegative. We denote it by $A \succeq 0$.
\end{defn}

We have equivalent definitions of semidefinite matrices.
\begin{theorem}
  For a real symmetric $n \times n$ matrix $A$, the following are equivalent:
  \begin{enumerate}[(i)]
    \item $A$ is positive semidefinite,
    \item \label{en:ei2} for every $x \in \mathbb{R}^n$, $x^\intercal Ax$ is nonnegative,
    \item for some matrix $U$, $A = U^\intercal U$,
    \item $A$ is a nonnegative linear combination of matrices of the type $xx^\intercal$.
  \end{enumerate}
\end{theorem}

From (\ref{en:ei2}) it follows that diagonal entries of any positive semidefinite matrix are nonnegative and the sum of two positive semidefinite matrices is positive semidefinite.

We may think equivalently of $n \times n$ matrices as vectors with $n^2$ coordinates.
\begin{defn}[convex cone]
  A subset $C$ of $\mathbb{R}^n$ is a \emph{convex cone}, if for any positive scalars $\alpha, \beta$ and for any $x, y \in C$, $\alpha x + \beta y \in C$.
\end{defn}

The fact that the sum of two positive semidefinite matrices is again positive semidefinite, with the fact that every positive scalar multiple of a positive semidefinite matrix is positive semidefinite, translates into the geometric statement that the set of all positive semidefinite matrices forms a convex closed cone $\mathcal{P}_n$ in $\mathbb{R}^{n \times n}$ with vertex 0. This cone $\mathcal{P}_n$ is important but its structure is not trivial.

\paragraph{Semidefinite programs.}

Now, we can define a \emph{semidefinite program} to be an optimization problem of the following form:

\begin{equation*}
  \begin{array}{ll@{}ll}
    \text{minimize}   & c^\intercal x                          & \\
    \text{subject to} & x_1A_1 + \ldots + x_nA_n - B \succeq 0   \\
  \end{array}
\end{equation*}
Here $A_1, \ldots, A_n, B$ are given symmetric $m \times m$ matrices and $c \in \mathbb{R}^n$ is a given vector. Any choice of the values $x_i$ that satisfies the given constraint is called a \emph{feasible solution}.

The special case when $A_1, \ldots A_n, B$ are diagonal matrices is a ''generic'' linear program, in fact we can think of semidefinite programs as generalizations of linear programs. Not all properties of linear programs are carried over to semidefinite programs, but the intuition is helpful.

Solving semidefinite programs is a complex topic, we refer to \cite{grotschel1993} for reference. All we need to know is that we can solve semidefinite programs up to an arbitrarily small error in polynomial time. One of the methods to do this is called the \emph{ellipsoid method}, hence the name for the coloring algorithm.

\paragraph{Calculating \boldmath$\vartheta$.}

Let us recall, that an orthogonal representation of a graph $G = (V, E)$ is a labeling $u: V \rightarrow \mathbb{R}^d$ for some $d$, such that $u_i^\intercal u_j = 0$ for all nonedges $ij$. An \emph{orthonormal} representation is an orthogonal representation with $|u_i| = 1$ for all $i$. The \emph{angle} of an orthogonal representation is the smallest half-angle of a rotational cone containing the representing vectors.

\begin{theorem}[Proposition 5.1 of \cite{Lovasz95}]
  The minimum angle $\phi$ of any orthogonal representation of $G$ is given by $\cos^2\phi = 1/\vartheta(G)$.
\end{theorem}
\TODO{mark all theorems everywhere accordingly}

This leads us to definition of $\vartheta(G)$ in terms of semidefinite programming.

\begin{theorem}[Proposition 5.3 of \cite{Lovasz95}]
  $\vartheta(G)$ is the optimum of the following semidefinite program:

  \begin{equation*}
    \begin{array}{ll@{}ll}
      \text{minimize}   & t      &                                              \\
      \text{subject to} & Y      & \succeq 0                                    \\
                        & Y_{ij} & = -1      & (\forall~ij \in E(\overline{G})) \\
                        & Y_{ii} & = t - 1                                      \\
    \end{array}
  \end{equation*}

  It is also the optimum of the dual program

  \begin{equation*}
    \begin{array}{l@{}rlll}
      \text{maximize~~}   & \sum_{i \in V} \sum_{j \in V}  Z_{ij} &                                     \\
      \text{subject to~~} & Z                                     & \succeq & 0                         \\
                          & Z_{ij}                                & =       & 0 & (\forall~ij \in E(G)) \\
                          & tr(Z)                                 & =       & 1                         \\
    \end{array}
  \end{equation*}
\end{theorem}

Any stable set $S$ of $G$ provides a feasible solution of the sual program, by choosing $Z_{ij} = \frac{1}{S}$, if $i, j \in S$ and 0 otherwise. Similarly, any $k$-coloring of $\overline{G}$ provides a feasible solution of the former semidefinite program, by choosing $Y_{ij} = -1$ if $i$ and $j$ have different colors, $Y_{ii} = k-1$, and $Y_{ij} = 0$ otherwise.

Now, that we know how to calculate $\vartheta(G)$, let us describe the algorithm to calculate the coloring of $G$.

\section{Coloring perfect graph using ellipsoid method}
\label{sec:coloringEllipsoid}
The following is based on \citetitle{Laurent2005} by \citeauthor{Laurent2005} \cite{Laurent2005}.

\subsection{Maximum cardinality stable set}

Given graph $G$, recall that stability number of $G$ is equal clique number of the complement of $G$. This gives us a way to compute $\alpha(G)$ for any perfect graph $G$.

In fact, to calculate $\chi(\overline{G})$ and $\alpha(G)$ we only need an approximated value of $\vartheta(G)$ with precision smaller than $1/2$, as the former values are always integral.

We will now show how to find a stable set in $G$ of size $\alpha(G)$.

\begin{alg}[maximum cardinality stable set in a perfect graph]
  \label{alg:maxStableSet}
  Input: A perfect graph $G = (V, E)$.

  \noindent Output: A maximum cardinality stable set in $G$.
\end{alg}
\begin{algtext2}
  Let $v_1, \ldots v_n$ be an ordering of vertices of $G$. We will construct a sequence of induced subgraphs $G = G_0 \supseteq G_1 \supseteq \ldots \supseteq G_n$, so that $G_n$ is a required stable set.

  Let $G_0 \leftarrow G$. Then, for each $i \geq 1$, compute $\alpha(G_{i-1} \setminus v_i)$. If $\alpha(G_{i-1} \setminus v_i) = \alpha(G)$, then set $G_i \leftarrow G_{i-1} \setminus \{v_i\}$, else set $G_i \leftarrow G_{i-1}$.

  Return $G_n$.
\end{algtext2}

Let us prove that $G_n$ is indeed a stable set. Suppose otherwise and let $v_iv_j$ be an edge in $G_n$ with $i < j$ and $i$ minimal. But then $\alpha(G_{i-1} \setminus v_i) = \alpha(G_{i-1}) = \alpha(G)$ so by our construction $v_i$ is not in $G_i$ and $v_iv_j$ is not an edge of $G_n$. Therefore there are no edges in $G_n$.

Because at every step we have $\alpha(G_i) = \alpha(G_{i-1})$, therefore $\alpha(G_n) = \alpha(G)$, so $G_n$ is required maximum cardinality stable set.

The running time of \cref{alg:maxStableSet} is polynomial, because we construct $n$ auxiliary graphs, each requiring calculating $\alpha$ once plus additional $O(|V|^2)$ time for constructing the graph.

Given a weight function $w : V \rightarrow \mathbb{N}$ we could calculate the maximum weighted stable set in $G$ in the following manner. Create graph $G'$ by replacing every node $v$ by a set $W_v$ of $w(v)$ nonadjacent nodes, making two nodes $x \in W_v$, $y \in W_u$ adjacent in $G'$ if and only if the nodes $v$, $u$ are adjacent in $G$. Then calculate a maximum cardinality stable set in $G'$ (we remark that $G'$ is still perfect because every new introduced hole is even) and return a result of those vertices in $G$ whose any (and therefore all) copies were chosen. We will use this technique later on.

\subsection{Stable set intersecting all maximum cardinality cliques}
Next, let us show how to find a stable set intersecting all the maximum cardinality cliques of $G$.
\begin{alg}
  \label{alg:ssIntersectingCliques}
  Input: A perfect graph $G = (V, E)$.

  \noindent Output: A stable set which intersects all the maximum cardinality cliques of $G$.
\end{alg}
\begin{algtext2}
  We will create a list $Q_1, \ldots Q_t$ of all maximum cardinality cliques of $G$.

  Let $Q_1 \leftarrow$ a maximum cardinality clique of $G$. We calculate this by running \cref{alg:maxStableSet} on $\overline{G}$.

  Now suppose $Q_1, \ldots, Q_t$ have been found. We show how to calculate $Q_{t+1}$ or see that we are done.

  Let us define a weight function $w : V \rightarrow \mathbb{N}$, so that for $v \in V$, $w(v)$ is equal to the number of cliques $Q_1, \ldots Q_t$ that contain $v$.

  Assign $S \leftarrow$ the maximum $w$-weighted stable set, as described in a remark to \cref{alg:maxStableSet}. It is easy to see that $S$ has weight $t$, which means that $S$ meets each of $Q_1, \ldots Q_t$.

  If $\omega(G \setminus S) < \omega(G)$, then $S$ meets all the maximum cardinality cliques in G so we return $S$. Otherwise we find a maximum cardinality clique in $G \setminus S$ (it will be of size $\omega(G)$, because $\omega(G \setminus S) = \omega(G)$), add it to our list as $Q_{t+1}$ and continue with longer list.

\end{algtext2}

There are at most $|V|$ maximum cardinality cliques in $G$. Adding a single clique to the list of maximum cardinality cliques requires constructing auxiliary graph for weighted maximum stable set, which is of size $O(|V|^2)$ and running \cref{alg:maxStableSet} on it. Therefore total running time is polynomial.

\subsection{Minimum coloring}

\begin{alg}
  \label{alg:minColoring}
  Input: A perfect graph $G = (V, E)$.

  \noindent Output: A coloring of $G$ using $\chi(G)$ colors.
\end{alg}
\begin{algtext2}
  If $G$ is equal to its maximum cardinality stable set, color all vertices one color and return.

  Else find $S$ intersecting all maximum cardinality cliques of $G$ (\cref{alg:ssIntersectingCliques}). Color recursively all vertices of $G \setminus S$ with $\chi(G \setminus S) = \omega(G \setminus S) = \omega(G) -1$ colors and all vertices of $S$ with one additional color.
\end{algtext2}

We will call recursion at most $O(|V|)$ times, each step of recursion is polynomial in time. Therefore the running time of \cref{alg:minColoring} is polynomial.

\section{Classical algorithms}
\label{sec:classicalColoring}

Ever since Grötschel et al. published an ellipsoid-method-based polynomial algorithm for coloring perfect graphs, a combinatorial algorithm for doing the same has been sought. As of yet, it is not known, although there is much progress in the field.

\begin{defn}[prism]
  A \emph{prism} is a graph consisting of two disjoint triangles and two disjoint paths between them. Notice, that for a graph to contain no odd hole, all three paths in a prism must have the same parity. A prism with all three paths odd is called an \emph{odd prism}.
  \label{def:prism}
\end{defn}

In 2005 Maffray and Trotignon a coloring algorithm that colors graphs containing no odd hole, no antihole and no prism (sometimes called Artemis graphs) in $O(|V|^4|E|)$ time \cite{Maffray2006}. They later improved the time complexity to $O(|V|^2|E|)$ \cite{Lvque2009}.

In 2015 Maffray showed an algorithm for coloring Berge graphs with no squares (a square is a $C_4$) and no odd prism \cite{Maff2015}.

In 2016 Chudnovsky et al. published an algorithm that given a perfect graph $G$ with $\omega(G) = k$ colors it optimally in a time polynomial for a fixed $k$ \cite{Chudnovsky2017}.

A most recent advancement (2018) is an algorithm by Chudnovsky et al. that colors any square-free Berge graphs in time of $O(|V|^9)$ \cite{Chudnovsky2019}. Before proving strong perfect graph conjecture, a similar conjecture for square-free Berge graphs has been proven by Conforti et al. \cite{Conforti2004} During one of her lectures, Maria Chudnovsky expressed hope that discovery of full algorithm for coloring Berge graphs might follow a similar pattern. We analyze this algorithm and provide its pseudocode in \cref{ch:coloringSquareFree}.

\TODO{Dałbym ze 2-3 zdania braggowania, ze algorytm jest duzo bardziej zlozony niz rozpoznawanie + na czym sie opiera.}

\chapter{Implementation}

\section{Berge graphs recognition}

Anything interesting about algo/data structure?\\

\subsection{Optimizations}\label{Optimizations}
Bottlenecks in performance (next path, are vectors distinct etc).\\
\TODO{In our graph preprocessing we have a pointer to next edge in order to speed up generating next path.}
\TODO{We used callgrind to get idea of methods crucial for time.}
\TODO{In general enumerating all paths is crucial. As is checking if vector has distinct values.}
\TODO{Jewels -- we iterate all possibly chordal paths and check if they are ok - much faster}
\TODO{$\T_1$ -- we iterate all paths of length 4 and check if there exists a fifth vertex to complete the hole - much faster than iterating vertices.}

Validity tests - unit tests, tests of bigger parts, testing vs known answer and vs naive.

\subsection{Parallelism with CUDA (?)}

TODO

\subsection{Experiments}

Naive algorithm - brief description, bottlenecks optimizations (makes huge difference).\\

Description of tests used.\\

Results and Corollary - almost usable algorithm.


\section{Coloring Berge Graphs}

\subsection{Ellipsoid method}

Description.\\

Implementation.\\

Experiments and results.\\

\subsection{Combinatorial Method}

Cite the paper.\\

On its complexity - point to appendix for pseudo-code.

\appendix
\appendixpage
\addappheadtotoc

\chapter{Perfect Graph Coloring algorithm}
TODO


% \bibliographystyle{alpha}
\printbibliography

\end{document}