\documentclass{article}

\usepackage{appendix}
\usepackage{todonotes}

\usepackage[
	style=alphabetic
]{biblatex}
\addbibresource{bbl.bib}


\newcommand{\TODO}{\todo[inline]}
\newcommand\Lovasz{Lovász }

% Language and typesetting notes:
%
% - Color, not Colour (American English)
% - Node not a Vertex of a graph

\author{Adrian Siwiec}
\date{\today{}}
\begin{document}
\begin{titlepage}
	\begin{center}
        
		\large
		\textbf{Jagiellonian University}\\
		Department of Theoretical Computer Science\\

		\vspace{1.5cm}

		\Large
		\textbf{Adrian Siwiec}

		\vspace*{2cm}

		\textbf{\LARGE Perfect Graph Recognition and Coloring}
		
		\vspace{0.5cm}
		\large
		
		\vfill
		\Large
		Master Thesis

		\vfill
		\Large
		Supervisor: dr inż. Krzysztof Turowski
		
		\vspace{0.8cm}
		
		September 2020
		
\end{center}
\end{titlepage}

\pagebreak

\begin{abstract}
TODO
\end{abstract}

\tableofcontents

\pagebreak

\section{Perfect Graphs}
All graphs in this paper are finite, undirected and have no loops or parallel edges. We denote the chromatic number of graph $G$ by $\chi(G)$ and the cardinality of the largest clique of $G$ by $\omega(G)$. \emph{Coloring} of a graph means assigning every node of a graph a color. A coloring is \emph{valid} iff every two nodes sharing an edge have different colors. An \emph{optimal} coloring (if exists) is a valid coloring using only $\omega(G)$ colors.

Given a graph $G = (V,E)$ and a set $X \subseteq V$ by $G[X]$ we will denote a graph induced on $X$. A graph $G = (V,E)$ is \emph{perfect} iff for all $X \subseteq V$ we have $\chi(G[X]) = \omega(G[X])$.

\TODO{Give some examples why are these interesting, some subclasses, and problems that are solvable for perfect graphs, including recognition and coloring}

Given a graph $G$, its \emph{complement} $\overline{G}$ is a graph with the same vertex set and in which two distinct nodes $u, v$ are connected in $\overline{G}$ iff they are not connected in $G$. For example a clique in a graph becomes an independent set in its complement. A perfect graph theorem, first conjured by Berge in 1961 \cite{CB61} and then proven by \Lovasz in 1972 \cite{LL72} states that a graph is perfect iff its complement graph is also perfect. \todo{Should we give some proof of that here?}

\subsection{Strong Perfect Graph Theorem}
A \emph{hole} is an induced chordless cycle of length at least 4. An \emph{antihole} is an induced subgraph whose complement is a hole. A \emph{Berge} graph is a graph with no holes or antiholes of odd length.

In 1961 Berge conjured that a graph is perfect iff it is Berge in what has become known as a strong perfect graph conjecture. In 2001 Chudnovsky et al. have proven it and published the proof in an over 150 pages long paper \citetitle{MC06} \cite{MC06}.


\section{Recognizing Berge Graphs}
Cite the paper.

\subsection{Recognition algorithm Overview}
Recognizing simple structures (Diamonds, Jewels, T1, T2, T3).\\

Finding and Using Half-Cleaners.\\

Overview of proof of why algorithm using Half-Cleaners is correct.\\

\subsection{Implementation}

Anything interesting about algo/data structure?\\

Optimizations - Bottlenecks in performance (next path, are vectors distinct etc).\\

Validity tests - unit tests, tests of bigger parts, testing vs known answer and vs naive.

\subsection{Parallelism with CUDA (?)}

TODO

\subsection{Experiments}

Naive algorithm - brief description, bottlenecks optimizations (makes huge difference).\\

Description of tests used.\\

Results and Corollary - almost usable algorithm.



\section{Coloring Berge Graphs}

\subsection{Ellipsoid method}

Desctiption.\\

Implementation.\\

Experiments and results.\\

\subsection{Combinatorial Method}

Cite the paper.\\

On its complexity - point to appendix for pseudo-code.

\appendix
\appendixpage
\addappheadtotoc

\section{Perfect Graph Coloring algorithm}
TODO


% \bibliographystyle{alpha}
\printbibliography

\end{document}