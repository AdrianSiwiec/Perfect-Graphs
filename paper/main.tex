\documentclass{article}
\usepackage{import}

\import{./}{preamble.tex}

% Language and typesetting notes:
%
% - Color, not Colour (American English)

\author{Adrian Siwiec}
\date{\today{}}
\begin{document}

\begin{titlepage}
	\begin{center}
        
		\large
		\textbf{Jagiellonian University}\\
		Department of Theoretical Computer Science\\

		\vspace{1.5cm}

		\Large
		\textbf{Adrian Siwiec}

		\vspace*{2cm}

		\textbf{\LARGE Perfect Graph Recognition and Coloring}
		
		\vspace{0.5cm}
		\large
		
		\vfill
		\Large
		Master Thesis

		\vfill
		\Large
		Supervisor: dr inż. Krzysztof Turowski
		
		\vspace{0.8cm}
		
		September 2020
		
\end{center}
\end{titlepage}
\pagebreak

\begin{abstract}
	TODOa
\end{abstract}

\listoftheorems[ignoreall,show={defn}]
\tableofcontents

\pagebreak

\section*{Definitions}
We use standard definitions, sourced from the book by \citeauthor{BB98} \citetitle*{BB98}, modified and extended as needed.

\begin{defn}[graph]
  A \emph{graph} $G$ is an ordered pair of disjoint sets $(V, E)$ such that $E$ is the subset of the set $V \choose 2$ that is of unordered pairs of $V$.
\end{defn}

\begin{wrapfigure}{r}{0.4\textwidth}
  \centering\begin{tikzpicture}[scale=.7, simplegraph]
    \node(a) at (0, 0) {$v_1$};
    \node(b) at (2, 0) {$v_2$};
    \node(c) at (4, 0) {$v_3$};
    \node(d) at (6, 0) {$v_4$};

    \draw(a) to (b);
    \draw(b) to (c);
    \draw(c) to (d);

    \draw[dashed](a) to[in=90, out=90] (d);
    \draw[dashed](a) to[in=90, out=90] (c);
    \draw[dashed](b) to[in=90, out=90] (d);
  \end{tikzpicture}
  \caption{An example graph $G_0$}
  \label{fig:examplePath}
\end{wrapfigure}

We will only consider finite graphs, that is $V$ and $E$ are always finite. If $G$ is a graph, then $V = V(G)$ is the \emph{vertex set} of $G$, and $E = E(G)$ is the \emph{edge set}. When the context of $G$ is clear we will use $V$ and $E$ to denote its vertex and edge set. The size of the vertex set $V(G)$ is sometimes called a \emph{cardinality} of $G$.

An edge $\{x, y\}$ is said to \emph{join}, or be between vertices $x$ and $y$ and is denoted by $xy$. Thus $xy$ and $yx$ mean the same edge (all our graphs are \emph{undirected}). If $xy \in E(G)$ then $x$ and $y$ are adjacent, connected or neighboring. By $N(x)$ we will denote the \emph{neighborhood} of $x$, that is all vertices $y$ such that $xy$ is an edge. Similarly, for $X \subseteq V(G)$, by $N(X)$ we will denote the neighborhood of $X$, that is all vertices of $v \in V(G) \setminus X$, so that there is a $x \in X$, that $xv$ is an edge in $G$. If $xy \notin E(G)$ then $xy$ is a \emph{nonedge} and $x$ and $y$ are \emph{anticonnected}.

\Cref{fig:examplePath} shows an example of a graph $G_0 = (V, E)$ with $V = \{v_1, v_2, v_3, v_4\}$ and $E = \{v_1v_2, v_2v_3, v_3v_4\}$. We will mark edges as solid lines on figures. Nonedges significant to the ongoing reasoning will be marked as dashed lines.

\begin{defn}[subgraph]
  $G' = (V', E')$ is a \emph{subgraph} of $G = (V, E)$ if $V' \subseteq V$ and $E' \subseteq E$.
\end{defn}

\begin{defn}[induced subgraph]
  If $G' = (V', E')$ is a subgraph of $G$ and it contains \emph{all edges} of $G$ that join two vertices in $V'$, then $G'$ is said to be \emph{induced subgraph} of $G$ and is denoted $G[V']$.
\end{defn}

Given a graph $G = (V, E)$ and a set $X \subseteq V$ by $G\setminus X$ we will denote a induced subgraph $G[V\setminus X]$.

For example $(\{v_1, v_2, v_3\}, \{v_1v_2\})$ is \emph{not} an induced subgraph of the example graph $G_0$, while $(\{v_1, v_2, v_3\}, \{v_1v_2, v_2v_3\}) = G_0[\{v_1, v_2, v_3\}] = G_0 \setminus \{v_0\}$ is.

\begin{defn}[$X$-completeness]
  Given set $X \subseteq V$, vertex $v \notin X$ is \emph{$X$-complete} if it is adjacent to every node $x \in X$. A set $Y \subseteq V$ is $X$-complete if $X \cap Y = \emptyset$ and every node $y \in Y$ is $X$-complete.
\end{defn}

\begin{defn}[path]
  A \emph{path} is a graph $P$ of the form
  \[ V(P) = \{x_1, x_2, \ldots, x_l\},\quad E(P) = \{x_1x_2, x_2x_3, \ldots, x_{l-1}x_l\} \]
\end{defn}
This path $P$ is usually denoted by $x_1x_2\ldots x_l$ or $x_1 - x_2 - \ldots - x_l$. The vertices $x_1$ and $x_l$ are the \emph{endvertices} and ${l-1} = |E(P)|$ is the \emph{length} of the path P. $\{x_2, \ldots x_{l-1}\}$ is the \emph{inside} of the path $P$, denoted as $P^*$. Notice that we don't allow any edges other than the ones between consecutive vertices for a graph to be called a path.

Graph $G_0$ is a path of length 3, with the inside $G_0^* = \{v_2, v_3\}$. If we would add any edge to $G_0$ it would stop being a path.


\begin{defn}[connected graph, subset]
  A graph $G$ is \emph{connected} if for every pair $\{x, y\} \subseteq V(G)$ of distinct vertices, there is a path from $x$ to $y$.
  A subset $X \subseteq V(G)$ is connected if the graph $G[X]$ is connected.
\end{defn}

\begin{defn}[component]
  A \emph{component} of a graph $G$ is its maximal connected induced subgraph.
\end{defn}


\begin{defn}[cycle]
  A \emph{cycle} is a graph $C$ of the form
  \[ V(C) = \{x_1, x_2, \ldots, x_l\},\quad E(C) = \{x_1x_2, x_2x_3, \ldots, x_{l-1}x_l, x_lx_1\} \]
\end{defn}

This cycle $C$ is usually denoted by $x_1x_2\ldots x_lx_1$ or $x_1 - x_2 - \ldots - x_l - x_1$. $l = |E(C)|$ is the \emph{length} of the cycle $C$. Sometimes we will denote the cycle of length $l$ as $C_l$.

Notice, that a cycle is not a path (nor is a path a cycle). If we add an edge $v_1v_4$ to the path $G_0$ it becomes an even cycle $C_4$.

\begin{defn}[hole]
  A \emph{hole} is a cycle of length at least four.
\end{defn}

If a path, a cycle or a hole has an odd length, it will be called \emph{odd}. Otherwise, it will be called \emph{even}.

\begin{defn}[complement]
  A \emph{complement} of a graph $G = (V, E)$ is a graph $\overline{G} = (V, {V \choose 2} \setminus E)$, that is two vertices $x, y$ are adjacent in $\overline{G}$ if and only if they are not adjacent in $G$.
\end{defn}

% Sometimes, we will call a complement of a member of a class $\Gamma$ an \emph{anti-
% $\Gamma$}, e.g. graph $G = (\{v_1, v_2, v_3, v_4\}, \{v_1v_3, v_2v_4\})$ is an anticycle.

\begin{defn}[anticonnected graph, subset]
  A graph $G$ is anticonnected if $\overline{G}$ is connected.
  A subset $X$ is anticonnected if $\overline{G}[X]$ is connected.
\end{defn}

\begin{defn}[anticomponent]
  An \emph{anticomponent} of a graph $G$ is an induced subgraph whose complement is a component in $\overline{G}$.
\end{defn}

\begin{defn}[antihole]
  An \emph{antihole} is a subgraph of $G$ whose complement is a hole in $G$.
\end{defn}

\begin{defn}[clique]
  A \emph{complete} graph or a \emph{clique} is a graph of the form $G = (V, {V \choose 2})$, that is every two vertices are connected. We will denote a clique on $n$ vertices as $K_n$.
\end{defn}

\begin{defn}[clique number]
  A \emph{clique number} of a graph $G$, denoted as $\omega(G)$, is a cardinality of its largest induced clique.
\end{defn}

\begin{defn}[anticlique]
  An \emph{anticlique} is a graph in which there are no edges. We will also call anticliques \emph{independent sets}. 
\end{defn}
In a similar fashion, given a graph $G = (V, E)$, a subset of its vertices $V' \subseteq V$ will be called \emph{independent} (in the context of $G$) if and only if $G[V']$ is an anticlique.

\begin{defn}[stability number]
  A \emph{stability number} of a graph $G$, denoted as $\alpha(G)$, is a cardinality of its largest induced stable set.
\end{defn}

\begin{defn}[coloring]
  Given a graph $G$, its \emph{coloring} is a function $c: V(G) \rightarrow \mathbb{N}^+$, such that $c(x) \neq c(y)$ for every edge $xy \in E(G)$. A $k$-coloring of $G$ (if exists) is a coloring, such that $c(x) \leq k$ for all vertices $x \in V(G)$.
\end{defn}

\begin{defn}[chromatic number]
  A \emph{chromatic number} of a graph $G$, denoted as $\chi(G)$, is a smallest natural number $k$, for which there exists a $k$-coloring of $G$.
\end{defn}

\begin{defn}[line graph]
  The \emph{line graph} of a graph $G = (V, E)$ is the graph $L(G)$ with vertex set equal $E$, where $e, h \in E$ are adjacent in $L(G)$ if and only if they share an end in $G$.
\end{defn}




\section{Perfect Graphs}

Given a graph $G$, let us consider a problem of coloring it using as few colors as possible. If $G$ contains a clique $K$ as a subgraph, we must use at least $|V(K)|$ colors to color it. This gives us a lower bound for a chromatic number $\chi(G)$ -- it is always greater or equal to the cardinality of the largest clique $\omega(G)$. The reverse is not always true, in fact we can construct a graph with no triangle and requiring arbitrarily large numbers of colors (e.g. construction by Mycielski \cite{Mycielski1955}).

Do graphs that admit coloring using only $\omega(G)$ color are "simpler" to further analyze? Not necessarily so. Given a graph $G = (V, E)$, $|V| = n$, let us construct a graph $G'$ equal to the union of $G$ and a clique $K_n$. We can see that indeed $\chi(G') = n = \omega(G')$, but it gives us no indication of the structure of $G$ or $G'$.

A definition of perfect graphs (\cref{def:perfectGraph}) states that given a graph $G$, the chromatic number and cardinality of the largest clique of \emph{its every induced subgraph} should be equal. The notion of perfect graphs was first introduced by Berge in 1961 \cite{CB61} and it indeed captures the idea of graph being ''simple'' -- in all perfect graphs the coloring problem, maximum clique problem, and maximum independent set problem can be solved in polynomial time \cite{grotschel1993}.

One of properties of perfect graphs is the \emph{perfect graph theorem} first conjured by Berge in 1961~\cite{CB61} and then proven by \Lovasz in 1972~\cite{LL72}.

\begin{theorem}[Perfect graph theorem]
	A graph is perfect if and only if its complement graph is also perfect. \todo{Should we give some proof of that here? Maybe based on proof in \cite{GC03}}
\end{theorem}

\subsection{Strong Perfect Graph Theorem}

Odd holes are not perfect, since their chromatic number is 3 and their largest cliques are of size 2. It is also easy to see, that an odd antihole of size $n$ has a chromatic number of $\frac{n+1}{2}$ and largest cliques of size $\frac{n-1}{2}$. A graph with no odd hole and no odd antihole is called \emph{Berge} (\cref{def:bergeGraph}) after Claude Berge who studied perfect graphs.

In 1961 Berge conjured that a graph is perfect iff it contains no odd hole and no odd antihole in what has become known as a strong perfect graph conjecture. In 2001 Chudnovsky et al. have proven it and published the proof in an over 150 pages long paper \citetitle{MC06} \cite{MC06}. The following overview of the proof will be based on this paper and on an article with the same name by Cornuéjols \cite{GC03}.

\begin{theorem}[Strong perfect graph theorem]
	A graph is perfect if and only if it is Berge.
\end{theorem}


\TODO{How long and detailed overview of the proof should we provide?}
\TODO{Znakomite pytanie. Generalnie ponieważ dowód jest drobiazgowy i trikowy, to wystarczy "z lotu ptaka" tj. to, co Chudnovsky i Seymour piszą w pracach popularnych i referują w wystąpieniach gościnnych typu te nagrania na YT. Głębiej nie ma sensu. Warto podkreślić, że technika dowodu i sam algorytm są zależne na dużo głębszym poziomie niż widać tj. ktoś mając dowód Stron Perfect Graph Theorem pewnie nie rozgryzłby algorytmu i vice versa.}

\section{Recognizing Berge Graphs}

The following is based on the paper by Maria Chudnovsky et al. \citetitle*[]{MC05}. We will not provide full proof of its correctness, but will aim to show the intuition behind the algorithm.

\subsection{Recognition algorithm Overview}

Berge graph recognition algorithm could be divided into two parts: first we check if either $G$ or $\overline{G}$ contain any of a number of simple structures as a induced subgraph (\ref{SimpleStructures}). If they do, we output that graph is not Berge and stop. Else, we check if there is a near-cleaner for a shortest odd hole.(\ref{AmenableHoles}).

\subsubsection{Simple structures}
\label{SimpleStructures}

\paragraph{Pyramids}

% \begin{floatingfigure}{0.35\textwidth}

\begin{wrapfigure}{r}{0.35\textwidth}
	\centering\begin{tikzpicture}[scale=0.7,simplegraph]
  \node(a) at (0,0) {$a$};
  \node(b1) at (-2,-6) {$b_1$};
  \node(b2) at (0,-4.268) {$b_2$};
  \node(b3) at (2,-6) {$b_3$};

  \node(P12) at (-2/4, -6/4) {\small$P_{12}$};
  \node(P13) at (-4/4, -12/4) {\small$P_{13}$};
  \node(P14) at (-6/4, -18/4) {\small$P_{14}$};

  \node(P32) at (2/3, -6/3) {\small$P_{32}$};
  \node(P33) at (4/3, -12/3) {\small$P_{33}$};

  \draw (b1) to (b3);
  \draw (b1) to (b2);
  \draw (b2) to (b3);

  \draw (a) to (P12);
  \draw (P12) to (P13);
  \draw (P13) to (P14);
  \draw (P14) to (b1);

  \draw (a) to (P32);
  \draw (P32) to (P33);
  \draw (P33) to (b3);

  \draw (a) to (b2);
\end{tikzpicture}%
	\caption{An example  of a pyramid.}%
	\vspace{-1.2cm}
	% \end{floatingfigure}
\end{wrapfigure}

A \emph{pyramid} in G is an induced subgraph formed by the union of a triangle \footnote{A triangle is a clique $K_3$.} $\{b_1,b_2,b_3\}$, three paths $\{P_1, P_2, P_3\}$ and another vertex $a$, so that:
\begin{itemize}
	\item $\forall_{1\leq i \leq 3}$ $P_i$ is a path between $a$ and $b_i$
	\item $\forall_{1\leq i < j \leq 3}$ $a$ is the only vertex in both $P_i$ and $P_j$ and $b_ib_j$ is the only edge between $V(P_i)\setminus\{a\}$ and $V(P_j)\setminus\{a\}$.
	\item $a$ is adjacent to at most one of $\{b_1, b_2, b_3\}$.
\end{itemize}

We will say that $a$ can be \emph{linked onto} the triangle $\{b_1, b_2, b_3\}$ \emph{via} the paths $P_1$, $P_2$, $P_3$. Let us notice, that a pyramid is determined by its paths $P_1$, $P_2$, $P_3$.

It is easy to see that every graph containing a pyramid contains an odd hole -- at least two of the paths $P_1$, $P_2$, $P_3$ will have the same parity.

\subparagraph{Finding Pyramids}

If $K$ is a pyramid $(a, b_1, b_2, b_3, P_1, P_2, P_3)$ we say its \emph{frame} is the 10-tuple iff
\TODO{finish it up}


First, let us enumerate all 6-tuples $b_1, b_2, b_3, s_1, s_2, s_3$ such that:
\begin{itemize}
	\item $\{b_1, b_2, b_3\}$ is a triangle
	\item for $1 \leq i < j \leq 3$, ${b_i, s_i}$ is disjoint from ${b_j, s_j}$ and $b_ib_j$ is the only edge between them
	\item there is a vertex $a$ adjacent to all of $s_1, s_2, s_3$ and to at most one of $b_1, b_2, b_3$, such that if $a$ is adjacent to $b_i$, then $s_i \ b_i$.
\end{itemize}

There are $O(|V(G)|^6)$ 6-tuples, and it takes $O(|V(G)|)$ time to check each one. For each such 6-tuple we follow with the rest of the algorithm.

We define $M = V(G) \setminus \{b_1, b_2, b_3, s_1, s_2, s_3\}$. Now, for each $m \in M$, we set $S_1(m)$ equal to the shortest path between $s_1$ and $m$ such that $s_2, s_3, b_2, b_3$ have no neighbors in its interior, if such a path exists. We set $S_2$ and $S_3$ similarly. Then similarly we set $T_1(m)$ to be the shortest path between $m$ and $b_1$, such that $s_2, s_3, b_2, b_3$ have no neighbors in its interior, if such a path exists. We do similar for $T_2$ and $T_3$. It takes $O(|V(G)|^2)$ time to calculate paths $T_i(m)$ for all $i$ and $m$.

Now, we will calculate all possible paths $P_i$. For each $m \in M \cup \{b_1\}$ we will define a path $P_1(m)$ and paths $P_2(m)$, $P_3(m)$ will be defined in a similar manner.

If $s_1 = b_1$ let $P_1(b_1)$ be the one-vertex path with vertex $b_1$, and let $P_1(m)$ be undefined for each $m \in M$.

If $s_1 \neq b_1$, then $P_1(b_1)$ is undefined and for all $m \in M$ we will check if all the following are true:
\begin{itemize}
	\item $m$ is nonadjacent to all of $b_2, b_3, s_2, s_3$
	\item $S_1(m)$ and $T_1(m)$ both exist
	\item $V(S_1(m) \cap T_1(m)) = \{m\}$
	\item there are no edges between $V(S_1(m) \setminus m)$ and $V(T_1(m) \setminus m)$
\end{itemize}
If so, then we assign a path $s_1-S_1(m)-m-T_1(m)-b_1$ to $P_1(m)$, otherwise we let $P_1(m)$ be undefined. It takes $O(|V(G)|^2)$ to check this, given $m$. We assign $P_2$ and $P_3$ in a similar manner. Total time of finding all $P_i(m)$ paths for a given 6-tuple is $O(|V(G)|^3)$.

Now we want to check if there is a triple $m_1, m_2, m_3$, so that $P_1(m_1)$, $P_2(m_2)$, $P_3(m_3)$ form a pyramid. A most obvious approach of enumerating them all would be too slow, so we do it carefully.

For $1 \leq i < j \leq 3$, we say that $(m_i, m_j)$ is a \emph{good $(i, j)$-pair}, iff $m_i \in M \cup \{b_i\}$, $m_j \in M \cup \{b_j\}$, $P_i(m_i)$, $P_j(m_j)$ both exist and the sets $V(P_i(m_i))$,$V(P_j(m_j))$ are both disjoint and $b_ib_j$ is the only edge between them.

We show how to find the list of all good $(1, 2)$-pairs, with similar algorithm for all other good $(i, j)$-pairs. For each $m_1 \in M \cup \{b_1\}$, we find the set of all $m_2$ such that $(m_1, m_2)$ is a good $(1,2)$-pair as follows.

If $P_1(m_1)$ does not exist, there are no such good pairs. If it exists, color black the vertices of $M$ that either belong to $P_1(m_1)$ or have a neighbor in $P_1(m_1)$. Color all other vertices white. (We can do this in $O(|V(G)|^2)$) Then for each $m_2 \in M \cup \{b_2\}$, test whether $P_2(m_2)$ exists and contains no black vertices. We do this for all $m_1$ and get a set of all $(1,2)$-good pairs. In similar way we calculate all good $(1,3)$-pair and $(2,3)$-pairs (in $O(|V(G)|^3)$ time).

Now, we examine all triples $m_1, m_2, m_3$ such that $m_i \in M \cup \{b_i\}$ and test whether $(m_i, m_j)$ is a good $(i, j)$-pair. If we find a triple such that all three pairs are good, we output that G contains a pyramid and stop.

If after examining all choices of $b_1, b_2, b_3, s_1, s_2, s_3$ we find no pyramid, output that $G$ contains no pyramid. Since there are $O(|V(G)|^6)$ such choices and it takes a time of $O(|V(G)|^3)$ to analyze each one, the total time is $O(|V(G)|^9)$.

\TODO{some proofs}


\paragraph{Jewels}

% \begin{wrapfigure}[7]{r}{0.35\textwidth}
\begin{wrapfigure}{r}{0.35\textwidth}
	\centering\begin{tikzpicture}[scale=0.7,simplegraph]
  \def\ngon{5}
  \node[regular polygon,regular polygon sides=\ngon,minimum size=3cm, draw=none] (p) {};
  \foreach\x in {1,...,\ngon}{\node[] (p\x) at (p.corner \x){$v_\the\numexpr\intcalcMod{\x+3}{5}+1$};}
  %p1 - v5, p2 - v1 ...
  \draw (p1) to (p2);
  \draw (p2) to (p3);
  \draw (p3) to (p4);
  \draw (p4) to (p5);
  \draw (p1) to (p5);

  \draw[dashed] (p2) to (p5);
  \draw[dashed] (p2) to (p4);
  \draw[dashed] (p3) to (p5);

  \tikzset{decoration={snake,amplitude=.6mm,segment length=6mm,
        post length=0mm,pre length=0mm}}
  \draw[decorate] (p2) to [out=-100, in=180] (0, -5) to [out=0, in=-80] (p5);
  \node[draw=none] at (0, -4.7) {$P$};
\end{tikzpicture}
	\caption{An example of a jewel.}
	\vspace{-1.5cm}
\end{wrapfigure}


Five vertices $v_1, \ldots, v_5$ and a path $P$ form a \emph{jewel} iff:

\begin{itemize}
	\item $v_1, \ldots, v_5$ are distinct vertices.
	\item $v_1v_2, v_2v_3, v_3v_4, v_4v_5, v_5v_1$ are edges.
	\item $v_1v_3, v_2v_4, v_1,v_4$ are nonedges.
	\item $P$ is a path between $v_1$ and $v_4$, such that $v_2, v_3, v_5$ have no neighbors in its inside.
\end{itemize}

Most obvious way to find a jewel would be to enumerate all choices of $v_1, \ldots v_5$, check if a choice is correct and if it is try to find a path $P$ as required. This gives us a time of $O(|V|^7)$. We could speed it up to $O(|V|^6)$ with more careful algorithm, but since whole algorithms takes time $O(|V|^9)$ and our testing showed that time it takes to test for jewels is negligible we decided against it.

\paragraph{Configurations of type $\T_1$}

A configuration of type $\T_1$ is a hole of length 5. To find it we simply iterate all choices of paths of length of 4, and check if there exists a fifth vertex to complete the hole. See paragraph \ref{Optimizations} for more implementation details.

\paragraph{Configurations of type $\T_2$}

% \begin{wrapfigure}[15]{r}{0.35\textwidth}
\begin{wrapfigure}{r}{0.35\textwidth}
	\centering\begin{tikzpicture}[scale=0.7,simplegraph]
  \node(v1) at (0,0) {$v_1$};
  \node(v2) at (1.5,0) {$v_2$};
  \node(v3) at (3,0) {$v_3$};
  \node(v4) at (4.5,0) {$v_4$};

  \draw (v1) to (v2);
  \draw (v2) to (v3);
  \draw (v3) to (v4);

  \tikzset{decoration={snake,amplitude=.6mm,segment length=6mm,
        post length=0mm,pre length=0mm}}
  \draw[decorate] (v1) to [out=90, in=180] (4.5/2, 3) to [out=0, in=90] (v4);
  \node[draw=none] at (4.5/2, 3.5) {$P$};

  \draw[dashed] (v2) to (.7,2.5);
  \draw[dashed] (v3) to (.7,2.5);
  \draw[dashed] (v2) to (3.8,2.5);
  \draw[dashed] (v3) to (3.8,2.5);

  \node[minimum size=4mm](x1) at (1.2, -2){};
  \node[minimum size=4mm](x2) at (3.3, -2){};
  \node[draw,dotted,inner sep=3pt, circle,yscale=.5, fit={(x1) (x2)},label=below:{$X$}] {};

  \draw[dashed] (x1) to (x2);
  \draw(v1) to (x1);
  \draw(v2) to (x1);
  \draw(v4) to (x1);
  \draw(v1) to (x2);
  \draw(v2) to (x2);
  \draw(v4) to (x2);

  % \draw (a) to (b2);
\end{tikzpicture}%
	\caption{An example of a $\T_2$.}%
	\vspace{-1.5cm}
\end{wrapfigure}

A configuration of type $\T_2$ is a tuple $(v_1, v_2, v_3, v_4, P, X)$, such that:
\begin{itemize}
	\item $v_1v_2v_3v_4$ is a path in $G$.
	\item $X$ is an anticomponent of the set of all $\{v_1, v_2, v_4\}$-complete vertices.
	\item $P$ is a path in $G\setminus(X \cup \{v_2, v_3\})$ between $v_1$ and $v_4$ and no vertex in $P^*$ is $X$-complete or adjacent to $v_2$ or adjacent to $v_3$.
\end{itemize}

Checking if configuration of type $\T_2$ exists in our graph is straightforward: we enumerate all paths $v_1\ldots v_4$, calculate set of all $\{v_1, v_2, v_4\}$-complete vertices and its anticomponents. Then, for each anticomponent $X$ we check if required path $P$ exists.

To prove that existence of $\T_2$ configuration implies that the graph is not berge, we will need the following Roussel-Rubio lemma:

\begin{lemma}[Roussel-Rubio Lemma \cite{RR01,MC05}]\label{lem:Roussel-Rubio}
	Let $G$ be Berge, $X$ be an anticonnected subset of $V(G)$, $P$ be an odd path $p_1\ldots p_n$ in $G\setminus X$ with length at least 3, such that $p_1$ and $p_n$ are $X$-complete and $p_2, \ldots, p_{n-1}$ are not. Then:
	\begin{itemize}
		\item $P$ is of length at least 5 and there exist nonadjacent $x, y \in X$, such that there are exactly two edges between $x, y$ and $P^*$, namely $xp_2$ and $yp_{n-1}$,
		\item or $P$ is of length 3 and there is an odd antipath joining internal vertices of $P$ with interior in $X$.
	\end{itemize}
	\TODO{We may use this lemma quite often, might want to provide proof if so.}
\end{lemma}

Now, we shall prove the following:

\begin{lemma}
	If $G$ contains configuration of type $\T_2$ then $G$ is not Berge.
\end{lemma}
\begin{proof}
	Let $(v_1, v_2, v_3, v_4, P, X)$ be a configuration of type $\T_2$. Let us assume that $G$ is not Berge and consider the following:
	\begin{itemize}
		\item If $P$ is even, then $v_1, v_2, v_3, v_4, P, v_1$ is an odd hole,
		\item If $P$ is of length 3. \todo{I merged a couple of proofs from \cite{MC06}, check in the morning if this is correct.} Let us name its vertices $v_1, p_2, p_3, v_4$. It follows from \cref{lem:Roussel-Rubio}, that there exists an odd antipath between $p_2$ and $p_3$ with interior in $X$. We can complete it with $v_2p_2$ and $v_2p_3$ into an odd antihole.
		\item If $P$ is odd with the length of at least 5 \todo{check in the morning}, it follows from \cref{lem:Roussel-Rubio} that we have $x, y \in X$ with only two edges to $P$ being $xp_2$ and $yp_{n-1}$. This gives us an odd hole: $v_2, x, p_2, \ldots, p_{n-1}, y, v_2$.
	\end{itemize}
\end{proof}

\paragraph{Configurations of type $\T_3$}

A configuration of type $\T_3$ is a sequence $v_1, \ldots, v_6$, $P$, $X$, such that:
\begin{itemize}
	\item $v_1, \ldots v_6$ are distinct vertices.
	\item $v_1v_2$, $v_3v_4$, $v_1v_4$, $v_2v_3$, $v_3v_5$, $v_4v_6$ are edges, and $v_1v_3$, $v_2v_4$, $v_1v_5$, $v_2v_5$, $v_1v_6$, $v_2v_6$, $v_4v_5$ are nonedges.
	\item $X$ is an anticomponent of the set of all $\{v_1, v_2, v_5\}$-complete vertices, and $v_3$, $v_4$ are not $X$-complete.
	\item $P$ is a path of $G \setminus ( X \cup \{v_1, v_2, v_3, v_4\} )$ between $v_5$ and $v_6$ and no vertex in $P*$ is $X$-complete or adjacent to $v_1$ or adjacent to $v_2$.
	\item If $v_5v_6$ is an edge, then $v_6$ is not $X$-complete.
\end{itemize}

\TODO{picture!}

The following algorithm with running time of $O(|V(G)|^6)$ checks whether $G$ contains a configuration of type $T_3$:

For each triple $v_1, v_2, v_5$ of vertices such that $v_1v_2$ is an edge and $v_1v_5, v_2v_5$ are nonedges find the set $Y$ of all $\{v_1, v_2, v_5\}$-complete vertices. For each anticomponent $X$ of $Y$ find the maximal connected subset $F'$ containing $v_5$ such that $v_1, v_2$ have no neighbors in $F'$ and no vertex of $F'\setminus\{v_5\}$ is $X$-complete. Let $F$ be the union of $F'$ and the set of all $X$-complete vertices that have a neighbor in $F'$ and are nonadjacent to all of $v_1, v_2$ and $v_5$.

Then, for each choice of $v_4$ that is adjacent to $v_1$ and not to $v_2$ and $v_5$ and has a neighbor in $F$ (call it $v_6$) and a nonneibhbor in $X$, we test whether there is a vertex $v_3$, adjacent to $v_2, v_4, v_5$ and not to $v_1$, with a nonneibhbor in $X$. If there is such a vertex $v_3$, find $P$ -- a path from $v_6$ to $v_5$ with interior in $F'$ and return that $v_1, \ldots v_6, P, X$ is a configuration of type $\T_3$. If we exhaust our search and find none, report that graph does not contain it.

To see that the algorithm below has a running time of $O(|V(G)|^6)$, let us note that for each triple $v_1, v_2, v_5$ we examine, of which there are $O(|V(G)|^3)$, there are linear many choices of $X$, each taking $O(|V(G)|^2)$ time to process and generating a linear many choices of $v_4$ which take a linear time to process in turn. This gives us the total running time of $O(|V(G)|^6)$.

We will skip the proof that each graph containing $\T_3$ is not Berge. See paragraph 6.7 of \cite{MC05} for the proof.

\subsubsection{Amenable holes.}
\label{AmenableHoles}

\begin{theorem}
	Let $G$ be a graph, such that $G$ and $\overline{G}$ contain no Pyramid, no Jewel and no configuration of types $\T_1, \T_2$ or $\T_3$. Then every shortest hole in $G$ is amenable.
	\TODO{Any ideas on what else to say here? List all 9 steps?}
	\TODO{Dobre pytanie. Te 9 kroków to da się zawrzeć w 1-2 zdaniach każde? Jeśli nie, to tylko napisałbym ogólnikowo, że "jest sekwencja kroków, która to robi, tu jest odnośnik"}
\end{theorem}


\TODO{Finding and Using near-cleaners.}

\TODO{Overview of proof of why algorithm using Half-Cleaners is correct.}

\subsection{Implementation}

Anything interesting about algo/data structure?\\

\subsubsection{Optimizations}\label{Optimizations}
Bottlenecks in performance (next path, are vectors distinct etc).\\
\TODO{In our graph preprocessing we have a pointer to next edge in order to speed up generating next path.}
\TODO{We used callgrind to get idea of methods crucial for time.}
\TODO{In general enumerating all paths is crucial. As is checking if vector has distinct values.}
\TODO{Jewels -- we iterate all possibly chordal paths and check if they are ok - much faster}
\TODO{$\T_1$ -- we iterate all paths of length 4 and check if there exists a fifth vertex to complete the hole - much faster than iterating vertices.}

Validity tests - unit tests, tests of bigger parts, testing vs known answer and vs naive.

\subsection{Parallelism with CUDA (?)}

TODO

\subsection{Experiments}

Naive algorithm - brief description, bottlenecks optimizations (makes huge difference).\\

Description of tests used.\\

Results and Corollary - almost usable algorithm.



\section{Coloring Berge Graphs}

\subsection{Ellipsoid method}

Description.\\

Implementation.\\

Experiments and results.\\

\subsection{Combinatorial Method}

Cite the paper.\\

On its complexity - point to appendix for pseudo-code.

\appendix
\appendixpage
\addappheadtotoc

\section{Perfect Graph Coloring algorithm}
TODO


% \bibliographystyle{alpha}
\printbibliography

\end{document}