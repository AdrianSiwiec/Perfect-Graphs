\documentclass{article}

\usepackage{appendix}

\author{Adrian Siwiec}
\date{\today{}}
\begin{document}
\begin{titlepage}
	\begin{center}
        
		\large
		\textbf{Jagiellonian University}\\
		Department of Theoretical Computer Science\\

		\vspace{1.5cm}

		\Large
		\textbf{Adrian Siwiec}

		\vspace*{2cm}

		\textbf{\LARGE Perfect Graph Recognition and Coloring}
		
		\vspace{0.5cm}
		\large
		
		\vfill
		\Large
		Master Thesis

		\vfill
		\Large
		Supervisor: dr inż. Krzysztof Turowski
		
		\vspace{0.8cm}
		
		September 2020
		
\end{center}
\end{titlepage}

\pagebreak

\begin{abstract}
TODO
\end{abstract}

\tableofcontents

\pagebreak

\section{Perfect Graphs}
Definition of perfect graphs.\\

Why are they interesting (Some examples of subclasses, and problems that are solvable for perfect graphs, including recognition and coloring).\\

Weak perfect graph theorem.\\

Berge graphs.

\subsection{Strong Perfect Graph Theorem}
Cite the paper, brief description of the theorem.


\section{Recognizing Berge Graphs}
Cite the paper.

\subsection{Recognition algorithm Overview}
Recognizing simple structures (Diamonds, Jewels, T1, T2, T3).\\

Finding and Using Half-Cleaners.\\

Overview of proof of why algorithm using Half-Cleaners is correct.\\

\subsection{Implementation}

Anything interesting about algo/data structure?\\

Optimizations - Bottlenecks in performance (next path, are vectors distinct etc).\\

Validity tests - unit tests, tests of bigger parts, testing vs known answer and vs naive.

\subsection{Parallelism with CUDA (?)}

TODO

\subsection{Experiments}

Naive algorithm - brief description, bottlenecks optimizations (makes huge difference).\\

Description of tests used.\\

Results and Corollary - almost usable algorithm.



\section{Coloring Berge Graphs}

\subsection{Ellipsoid method}

Desctiption.\\

Implementation.\\

Experiments and results.\\

\subsection{Combinatorial Method}

Cite the paper.\\

On its complexity - point to appendix for pseudo-code.

\appendix
\appendixpage
\addappheadtotoc

\section{Perfect Graph Coloring algorithm}
TODO

\end{document}