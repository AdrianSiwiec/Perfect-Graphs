Given a graph $G$, let us consider a problem of coloring it using as few colors as possible. If $G$ contains a clique $K$ as a subgraph, we must use at least $|V(K)|$ colors to color it. This gives us a lower bound for a chromatic number $\chi(G)$ -- it is always greater or equal to the cardinality of the largest clique $\omega(G)$. The reverse is not always true, in fact we can construct a graph with no triangle and requiring arbitrarily large numbers of colors (e.g. construction by Mycielski \cite{Mycielski1955}).
 
Do graphs that admit coloring using only $\omega(G)$ color are "simpler" to further analyze? Not necessarily so. Given a graph $G = (V, E)$, $|V| = n$, let us construct a graph $G'$ equal to the union of $G$ and a clique $K_n$. We can see that indeed $\chi(G') = n = \omega(G')$, but it gives us no indication of the structure of $G$ or $G'$.

A definition of perfect graphs (\cref{def:perfectGraph}) states that given a graph $G$, the chromatic number and cardinality of the largest clique of \emph{its every induced subgraph} should be equal. The notion of perfect graphs was first introduced by Berge in 1961 \cite{CB61} and it indeed captures some of the idea of graph being ''simple'' -- in all perfect graphs the coloring problem, maximum (weighted) clique problem, and maximum (weighted) independent set problem can be solved in polynomial time \cite{grotschel1993}. Other classical NP-complete problems are still NP-complete in perfect graphs e.g. Hamiltonian path \cite{Mller1996}, maximum cut problem \cite{Bodlaender1994} or dominating set problem \cite{Dewdney81}. \todo{all theese citations are about subclasses e.g. biparite graphs. Should we mention some subclasses?}

The most fundamental problem -- the problem of recognizing perfect graphs -- was open since its posing in 1961 until recently. Its solution, a polynomial algorithm recognizing perfect graphs is a union of the strong perfect graph theorem (\cref{sec:SPGT}) stating that a graph is perfect if and only if it is Berge (\cref{def:bergeGraph}) and an algorithm for recognizing Berge graphs in polynomial time (\cref{sec:recognizingBerge}).

Perfect graphs are interesting not only because of their theoretical properties, but they are also used in other areas of study e.g. integrality of polyhedra \cite{Chvtal1975, Chudnovsky2003}, radio channel assignment problem \cite{McDiarmid99, McDiarmid2000} and appear in the theory of Shannon capacity of a graph \cite{Lovasz1979}. Also, as pointed out in \cite{alfonsinPerfect2001, Chudnovsky2003} algorithms to solve semi-definite programs grew out of the theory of perfect graphs. We will take a look at semi-definite programs and at perfect graph's relation to Shannon capacity in \cref{sec:ellipsoidMethod}.

\section{Strong Perfect Graph Theorem}
\label{sec:SPGT}

The first step to solve the problem of recognizing perfect graphs was the \emph{(weak) perfect graph theorem} first conjured by Berge in 1961~\cite{CB61} and then proven by \Lovasz in 1972~\cite{LL72}.

\begin{theorem}[Perfect graph theorem]
	A graph is perfect if and only if its complement graph is also perfect. \todo{Should we give some proof of that here? Maybe based on proof in \cite{GC03}}
\end{theorem}

Odd holes are not perfect, since their chromatic number is 3 and their largest cliques are of size 2. It is also easy to see, that an odd antihole of size $n$ has a chromatic number of $\frac{n+1}{2}$ and largest cliques of size $\frac{n-1}{2}$. A graph with no odd hole and no odd antihole is called \emph{Berge} (\cref{def:bergeGraph}) after Claude Berge who studied perfect graphs.

In 1961 Berge conjured that a graph is perfect iff it contains no odd hole and no odd antihole in what has become known as a strong perfect graph conjecture. In 2001 Chudnovsky et al. have proven it and published the proof in an over 150 pages long paper \citetitle{MC06} \cite{MC06}. The following overview of the proof will be based on this paper and on an article with the same name by Cornuéjols \cite{GC03}.

\begin{theorem}[Strong perfect graph theorem]
	A graph is perfect if and only if it is Berge.
\end{theorem}


\TODO{How long and detailed overview of the proof should we provide?}
\TODO{Znakomite pytanie. Generalnie ponieważ dowód jest drobiazgowy i trikowy, to wystarczy "z lotu ptaka" tj. to, co Chudnovsky i Seymour piszą w pracach popularnych i referują w wystąpieniach gościnnych typu te nagrania na YT. Głębiej nie ma sensu. Warto podkreślić, że technika dowodu i sam algorytm są zależne na dużo głębszym poziomie niż widać tj. ktoś mając dowód Stron Perfect Graph Theorem pewnie nie rozgryzłby algorytmu i vice versa.}

\section{Recognizing Berge Graphs}
\label{sec:recognizingBerge}

The following is based on the paper by Maria Chudnovsky et al. \citetitle*[]{MC05}. We will not provide full proof of its correctness, but will aim to show the intuition behind the algorithm.

Berge graph recognition algorithm could be divided into two parts: first we check if either $G$ or $\overline{G}$ contain any of a number of simple structures as a induced subgraph (\ref{SimpleStructures}). If they do, we output that graph is not Berge and stop. Else, we check if there is a near-cleaner for a shortest odd hole (\ref{AmenableHoles}).

\subsection{Simple structures}
\label{SimpleStructures}

\subsubsection{Pyramids}

% \begin{floatingfigure}{0.35\textwidth}

\begin{wrapfigure}{r}{0.35\textwidth}
	\centering\begin{tikzpicture}[scale=0.7,simplegraph]
  \node(a) at (0,0) {$a$};
  \node(b1) at (-2,-6) {$b_1$};
  \node(b2) at (0,-4.268) {$b_2$};
  \node(b3) at (2,-6) {$b_3$};

  \node(P12) at (-2/4, -6/4) {\small$P_{12}$};
  \node(P13) at (-4/4, -12/4) {\small$P_{13}$};
  \node(P14) at (-6/4, -18/4) {\small$P_{14}$};

  \node(P32) at (2/3, -6/3) {\small$P_{32}$};
  \node(P33) at (4/3, -12/3) {\small$P_{33}$};

  \draw (b1) to (b3);
  \draw (b1) to (b2);
  \draw (b2) to (b3);

  \draw (a) to (P12);
  \draw (P12) to (P13);
  \draw (P13) to (P14);
  \draw (P14) to (b1);

  \draw (a) to (P32);
  \draw (P32) to (P33);
  \draw (P33) to (b3);

  \draw (a) to (b2);
\end{tikzpicture}%
	\caption{An example  of a pyramid.}%
	\vspace{-1.2cm}
	% \end{floatingfigure}
\end{wrapfigure}

A \emph{pyramid} in G is an induced subgraph formed by the union of a triangle \footnote{A triangle is a clique $K_3$.} $\{b_1,b_2,b_3\}$, three paths $\{P_1, P_2, P_3\}$ and another vertex $a$, so that:
\begin{itemize}
	\item $\forall_{1\leq i \leq 3}$ $P_i$ is a path between $a$ and $b_i$
	\item $\forall_{1\leq i < j \leq 3}$ $a$ is the only vertex in both $P_i$ and $P_j$ and $b_ib_j$ is the only edge between $V(P_i)\setminus\{a\}$ and $V(P_j)\setminus\{a\}$.
	\item $a$ is adjacent to at most one of $\{b_1, b_2, b_3\}$.
\end{itemize}

We will say that $a$ can be \emph{linked onto} the triangle $\{b_1, b_2, b_3\}$ \emph{via} the paths $P_1$, $P_2$, $P_3$. Let us notice, that a pyramid is determined by its paths $P_1$, $P_2$, $P_3$.

It is easy to see that every graph containing a pyramid contains an odd hole -- at least two of the paths $P_1$, $P_2$, $P_3$ will have the same parity.

\paragraph{Finding Pyramids}

First, let us enumerate all 6-tuples $b_1, b_2, b_3, s_1, s_2, s_3$ such that:
\begin{itemize}
	\item $\{b_1, b_2, b_3\}$ is a triangle
	\item for $1 \leq i < j \leq 3$, ${b_i, s_i}$ is disjoint from ${b_j, s_j}$ and $b_ib_j$ is the only edge between them
	\item there is a vertex $a$ adjacent to all of $s_1, s_2, s_3$ and to at most one of $b_1, b_2, b_3$, such that if $a$ is adjacent to $b_i$, then $s_i \ b_i$.
\end{itemize}

There are $O(|V(G)|^6)$ 6-tuples, and it takes $O(|V(G)|)$ time to check each one. For each such 6-tuple we follow with the rest of the algorithm.

We define $M = V(G) \setminus \{b_1, b_2, b_3, s_1, s_2, s_3\}$. Now, for each $m \in M$, we set $S_1(m)$ equal to the shortest path between $s_1$ and $m$ such that $s_2, s_3, b_2, b_3$ have no neighbors in its interior, if such a path exists. We set $S_2$ and $S_3$ similarly. Then similarly we set $T_1(m)$ to be the shortest path between $m$ and $b_1$, such that $s_2, s_3, b_2, b_3$ have no neighbors in its interior, if such a path exists. We do similar for $T_2$ and $T_3$. It takes $O(|V(G)|^2)$ time to calculate paths $T_i(m)$ for all $i$ and $m$.

Now, we will calculate all possible paths $P_i$. For each $m \in M \cup \{b_1\}$ we will define a path $P_1(m)$ and paths $P_2(m)$, $P_3(m)$ will be defined in a similar manner.

If $s_1 = b_1$ let $P_1(b_1)$ be the one-vertex path with vertex $b_1$, and let $P_1(m)$ be undefined for each $m \in M$.

If $s_1 \neq b_1$, then $P_1(b_1)$ is undefined and for all $m \in M$ we will check if all the following are true:
\begin{itemize}
	\item $m$ is nonadjacent to all of $b_2, b_3, s_2, s_3$
	\item $S_1(m)$ and $T_1(m)$ both exist
	\item $V(S_1(m) \cap T_1(m)) = \{m\}$
	\item there are no edges between $V(S_1(m) \setminus m)$ and $V(T_1(m) \setminus m)$
\end{itemize}
If so, then we assign a path $s_1-S_1(m)-m-T_1(m)-b_1$ to $P_1(m)$, otherwise we let $P_1(m)$ be undefined. It takes $O(|V(G)|^2)$ to check this, given $m$. We assign $P_2$ and $P_3$ in a similar manner. Total time of finding all $P_i(m)$ paths for a given 6-tuple is $O(|V(G)|^3)$.

Now we want to check if there is a triple $m_1, m_2, m_3$, so that $P_1(m_1)$, $P_2(m_2)$, $P_3(m_3)$ form a pyramid. A most obvious approach of enumerating them all would be too slow, so we do it carefully.

For $1 \leq i < j \leq 3$, we say that $(m_i, m_j)$ is a \emph{good $(i, j)$-pair}, iff $m_i \in M \cup \{b_i\}$, $m_j \in M \cup \{b_j\}$, $P_i(m_i)$, $P_j(m_j)$ both exist and the sets $V(P_i(m_i))$,$V(P_j(m_j))$ are both disjoint and $b_ib_j$ is the only edge between them.

We show how to find the list of all good $(1, 2)$-pairs, with similar algorithm for all other good $(i, j)$-pairs. For each $m_1 \in M \cup \{b_1\}$, we find the set of all $m_2$ such that $(m_1, m_2)$ is a good $(1,2)$-pair as follows.

If $P_1(m_1)$ does not exist, there are no such good pairs. If it exists, color black the vertices of $M$ that either belong to $P_1(m_1)$ or have a neighbor in $P_1(m_1)$. Color all other vertices white. (We can do this in $O(|V(G)|^2)$) Then for each $m_2 \in M \cup \{b_2\}$, test whether $P_2(m_2)$ exists and contains no black vertices. We do this for all $m_1$ and get a set of all $(1,2)$-good pairs. In similar way we calculate all good $(1,3)$-pair and $(2,3)$-pairs (in $O(|V(G)|^3)$ time).

Now, we examine all triples $m_1, m_2, m_3$ such that $m_i \in M \cup \{b_i\}$ and test whether $(m_i, m_j)$ is a good $(i, j)$-pair. If we find a triple such that all three pairs are good, we output that G contains a pyramid and stop.

If after examining all choices of $b_1, b_2, b_3, s_1, s_2, s_3$ we find no pyramid, output that $G$ contains no pyramid. Since there are $O(|V(G)|^6)$ such choices and it takes a time of $O(|V(G)|^3)$ to analyze each one, the total time is $O(|V(G)|^9)$.

\TODO{some proofs}


\subsubsection{Jewels}

% \begin{wrapfigure}[7]{r}{0.35\textwidth}
\begin{wrapfigure}{r}{0.35\textwidth}
	\centering\begin{tikzpicture}[scale=0.7,simplegraph]
  \def\ngon{5}
  \node[regular polygon,regular polygon sides=\ngon,minimum size=3cm, draw=none] (p) {};
  \foreach\x in {1,...,\ngon}{\node[] (p\x) at (p.corner \x){$v_\the\numexpr\intcalcMod{\x+3}{5}+1$};}
  %p1 - v5, p2 - v1 ...
  \draw (p1) to (p2);
  \draw (p2) to (p3);
  \draw (p3) to (p4);
  \draw (p4) to (p5);
  \draw (p1) to (p5);

  \draw[dashed] (p2) to (p5);
  \draw[dashed] (p2) to (p4);
  \draw[dashed] (p3) to (p5);

  \tikzset{decoration={snake,amplitude=.6mm,segment length=6mm,
        post length=0mm,pre length=0mm}}
  \draw[decorate] (p2) to [out=-100, in=180] (0, -5) to [out=0, in=-80] (p5);
  \node[draw=none] at (0, -4.7) {$P$};
\end{tikzpicture}
	\caption{An example of a jewel.}
	\vspace{-1.5cm}
\end{wrapfigure}


Five vertices $v_1, \ldots, v_5$ and a path $P$ form a \emph{jewel} iff:

\begin{itemize}
	\item $v_1, \ldots, v_5$ are distinct vertices.
	\item $v_1v_2, v_2v_3, v_3v_4, v_4v_5, v_5v_1$ are edges.
	\item $v_1v_3, v_2v_4, v_1,v_4$ are nonedges.
	\item $P$ is a path between $v_1$ and $v_4$, such that $v_2, v_3, v_5$ have no neighbors in its inside.
\end{itemize}

Most obvious way to find a jewel would be to enumerate all choices of $v_1, \ldots v_5$, check if a choice is correct and if it is try to find a path $P$ as required. This gives us a time of $O(|V|^7)$. We could speed it up to $O(|V|^6)$ with more careful algorithm, but since whole algorithms takes time $O(|V|^9)$ and our testing showed that time it takes to test for jewels is negligible we decided against it.

\subsubsection{Configurations of type $\T_1$}

A configuration of type $\T_1$ is a hole of length 5. To find it we simply iterate all choices of paths of length of 4, and check if there exists a fifth vertex to complete the hole. See \cref{Optimizations} for more implementation details.

\subsubsection{Configurations of type $\T_2$}

% \begin{wrapfigure}[15]{r}{0.35\textwidth}
\begin{wrapfigure}{r}{0.35\textwidth}
	\centering\begin{tikzpicture}[scale=0.7,simplegraph]
  \node(v1) at (0,0) {$v_1$};
  \node(v2) at (1.5,0) {$v_2$};
  \node(v3) at (3,0) {$v_3$};
  \node(v4) at (4.5,0) {$v_4$};

  \draw (v1) to (v2);
  \draw (v2) to (v3);
  \draw (v3) to (v4);

  \tikzset{decoration={snake,amplitude=.6mm,segment length=6mm,
        post length=0mm,pre length=0mm}}
  \draw[decorate] (v1) to [out=90, in=180] (4.5/2, 3) to [out=0, in=90] (v4);
  \node[draw=none] at (4.5/2, 3.5) {$P$};

  \draw[dashed] (v2) to (.7,2.5);
  \draw[dashed] (v3) to (.7,2.5);
  \draw[dashed] (v2) to (3.8,2.5);
  \draw[dashed] (v3) to (3.8,2.5);

  \node[minimum size=4mm](x1) at (1.2, -2){};
  \node[minimum size=4mm](x2) at (3.3, -2){};
  \node[draw,dotted,inner sep=3pt, circle,yscale=.5, fit={(x1) (x2)},label=below:{$X$}] {};

  \draw[dashed] (x1) to (x2);
  \draw(v1) to (x1);
  \draw(v2) to (x1);
  \draw(v4) to (x1);
  \draw(v1) to (x2);
  \draw(v2) to (x2);
  \draw(v4) to (x2);

  % \draw (a) to (b2);
\end{tikzpicture}%
	\caption{An example of a $\T_2$.}%
	\vspace{-1.5cm}
\end{wrapfigure}

A configuration of type $\T_2$ is a tuple $(v_1, v_2, v_3, v_4, P, X)$, such that:
\begin{itemize}
	\item $v_1v_2v_3v_4$ is a path in $G$.
	\item $X$ is an anticomponent of the set of all $\{v_1, v_2, v_4\}$-complete vertices.
	\item $P$ is a path in $G\setminus(X \cup \{v_2, v_3\})$ between $v_1$ and $v_4$ and no vertex in $P^*$ is $X$-complete or adjacent to $v_2$ or adjacent to $v_3$.
\end{itemize}

Checking if configuration of type $\T_2$ exists in our graph is straightforward: we enumerate all paths $v_1\ldots v_4$, calculate set of all $\{v_1, v_2, v_4\}$-complete vertices and its anticomponents. Then, for each anticomponent $X$ we check if required path $P$ exists.

To prove that existence of $\T_2$ configuration implies that the graph is not berge, we will need the following Roussel-Rubio lemma:

\begin{lemma}[Roussel-Rubio Lemma \cite{RR01,MC05}]\label{lem:Roussel-Rubio}
	Let $G$ be Berge, $X$ be an anticonnected subset of $V(G)$, $P$ be an odd path $p_1\ldots p_n$ in $G\setminus X$ with length at least 3, such that $p_1$ and $p_n$ are $X$-complete and $p_2, \ldots, p_{n-1}$ are not. Then:
	\begin{itemize}
		\item $P$ is of length at least 5 and there exist nonadjacent $x, y \in X$, such that there are exactly two edges between $x, y$ and $P^*$, namely $xp_2$ and $yp_{n-1}$,
		\item or $P$ is of length 3 and there is an odd antipath joining internal vertices of $P$ with interior in $X$.
	\end{itemize}
	\TODO{We may use this lemma quite often, might want to provide proof if so.}
\end{lemma}

Now, we shall prove the following:

\begin{lemma}
	If $G$ contains configuration of type $\T_2$ then $G$ is not Berge.
\end{lemma}
\begin{proof}
	Let $(v_1, v_2, v_3, v_4, P, X)$ be a configuration of type $\T_2$. Let us assume that $G$ is not Berge and consider the following:
	\begin{itemize}
		\item If $P$ is even, then $v_1, v_2, v_3, v_4, P, v_1$ is an odd hole,
		\item If $P$ is of length 3. \todo{I merged a couple of proofs from \cite{MC06}, check in the morning if this is correct.} Let us name its vertices $v_1, p_2, p_3, v_4$. It follows from \cref{lem:Roussel-Rubio}, that there exists an odd antipath between $p_2$ and $p_3$ with interior in $X$. We can complete it with $v_2p_2$ and $v_2p_3$ into an odd antihole.
		\item If $P$ is odd with the length of at least 5 \todo{check in the morning}, it follows from \cref{lem:Roussel-Rubio} that we have $x, y \in X$ with only two edges to $P$ being $xp_2$ and $yp_{n-1}$. This gives us an odd hole: $v_2, x, p_2, \ldots, p_{n-1}, y, v_2$.
	\end{itemize}
\end{proof}

\subsubsection{Configurations of type $\T_3$}

\begin{wrapfigure}{r}{0.35\textwidth}
	\centering\begin{tikzpicture}[scale=.7,simplegraph]
  \def\ngon{6}
  \node[regular polygon,regular polygon sides=\ngon,minimum size=3cm, draw=none] (p) {};
  \foreach\x in {1,...,\ngon}{\node[] (p\x) at (p.corner \x){$v_\the\numexpr\intcalcMod{\x+3}{6}+1$};}

  \draw (p3) to (p4);
  \draw (p5) to (p6);
  \draw (p3) to (p6);
  \draw (p4) to (p5);
  \draw (p5) to (p1);
  \draw (p6) to (p2);

  \draw[dashed] (p3) to (p5);
  \draw[dashed] (p4) to (p6);
  \draw[dashed] (p3) to (p1);
  \draw[dashed] (p4) to (p1);
  \draw[dashed] (p3) to (p2);
  \draw[dashed] (p4) to (p2);
  \draw[dashed] (p6) to (p1);

  \node[minimum size=4mm](x1) at (-1, -4.5){};
  \node[minimum size=4mm](x2) at (1, -4.5){};
  \node[draw,dotted,inner sep=3pt, circle,yscale=.5, fit={(x1) (x2)},label=below:{$X$}] {};

  \draw[dashed] (x1) to (x2);

  \draw (p3) to[out=-90] (x1);
  \draw (p3) to[out=-90, in=150] (x2);

  \draw (p4) to (x1);
  \draw (p4) to (x2);

  \draw (p1) to[out=0, in=0] (x1);
  \draw (p1) to[out=0, in=0] (x2);

  \draw[dashed] (p5) to (x1);
  \draw[dashed] (p6) to (x2);

  \tikzset{decoration={snake,amplitude=.6mm,segment length=4mm,
        post length=0mm,pre length=0mm}}
  \draw[decorate] (p1) to [out=90, in=0] (0, 4) to [out=180, in=90] (p2);
  \node[draw=none] at (0, 4.5) {$P$};

  \draw[dashed] (p4) to (0, 4.1);
  \draw[dashed] (x2) [out=100, in=-90] to (0, 4.1);
\end{tikzpicture}%
	\caption{An example of a $\T_3$.}%
	\vspace{-1.5cm}
\end{wrapfigure}

A configuration of type $\T_3$ is a sequence $v_1, \ldots, v_6$, $P$, $X$, such that:
\begin{itemize}
	\item $v_1, \ldots v_6$ are distinct vertices.
	\item $v_1v_2$, $v_3v_4$, $v_1v_4$, $v_2v_3$, $v_3v_5$, $v_4v_6$ are edges, and $v_1v_3$, $v_2v_4$, $v_1v_5$, $v_2v_5$, $v_1v_6$, $v_2v_6$, $v_4v_5$ are nonedges.
	\item $X$ is an anticomponent of the set of all $\{v_1, v_2, v_5\}$-complete vertices, and $v_3$, $v_4$ are not $X$-complete.
	\item $P$ is a path of $G \setminus ( X \cup \{v_1, v_2, v_3, v_4\} )$ between $v_5$ and $v_6$ and no vertex in $P*$ is $X$-complete or adjacent to $v_1$ or adjacent to $v_2$.
	\item If $v_5v_6$ is an edge, then $v_6$ is not $X$-complete.
\end{itemize}

The following algorithm with running time of $O(|V(G)|^6)$ checks whether $G$ contains a configuration of type $T_3$:

For each triple $v_1, v_2, v_5$ of vertices such that $v_1v_2$ is an edge and $v_1v_5, v_2v_5$ are nonedges find the set $Y$ of all $\{v_1, v_2, v_5\}$-complete vertices. For each anticomponent $X$ of $Y$ find the maximal connected subset $F'$ containing $v_5$ such that $v_1, v_2$ have no neighbors in $F'$ and no vertex of $F'\setminus\{v_5\}$ is $X$-complete. Let $F$ be the union of $F'$ and the set of all $X$-complete vertices that have a neighbor in $F'$ and are nonadjacent to all of $v_1, v_2$ and $v_5$.

Then, for each choice of $v_4$ that is adjacent to $v_1$ and not to $v_2$ and $v_5$ and has a neighbor in $F$ (call it $v_6$) and a nonneibhbor in $X$, we test whether there is a vertex $v_3$, adjacent to $v_2, v_4, v_5$ and not to $v_1$, with a nonneibhbor in $X$. If there is such a vertex $v_3$, find $P$ -- a path from $v_6$ to $v_5$ with interior in $F'$ and return that $v_1, \ldots v_6, P, X$ is a configuration of type $\T_3$. If we exhaust our search and find none, report that graph does not contain it.

To see that the algorithm below has a running time of $O(|V(G)|^6)$, let us note that for each triple $v_1, v_2, v_5$ we examine, of which there are $O(|V(G)|^3)$, there are linear many choices of $X$, each taking $O(|V(G)|^2)$ time to process and generating a linear many choices of $v_4$ which take a linear time to process in turn. This gives us the total running time of $O(|V(G)|^6)$.

We will skip the proof that each graph containing $\T_3$ is not Berge. See section6 6.7 of \cite{MC05} for the proof.

\subsection{Amenable holes.}
\label{AmenableHoles}

\begin{theorem}
	\label{thm:amenableHoles}
	Let $G$ be a graph, such that $G$ and $\overline{G}$ contain no Pyramid, no Jewel and no configuration of types $\T_1, \T_2$ or $\T_3$. Then every shortest hole in $G$ is amenable.
	% \TODO{Any ideas on what else to say here? List all 9 steps?}
	% \TODO{Dobre pytanie. Te 9 kroków to da się zawrzeć w 1-2 zdaniach każde? Jeśli nie, to tylko napisałbym ogólnikowo, że "jest sekwencja kroków, która to robi, tu jest odnośnik"}
\end{theorem}

The proof of this theorem is quite technical and we will not discuss it here. See section 8 of \cite{MC05} for the proof.

With \cref{thm:amenableHoles} we can describe the rest of the algorithm.

\begin{alg}[List possible near cleaners, 9.2 of \cite{MC05}]
	\label{alg:listNearCleaners}
	Input: A graph $G$.

	\noindent Output: $O(|V(G)|^5)$ subsets of $V(G)$, such that if $C$ is an amenable hole in $G$, then one of the subsets is a near-cleaner for $C$.
\end{alg}
\begin{algtext}
	We will call a triple $(a, b, c)$ of vertices \emph{relevant} if $a, b$ are distinct (possibly $c \in \{a, b\}$) and $G[\{a,b,c\}]$ is an independent set.

	Given a relevant triple $(a, b, c)$ we can compute the following:
	\begin{itemize}
		\item $r(a,b,c) \leftarrow$~the cardinality of the largest anticomponent of $N(a, b)$, that contains a nonneibhbor of $c$, or 0, if $c$ is $N(a, b)$-complete.
		\item $Y(a,b,c) \leftarrow$~the union of all anticomponents of $N(a, b)$ that have cardinality strictly greater than $r(a, b, c)$.
		\item $W(a, b, c) \leftarrow$~the anticomponent of $N(a,b) \cup \{c\}$ that contains $c$.
		\item $Z(a, b, c) \leftarrow$~the set of all $Y(a, b, c) \cup W(a,b,c)$-complete vertices.
		\item $X(a, b, c) \leftarrow$~$Y(a,b,c) \cup Z(a,b,c)$.
	\end{itemize}

	For every two adjacent vertices $u, v$ compute the set $N(u, v)$ and list all such sets.
	For each relevant triple $(a,b,c)$ compute the set $X(a,b,c)$ and list all such sets.

	Output all subsets of $V(G)$ that are the union of a set from the first list and a set from the second list.
\end{algtext}

To prove the correctness of \cref{alg:listNearCleaners} we will need the following theorem.

\begin{theorem}[9.1 of \cite{MC05}]
	\label{thm:91}
	Let $C$ be a shortest odd hole in $G$, with length at least 7. Then there is a relevant triple $(a, b, c)$ of vertices such that
	\begin{itemize}
		\item the set of all $C$-major vertices not in $X(a, b, c)$ is anticonnected
		\item $X(a, b, c) \cap V(C)$ is a subset of the vertex set of some 3-vertex path of $C$.
	\end{itemize}
\end{theorem}
\begin{proof}
	\TODO{do we want it here? 1-2 pages long}
\end{proof}

Let us suppose that $C$ is an amenable hole in $G$. By \cref{thm:91}, there is a relevant triple $(a, b, c)$ satisfying that theorem. Let $T$ be the set of all $C$-major vertices not in $X(a,b,c)$. From \cref{thm:91} we get that $T$ is anticonnected. Since $C$ is amenable, there is an edge $uv$ of $C$ that is $T$-complete, and therefore $T \subseteq N(u, v)$. But then $N(u, v) \cup X(a, b, c)$ is a near-cleaner for $C$. Therefore the output of the \cref{alg:listNearCleaners} is correct.

\begin{alg}[Test possible near clener, 5.1 of \cite{MC05}]
	\label{alg:testNearCleaner}
	Input: A graph $G$ containing no simple bad structure \todo{define this somewhere}, and a subset $X \subseteq V(G)$.

	\noindent Output: Determines one of the following:
	\begin{itemize}
		\item $G$ has an odd hole
		\item There is no shortest odd hole $C$ such that $X$ is a near-cleaner for $C$.
	\end{itemize}
\end{alg}
\begin{algtext}
	For every pair $x, y \in V(G)$ of distinct vertices find shortest path $R(x, y)$ between $x, y$ with no internal vertex in $X$. If there is one, let $r(x, y)$ be its length, if not, let $r(x, y)$ be infinite.

	For all $y_1 \in V(G)\setminus X$ and all 3-vertex paths $x_1-x_2-x_3$ of $G\setminus y_1$ we check the following:
	\begin{itemize}
		\item $R(x_1, y_1), R(x_2, y_2)$ both exist -- define $y_2$ as the neighbor of $y_1$ in $R(x_2, y_1)$.
		\item $r(x_2, y_1) = r(x_1, y_1) + 1 = r(x_1, y_2)$ ($=n$ say)
		\item $r(x_3, y_1), r(x_3, y_2) \geq n$
	\end{itemize}

	If there is such a choice of $x_1$, $x_2$, $x_3$, $y_1$ then we output that there is an odd hole. If not, we report that there is no shortest odd hole $C$ such that $X$ is a near-cleaner for $C$.
\end{algtext}

\begin{wrapfigure}{r}{0.35\textwidth}
	\centering\begin{tikzpicture}[scale=0.7,simplegraph]
  \node(x1) at (0,0) {$x_1$};
  \node(x3) at (1.5,0) {$x_3$};
  \node(x2) at (3,0) {$x_2$};

  \node (y1) at (.6, 3) {$y_1$};
  \node (y2) at (2.4, 3) {$y_2$};

  \draw (x1) to (x3);
  \draw (x2) to (x3);

  \tikzset{decoration={snake,amplitude=.6mm,segment length=4mm,
        post length=0mm,pre length=0mm}}
  \draw[decorate, color=c1] (x1) to (y1);
  \draw[decorate, color=c2] (x2) to (y2);
  \draw[color=c2] (y1) to (y2);

  \node[draw=none, color=c2] at (3.8, 2.5) {$R(x_2, y_1)$};
  \node[draw=none, color=c1] at (1.4, 1.5) {\small$R(x_1, y_1)$};

  \node[draw=none, minimum size=4mm](inx1) at (.8, -1.1){};
  \node[draw=none, minimum size=4mm](inx2) at (2.2, -1.1){};
  \node[draw,inner sep=3pt, circle,yscale=.5, fit={(inx1) (inx2)},label=below:{$X$}] {};
  \node[draw,dotted,inner sep=5pt, circle,yscale=.5, fit={(inx1) (inx2) (x3) (x1) (x2)}] {};

  % \draw[dashed] (x1) to (x2);
  % \draw(v1) to (x1);
  % \draw(v2) to (x1);
  % \draw(v4) to (x1);
  % \draw(v1) to (x2);
  % \draw(v2) to (x2);
  % \draw(v4) to (x2);

  % \draw (a) to (b2);
\end{tikzpicture}%
	\caption{An odd hole is found}%
	\vspace{-0.5cm}
\end{wrapfigure}

Below we will prove that if the \cref{alg:testNearCleaner} reports an odd hole in $G$, there indeed is one. The proof of the correctness of the other possible result is more complicated, see section 4 and theorem 5.1 of \cite{MC05}.

\begin{proof}
	Let us suppose that there is a choice of $x_1$, $x_2$, $x_3$, $y_1$ satisfying the three conditions in the \cref{alg:testNearCleaner} and let $y_2$ and $n$ be defined as in there. We claim that $G$ contains an odd hole.

	Let $R(x_1, y_1) = p_1-\ldots -p_n$, and let $R(x_2, y_1) = q_1-\ldots -q_{n+1}$, where $p_1 = x_1$, $p_n = q_{n+1} = y_1$, $q_1 = x_2$ and $q_n = y_2$. From the definition of $R(x_1, y_1)$ and $R(x_2, y_1)$, none of $p_2, \ldots, p_{n-1}, q_2, \ldots, q_n$ belong to $X$. Also, from the definition of $y_1$, $y_1 \notin X$.

	Since $r(x_1, y_1) = r(x_2, y_1) - 1$ and since $x_1$, $x_2$ are nonadjacent it follows that $x_2$ does not belong to $R(x_1, y_1)$ and $x_1$ does not belong to $R(x_2, y_1)$. Since $r(x_3, y_1), r(x_3, y_2) \geq n (= r(x_1, y_2))$ it follows that $x_3$ does not belong to $R(x_1, y_1)$ or to $R(x_2, y_1)$, and has no neighbors in $R(x_1, y_1) \cup R(x_2, y_1)$ other than $x_1$, $x_2$. Since $r(x_1, y_2) = n$ we get that $y_2$ does not belong to $R(x_1, y_1)$.

	We claim first that the insides of paths $R(x_1, y_1)$ and $R(x_2, y_1)$ have no common vertices. For suppose that there are $2 \leq i \leq n-1$ and $2 \leq j \leq n$ that $p_i = q_j$. Then the subpaths of these two paths between $p_i$, $y_1$ are both subpaths of the shortest paths, and therefore have the same length, that is $j=i+1$. So $p_1-\ldots-p_1-q_{j+1}-\ldots-q_n$ contains a path between $x_1$, $y_2$ of length $\leq n-2$, contradicting that $r(x_1, y_2) = n$. So $R(x_1, y_1)$ and $R(x_2, y_1)$ have no common vertex except $y_1$.

	If there are no edges between $R(x_1, y_1) \setminus y_1$ and $R(x_2, y_1) \setminus y_1$ then the union of these two paths and a path $x_1-x_3-x_2$ form an odd hole, so the answer is correct.

	Suppose that $p_iq_j$ is an edge for some $1\leq i \leq n-1$ and $1 \leq j \leq n$. We claim $i \geq j$. If $j = 1$ this is clear so let us assume $j > 1$. Then there is a path between $x_1$, $y_2$ within $\{p_1, \ldots , p_i, q_j, \ldots q_n\}$ which has length $\leq n-j+1$ and has no internal vertex in $X$ (since $j > 1$); and since $r(x_1, y_2) = n$, it follows that $n-j+i \geq n$, that is, $i \geq j$ as claimed.

	Now, since $x_1$, $x_2$ are nonadjacent $i \geq 2$. But also $r(x_2, y_1) \geq n$ and so $j+n-i \geq n$, that is $j \geq i$. So we get $i=j$. Let us choose $i$ minimum, then $x_3-x_1-\ldots-p_i-q_i-\ldots-x_2-x_3$ is an odd hole, which was what we wanted.
\end{proof}

\subsection{Summary}

\TODO{todo?}