\begin{subfigure}{.5\textwidth}
  \centering\begin{tikzpicture}[scale=.7,simplegraph]
    \foreach\y in {1,...,4}{
        \foreach\x in {1,...,4}{\node[] (\y\x) at (\y*2, \x*2){};}
      }

    \foreach \x/\y in {11/12, 21/22, 31/32, 41/42,
        12/13, 22/23, 32/33, 42/43,
        13/14, 23/24, 33/34, 43/44,
        11/21, 12/22, 13/23, 14/24,
        21/31, 22/32, 23/33, 24/34,
        31/41, 32/42, 33/43, 34/44}
      {\draw(\x) to (\y);}
  \end{tikzpicture}
  \caption{Lattice graph}
  \label{fig:lattice}
\end{subfigure}
\begin{subfigure}{.5\textwidth}
  \centering\begin{tikzpicture}[scale=.7,simplegraph]
    \foreach\y in {1,...,4}{
        \foreach\x in {1,...,4}{\node[] (\y\x) at (\y*2, \x*2){};}
      }

    \foreach \x/\y in {11/12, 21/22, 31/32, 41/42,
        12/13, 22/23, 32/33, 42/43,
        13/14, 23/24, 33/34, 43/44,
        11/21, 12/22, 13/23, 14/24,
        21/31, 22/32, 23/33, 24/34,
        31/41, 32/42, 33/43, 34/44}
      {\draw(\x) to (\y);}

    \draw[in=-120, out=120](11) to (13);
    \draw[in=-120, out=120](12) to (14);
    \draw[in=-120, out=120](11) to (14);
    \draw[in=-120, out=120](21) to (23);
    \draw[in=-120, out=120](22) to (24);
    \draw[in=-120, out=120](21) to (24);
    \draw[in=-120, out=120](31) to (33);
    \draw[in=-120, out=120](32) to (34);
    \draw[in=-120, out=120](31) to (34);
    \draw[in=-120, out=120](41) to (43);
    \draw[in=-120, out=120](42) to (44);
    \draw[in=-120, out=120](41) to (44);

    \draw[in=150, out=30](11) to (31);
    \draw[in=150, out=30](11) to (41);
    \draw[in=150, out=30](21) to (41);
    \draw[in=150, out=30](12) to (32);
    \draw[in=150, out=30](12) to (42);
    \draw[in=150, out=30](22) to (42);
    \draw[in=150, out=30](13) to (33);
    \draw[in=150, out=30](13) to (43);
    \draw[in=150, out=30](23) to (43);
    \draw[in=150, out=30](14) to (34);
    \draw[in=150, out=30](14) to (44);
    \draw[in=150, out=30](24) to (44);

  \end{tikzpicture}
  \caption{Rook graph}
  \label{fig:rook}
\end{subfigure}
\begin{subfigure}{.5\textwidth}
  \centering\begin{tikzpicture}[scale=.7,simplegraph]
    \foreach\y in {1,...,4}{
        \foreach\x in {1,...,4}{\node[] (\y\x) at (\y*2, \x*2){};}
      }

    \foreach \x/\y in {11/23,11/32,14/33,14/22,
        41/33,41/22,44/32,44/23,
        21/33,21/42,21/13,24/12,
        24/32,24/43,31/12,31/23,
        31/43,34/13,34/22,34/42,
        12/33,22/43,32/13,42/23}
      {\draw(\x) to (\y);}
  \end{tikzpicture}
  \caption{Knight's graph}
  \label{fig:knight}
\end{subfigure}
\begin{subfigure}{.5\textwidth}
  \centering\begin{tikzpicture}[scale=1.1,simplegraph]
    \HyperCube{4}
  \end{tikzpicture}
  \caption{Hypercube, $n=16$}
  \label{fig:hypercube}
\end{subfigure}
\begin{subfigure}{\textwidth}
  \centering\begin{tikzpicture}[scale=0.8,simplegraph]
    \def\ngon{5}
    \node[regular polygon,regular polygon sides=\ngon,minimum size=3cm, draw=none] (p) {};
    \foreach\x in {1,...,\ngon}{\node[] (p\x) at (p.corner \x){};}

    \node[regular polygon,regular polygon sides=\ngon,minimum size=3cm, draw=none] at (6.5, 0) (e) {};
    \foreach\x in {1,...,\ngon}{\node[] (e\x) at (e.corner \x){};}

    \draw (p1) to (p2);
    \draw (p1) to (p3);
    \draw (p1) to (p4);
    \draw (p1) to (p5);
    \draw (p2) to (p3);
    \draw (p2) to (p4);
    \draw (p2) to (p5);
    \draw (p3) to (p4);
    \draw (p3) to (p5);
    \draw (p4) to (p5);

    \draw (p1) to (e2);
    \draw (p5) to (e2);
    \draw (p3) to (e2);

    \draw(p4) to (e3);
    \draw(p4) to (e5);

  \end{tikzpicture}
  \caption{Example split graph, $n=10$}
  \label{fig:split}
\end{subfigure}
