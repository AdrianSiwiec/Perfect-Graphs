\chapter{Coloring Square-free Berge graphs}
\label{ch:coloringSquareFree}

The following appendix is based on the paper \citetitle{coloringSquareFree} by \citeauthor{coloringSquareFree} \cite{coloringSquareFree}.

\section{Introduction}
We give a brief overview of the square-free perfect graph coloring algorithm and a pseudocode thereof. The work is not intended to be self-contained and will be impossible to fully understand without \cite{coloringSquareFree} at hand. But when reading \cite{coloringSquareFree}, one may pose a question of what sort of an algorithm emerges from it, and we believe the pseudocode below gives a first overview and is a major step required before an attempt to implement it would be made. For some, looking at the pseudocode could provide a better way to analyze and possibly simplify the algorithm.

What we present would not be trivial to implement. For example in many places we say ''choose $x$, so that \dots'', without specifying how to do it. A most glaring example is calculating an inverse of a line graph (in \Cref{alg:growingJStrip}) which is a complicated algorithm in and of itself, see \cite{Liu2014} for reference.

Recall, that a prism (\cref{def:prism}) is a graph consisting of two disjoint triangles and two disjoint paths between them.

\begin{defnTwo}[Artemis graph]
  A graph $G$ is an \emph{Artemis graph} if and only if it contains no odd hole, no antihole of length at least five, and no prism.
\end{defnTwo}

In 2003 Maffray et al. published a paper on Artemis graphs where they show polynomial algorithms to recognize and color them \cite{Maffray2006}. In 2005 Maffray et al. published a paper on Artemis graphs \cite{Maffray2005} they showed an polynomial algorithm that checks if a perfect graph contains a prism, and returns it if so.

\begin{theorem}{\cite{Maffray2005}}
  \label{thm:getPrism}
  There is a polynomial algorithm, that given a perfect graph $G$, returns an induced prism $K$ of $G$, or an answer that $G$ contains no prisms.
\end{theorem}


In 2009 Maffray et al. published a paper on coloring Artemis graphs in time of $O(|V|^2|E|)$ \cite{coloringArtemis}. The paper contains simple pseudocode of the algorithm.

\begin{theorem}{\cite{coloringArtemis}}
  \label{thm:colorArtemis}
  There is a polynomial algorithm, that given an Artemis graph $G$, returns a $\omega(G)$-coloring of $G$.
\end{theorem}

So, we will focus on square-free perfect graphs that are not Artemis graphs. Let us notice, that every antihole of length at least 6 contains a square, therefore a square-free perfect graph that is not an Artemis graph must contain a prism. We will use this fact extensively.

Let us state a few definitions.

\begin{defnTwo}[subdivision]
  In a graph $G$, subdividing an edge $uv \in E(G)$ means removing the edge $uv$ and adding a new vertex $w$ and two new edges $uw, wv$. Starting with a graph $G$, the effect of repeatedly subdividing edges prioduces a graph $H$ called a \emph{subdivision of $G$}. Note that $V(G) \subseteq V(H)$.
\end{defnTwo}

\begin{defnTwo}[triad]
  In a graph $G$, a \emph{triad} is a set of three pairwise non-adjacent vertices.
\end{defnTwo}

\begin{defnTwo}[good partition]
  A \emph{good partition} of a graph $G$ is a partition $(K_1, K_2, K_3, L, R)$ of $V(G)$ such that:
  \begin{itemize}
    \item $L$ and $R$ are not empty, and $L$ is anticomplete to $R$,
    \item $K_1 \cup K_2$ and $K_2 \cup K_3$ are cliques,
    \item in the graph obtained from $G$ by removing all edges between $K_1$ and $K_3$, every path with one end in $K_1$, the other in $K_3$, and interior in $L$ contains a vertex from $L$ that is complete to $K_1$,
    \item either $K_1$ is anticomplete to $K_3$, or no vertex in $L$ has neighbors in both $K_1$ and $K_3$,
    \item for some $x \in L$ and $y \in R$, there is a triad of $G$ that contains $\{x, y\}$.
  \end{itemize}
\end{defnTwo}

The algorithm we present is derived from contructive proof of the following two theorems.

\begin{theorem}{}[Theorem 2.1 of \cite{coloringSquareFree}]
  Let $G$ be a square-free Berge graph. If $G$ contains a prism, then $G$ has a good partition.
\end{theorem}

\begin{theorem}{}[Lemma 2.2 of \cite{coloringSquareFree}]
  Let $G$ be a square-free Berge graph. Suppose that $V(G)$ has a good partition $(K_1, K_2, K_3, L, R)$. Let $G_1 = G \setminus R$, $G_2 = G \setminus L$ and for $i = 1,2$ let $c_i$ be an $\omega(G_i)$-coloring of $G_i$. Then an $\omega(G)$-coloring of $G$ can be obtained in polynomial time.
\end{theorem}

A few more structures are used throughout the algorithm. Let us define them.

For definitions \ref{def:majorNeiOfPrism} and \ref{def:localSubOfPrism}, let $K$ be a prism with triangles $\{a_1, a_2, a_3\}$ and $\{b_1, b_2, b_3\}$ and with paths $R_1, R_2, R_3$, where each $R_i$ has ends $a_i$ and $b_1$.

\begin{defnTwo}[major neighbor of a prism]
  A vertex $v \in V(G) \setminus K$ is a \emph{major neighbor} of pritm $K$ is and only if it has at least two neighbors in $\{a_1, a_2, a_3\}$ and at least two neighbors in $\{b_1, b_2, b_3\}$.
  \label{def:majorNeiOfPrism}
\end{defnTwo}

\begin{defnTwo}[local subset of a prism]
  A subset $X \subseteq V(K)$ is \emph{local} if and only if either $X \subseteq \{a_1, a_2, a_3\}$, or $X \subseteq \{b_1, b_2, b_3\}$, or $X \subseteq R_i$ for some $i \in \{1, 2, 3\}$.
  \label{def:localSubOfPrism}
\end{defnTwo}

\begin{defnTwo}[attachment]
  Let $F, K$ be an induced subgraphs of a graph $G$, with $V(F) \cap V(K) = \emptyset$. Any vertex $k \in V(K)$ that has a neighbor in $V(F)$ in a graph $G$ is called an \emph{attachment} of $F$ in $K$. Whenever any vertex $k \in V(K)$ has an attachment of $F$ we say that $F$ \emph{attaches} to $K$.
\end{defnTwo}

\begin{defnTwo}[hyperprism]
  A \emph{hyperprism} is a graph $H$, whose vertices can be partitioned into nine sets:
  $$
  \begin{tabular}{c c c}
    $A_1$ & $C_1$ & $B_1$ \\
    $A_2$ & $C_2$ & $B_2$ \\
    $A_3$ & $C_3$ & $B_3$
  \end{tabular}
  $$
  \noindent with the following properties:
  \begin{itemize}
    \item each of $A_1, A_2, A_3, B_1, B_2, B_3$ is nonempty,
    \item for distinct $i, j \in \{1, 2, 3\}$, $A_i$ is complete to $A_j$, and $B_i$ is complete to $B_j$, and there are no other edges between $A_i \cup B_i \cup C_i$ and $A_j \cup B_j \cup C_j$,
    \item for each $i \in \{1, 2, 3\}$, every vertex of $A_i \cup B_i \cup C_i$ belongs to a path between $A_i$ and $B_i$ with interior in $C_i$.
  \end{itemize}
\end{defnTwo}

Whenever we will speak about hyperprisms, we will denote its subsets as in the above definition, unless stated otherwise.

For a hyperprism $H$ we have a few more definitions:

\begin{defnTwo}[$i$-rung of a hyperprism]
  For each $i \in \{1, 2, 3\}$, any path from $A_i$ to $B_i$ with interior in $C_i$ is called an \emph{$i$-rung}.
\end{defnTwo}

\begin{defnTwo}[strip of a hyperprism]
  For each $i \in \{1, 2, 3\}$, the triple $(A_i, C_i, B_i)$ is called a \emph{strip} of the hyperprism.
\end{defnTwo}

\begin{defnTwo}[instance of a hyperprism]
  For each $i \in \{1, 2, 3\}$, let us pick any $i$-rung $R_i$. Then $R_1$, $R_2$, $R_3$ form a prism. Any such prism is called an \emph{instance} of a hyperprism.
\end{defnTwo}

From the definition of an instance of a hyperprism, we can see that any prism is also a hyperprism.

Let us note, that if a hyperprism $H$ contains no odd hole, all rungs have the same parity. We then call the hyperprism odd or even accordingly.

Given a graph $G$ that contains a hyperprism $H$, we can define a few more structures.

\begin{defnTwo}[major neighbor of a hyperprism]
  A vertex $v \in V(G) \setminus V(H)$ is a \emph{major neighbor} of $H$, if and only if it is a major neighbor of some instance of $H$.
\end{defnTwo}

\begin{defnTwo}[local subset of a hyperprism]
  A subset $X \subseteq V(H)$ is a local subset of a hyperprism $H$ if and only if either $X \subseteq A_i \cup A_2 \cup A_3$ or $X \subseteq B_1 \cup B_2 \cup B_3$ or $X \subseteq A_i \cup B_i \cup C_i$ for some $i \in \{1, 2, 3\}$.
\end{defnTwo}

\begin{defnTwo}[maximal hyperprism]
  A hyperprism $H$ is \emph{maximal} if and only if there is no hyperprism $H'$ in $G$, such that $V(H) \varsubsetneq V(H')$.
\end{defnTwo}




\Cref{alg:growHyperprism} for growing hyperprism is based upon the following theorem. Recall that by $K_4$ we denote a clique on 4 vertices.
\begin{theorem}[Lemma 3.3 of \cite{coloringSquareFree}]
  Let $G$ be a berge graph, let $H$ be a hyperprism in $G$ and let $M$ be the set of major neighbors of $H$ in $G$. Let $F$ be a component of $V(G) \setminus(V(H) \cup M)$, such that the set of attachments of $F$ in $H$ is not local. Then one can find in polynomial time one of the following
  \begin{itemize}
    \item a path $P$, with $\emptyset \neq V(P) \subseteq V(F)$, such that $V(H) \cup V(P)$ induces a hyperprism,
    \item a path $P$, with $\emptyset \neq V(P) \subseteq V(F)$, and for each $i \in \{1, 2, 3\}$ an $i$-rung $R_i$ of $H$, such that $V(P) \cup V(R_1) \cup V(R_2) \cup V(R_3)$ induces the line graph of a bipartite subdivision of $K_4$.
  \end{itemize}
  \label{thm:growingHyperprism}
\end{theorem}

If at any time during growing a hyperprism we encounter the latter outcome of the \Cref{thm:growingHyperprism} we stop it and instead focus on the newly found line graph of a bipartite subdivision of $K_4$. Algorithms \ref{alg:goodPartitionJStrip}, \ref{alg:goodPartitionSpecialK4}, \ref{alg:findSpecialK4} and \ref{alg:growingJStrip} take their roots in the following theorem.

\begin{theorem}[Theorem 6.1 of \cite{coloringSquareFree}]
  Let $G$ be a square-free Berge graph, and assume that $G$ contains the line graph of a bipartite subdivision of $K_4$. Then $G$ admits a good partition.
\end{theorem}

Before providing the pseudocode, we need a few more definitions.

\begin{defnTwo}[branch]
  Given a graph $G$, a \emph{branch} is a path whose interior vertices have degree 2 and whose ends have degree at leas 3. A \emph{branch-vertex} is any vertex of degree at least 3.
\end{defnTwo}

\begin{defnTwo}[appearance of a graph]
  Given a graph $G$, and \emph{appearance} of a graph $J$ is any induced subgraph of $G$ that is isomorphic to the line graph $L(H)$ of a bipartite subdivision $H$ of $J$. An appearance of $J$ is \emph{degenerate} if and only if either $J = H = K_{3, 3}$\footnote{$K_{n, m}$ is a complete bipartite graph with $n$ vertices on the one side and $m$ vertices on the other}, or $J = K_4$ and the four vertices of $J$ form a square in $H$.
\end{defnTwo}

Note that a degenerate appearance of a graph contains a square.

\begin{defnTwo}[overshadowed appearance]
  An appearance of $L(H)$ of $J$ in $G$ is \emph{overshadowed} if and only if there is a branch $B$ of $H$, of length at least 3, with ends $b_1, b_2$, such that some vertex of $G$ is non-adjacent in $G$ to at most one vertex in $\{b_1x \in V(L(H))$, for $x \in V(H), b_1x \in E(H)\}$ and at most one vertex in $\{b_2x \in V(L(H))$, for $x \in V(H), b_2x \in E(H)\}$.
\end{defnTwo}

\begin{defnTwo}[$J$-enlargement]
  An \emph{enlargement} of a 3-connected graph $J$, or a $J$-enlargement is any 3-connected graph $J'$ such that there is a proper induced subgraph of $J'$ that is isomorphic to a subdivision of $J$.
\end{defnTwo}

\begin{defnTwo}[$J$-strip system, $uv$-rung]
  Let $J$ be a 3-connected graph and let $G$ be a perfect graph. A \emph{$J$-strip system} $(S, N)$ in $G$ means
  \begin{itemize}
    \item for each edge $uv$ of $J$, a subset $S_{uv} = S_{vu}$ of $V(G)$,
    \item for each vertex $v$ of $J$, s subset $N_v$ of $V(G)$,
    \item $N_{uv} = S_{uv} \cap N_u$,
  \end{itemize}
  such that if we define a $uv$-rung to be a path $R$ of $G$ with ends $s, t$, where $V(R) \subseteq S_{uv}$, and $s$ is the unique vertex of $R$ in $N_u$, and $t$ is the unique vertex of $R$ in $N_v$, the following conditions are met:
  \begin{itemize}
    \item the sets $S_{uv}$, for $uv \in E(J)$ are pairwise disjoint,
    \item for each $u \in V(J)$, $N_u \subseteq \bigcup_{uv \in E(J)} S_{uv}$,
    \item for each $uv \in E(J)$, every vertex of $S_{uv}$ is in a $uv$-rung,
    \item for any two edges $uv, wx$ of $J$, with $u, v, w, x$ all distinct, there are no edges between $S_{uv}$ and $S_{wx}$,
    \item if $uv, uw \in E(J)$ with $v \neq w$, then $N_{uv}$ is complete to $N_{uw}$ and there are no other edges between $S_{uv}$ and $S_{uw}$,
    \item for each $uv \in E(J)$, there is a \emph{special} $uv$-rung, such that for every cycle $C$ of $J$, the sum of the lengths of the special $uv$-rungs for $uv \in E(C)$ has the same parity as $|V(C)|$.
  \end{itemize}
\end{defnTwo}

\noindent The vertex set of $(S, N)$, denoted by $V(S, N)$ is the set $\bigcup_{uv \in E(J)} S_{uv}$. Note that in general $N_{uv}$ is different from $N_{vu}$. On the other hand $S_{uv} = S_{vu}$.

A $J$-strip system has the following properties:
\begin{itemize}
  \item for distinct $u, v \in V(J)$, if $uv \in E(J)$, then $N_u \cap N_v \subseteq S_{uv}$ and if $uv \notin E(J)$, then $N_u \cap N_v = \emptyset$,
  \item for $uv \in E(J)$ and $w \in V(J)$, if $w \neq u$, then $S_{uv} \cap N_w = \emptyset$,
  \item for every $uv \in E(J)$, all $uv$-rungs have lengths of the same parity,
  \item for every cycle $C$ of $J$ and every choice of $uv$-rung for every $uv \in E(C)$, the sums of the lengths of the $uv$-rungs have the same parity as $|V(C)|$. In particular, for each edge $uv \in E(J)$, all $uv$-rungs have the same parity.
\end{itemize}

\begin{defnTwo}[choice of rungs]
  Given a $J$-strip system, by a \emph{choice of rungs} we mean the choice of one $uv$-rung for each edge $uv$ of $J$.
\end{defnTwo}

Given a square-free perfect graph $G$ and a $J$-strip system, for every choice of rungs, the subgraph of $G$ induced by their union is the line-graph of a bipartite subdivision of $J$.

\begin{defnTwo}[saturating $J$-strip system]
  $X \subseteq V(G)$ \emph{saturates} the strip system if and only if for every $u \in V(J)$ there is at most one neighbor $v$ of $u$ such that $N_{uv} \nsubseteq X$.
\end{defnTwo}

\begin{defnTwo}[major vertex w.r.t. a strip system]
  A vertex $v \in V(G) \setminus V(S, N)$  is \emph{major} with respect to (w.r.t.) the strip system if and only if the set of its neighbors saturates the strip system.
\end{defnTwo}

\begin{defnTwo}[major vertex w.r.t. some choice of rungs]
  A vertex $v \in V(G) \setminus V(S, N)$  is \emph{major} with respect to (w.r.t.) some choice of rungs if and only if the $J$-strip system defined by this choice of rungs is saturated by the set of neighbors of $v$.
\end{defnTwo}

\begin{defnTwo}[subset local w.r.t. a strip system]
  A subset $X \subseteq V(S, N)$ is \emph{local} with respect to (w.r.t.) the strip system, if and only if either $X \subseteq N_v$ for some $v \in V(J)$ or $X \subseteq S_{uv}$ for some $uv \in E(J)$.
\end{defnTwo}

The outline of the algorithm is as follows. Given a square-free perfect graph, first we test if it is an Artemis graph. If so, we color it according to \Cref{thm:colorArtemis}. If not, \Cref{thm:getPrism} gives us a prism $K$. We want to extend $K$ either to a maximal hyperprism or to the line graph of a bipartite subdivision of $K_4$. For each of them, we can construct a good partition, which we color recursively.

\section{Notation}

In the following algorithms we use a slightly different notation than before, with many concepts represented by inline symbols. This is intended to reduce the length of algorithm's text and simplify its analysis.

\begin{itemize}
	\item $a :\in X$, when $X$ is a set -- let $a$ be equal to any element of $X$,
	\item $a \xor b$, when $a$ and $b$ are logical values -- $a$ xor $b$,
	\item $V(X)$ -- vertices of structure $X$. Will be written as $X$ when obvious,
	\item $a - b$, when $a$ and $b$ are vertices -- $a$ and $b$ are neighbors,
	\item $a \cdots b$, when $a$ and $b$ are vertices -- $a$ and $b$ are not neighbors,
	\item $a - X$, when $a$ is a vertex and $X$ is a set of vertices -- $a$ has a neighbor in $X$,
	\item $a \cdots X$, when $a$ is a vertex and $X$ is a set of vertices -- $a$ has a nonneighbor in $X$,
	\item $a \blacktriangleleft  X$, when $a$ is a vertex and $X$ is a set of vertices -- $a$ is complete to $X$,
	\item $a \ntriangleleft X$, when $a$ is a vertex and $X$ is a set of vertices -- $a$ is anticomplete to $X$,
	\item $X \setComplete Y$, when $X$ and $Y$ are set of vertices -- $X$ is complete to $Y$,
	\item $X \setAntiComplete Y$, when $X$ and $Y$ are set of vertices -- $X$ is anticomplete to $Y$,
	\item $[n]$  -- $\{1, \ldots, n\}$,
	\item \LGBSK -- the line-graph of a bipartite subdivision of $K_4$.
\end{itemize}

Also, throughout the algorithms, we have many lines with asserts. These check some of the properties required before proceeding, and should all be true.

\clearpage
\section{Algorithms}

\graphAppendix{Color square-free perfect graph}{alg:colorSquareFree}
\begin{mynameforalgorithm}
  % \myalgorithmcommand
  \SetKwFunction{colorGraph}{\textsc{Color-Graph}}
  \Indm\nonl\colorGraph($G$)\\
  \KwData{$G$ -- square-free Berge graph}
  \KwResult{A $\omega(G)$-coloring of $G$}
  \Indp
  \If(\tcp*[f]{\Cref{thm:getPrism}}){$G$ is an Artemis graph}{
    \Return coloring of an Artemis graph $G$ \tcp*{\Cref{thm:colorArtemis}}
  }
  $H \gets$ an induced prism of $G$ \tcp*{\Cref{thm:getPrism}}
  \TODO{do we get $H$ ready from A.1.1?}

  \While{$P = $ \textsc{undefined}}{
    
    \If{$\exists$ a component of $G \setminus(H \cup M)$ with a set of attachments in $H$ not local}{
      $F \gets$ a minimal component of $G \setminus(H \cup M)$ with a set of attachments in $H$ not local

      $M \gets \{v: v$ is a major neighbor of H$\}$\;
      \TODO{how do we get M?}


      $H' \gets$ \textsc{Grow-Hyperprism}$(G, H, M, F)$\;
      \If{$H'$ is  a \LGBSK}{
        $J \gets H'$\;
        $(S, N) \gets$ a $J$-strip system\;
        \TODO{how?}
      }\Else{
        $H \gets H'$\;
      }
    }
    \Else(\tcp*[f]{$H$ is a maximal hyperprism}){
      $M \gets \{v: v$ is a major neighbor of H$\}$\;
      \If{$H$ is an even hyperprism}{
        $P \gets$ \textsc{Good-Partition-From-Even-Hyperprism}$(G, H, M)$
        \textbf{break}
      }
      \Else{
        $P \gets$ \textsc{Goog-Partition-From-Odd-Hyperprism}$(G, H, M)$\;
        \textbf{break}
      }
    }
    \If(\tcp*[f]{a $J$-strip system was encountered}){$J \neq$ \textsc{undefined}}{
      $J', (S', N') \gets$ \textsc{Growing-J-Strip}$(G, J, (S, N))$\;
      $M \gets$ a set of major vertives w.r.t. $(S, N)$\;
      \If{$J', (S', N')$ is a special $K_4$ system}{
        $P \gets$ \textsc{Good-Partition-From-Special-Strip-System}$(G, (S, N), M)$
      }\Else{
        $P \gets$ \textsc{Good-Partition-From-J-Strip-System}$(G, J, (S, N), M)$
      }
    }
  }
  \Return \textsc{Color-Good-Partition}$(G, P)$\;
\end{mynameforalgorithm}
\clearpage

\graphAppendix{Color good partition}{alg:colorGoodPartition}
\noindent% \usepackage{appendix}
\usepackage{todonotes}
\usepackage{wrapfig}
\usepackage{intcalc}
\usepackage{amsthm, amsfonts, thmtools, amsmath,amssymb}
\usepackage{cleveref}
\usepackage{floatflt}
\usepackage{pgfplots}

\usepackage{tikz}
\tikzset{
    simplegraph/.style={every node/.style={draw,circle, inner sep=0pt, minimum size=6mm}},
}
\usetikzlibrary{snakes}
\usetikzlibrary{shapes.geometric}
\usetikzlibrary{positioning, fit, calc}

\usepackage[
	style=numeric
]{biblatex}
\addbibresource{bbl.bib}

\newtheorem{theorem}{Theorem}[section]
\newtheorem{corollary}{Corollary}[theorem]
\newtheorem{lemma}[theorem]{Lemma}
\newtheorem{defn}{Definition}[chapter]
\renewcommand{\listtheoremname}{List of definitions}
\newtheoremstyle{break}
  {\topsep}{\topsep}%
  {\itshape}{}%
  {\bfseries}{}%
  {\newline}{}%
\theoremstyle{break}
\newtheorem{alg}{Algorithm}[section]
\newcommand*{\myproofname}{Begin}
\newenvironment{algtext}[1][\myproofname]{\begin{proof}[#1] $ $\newline \renewcommand*{\qedsymbol}{\(End\)}}{\end{proof}}

\newcommand{\TODO}{\todo[inline]}
\newcommand\Lovasz{Lovász }
\newcommand\T{\mathcal{T}}
\crefname{lemma}{Lemma}{Lemmas}

\definecolor{c1}{RGB}{55,126,184}
\definecolor{c2}{RGB}{255,127,0}
\definecolor{c3}{RGB}{77,175,74}


%For the appendix
\usepackage{graphicx}
\usepackage{caption}
\usepackage{subcaption}
\usepackage{placeins}
\usepackage{amsmath}
\usepackage{tikz}
\usepackage[utf8]{inputenc}
\usepackage{xcolor}
\newcommand\crule[3][black]{\textcolor{#1}{\rule{#2}{#3}}}
% \usepackage{hyperref} 
\usetikzlibrary{arrows}
\usepackage[linesnumbered,vlined]{algorithm2e}
\usepackage{amssymb}
\usepackage{graphicx}
 \usepackage{color}
 \usepackage{transparent}
 \usepackage{amssymb,stackengine}

% \definecolor{c1}{HTML}{8D75A0}
% \definecolor{c2}{HTML}{8B9CD6}
% \definecolor{c3}{HTML}{F4F1BB}
% \definecolor{c4}{HTML}{DB9D47}
% \definecolor{c5}{HTML}{FF784F}

\graphicspath{ {./img/} }

\SetNlSty{}{}{}
\SetKwInput{KwData}{Input}
\SetKwInput{KwResult}{Output}
\DontPrintSemicolon
\newcommand{\NULL}{\textsc{null}}
\SetAlgoVlined
\LinesNumbered

\SetKwBlock{IfNoEnd}{if}{}
\SetKwBlock{IfNoBegin}{}{End}

\let\oldnl\nl% Store \nl in \oldnl
\newcommand\nonl{\renewcommand{\nl}{\let\nl\oldnl}}% Remove line number for one line
\newcommand{\pushline}{\Indp}% Indent
\newcommand{\popline}{\Indm}% Undent

\newcommand{\Assert}[2][]{\nonl ASSERT: #2 \ifthenelse{\equal{#1}{}}{}{(as in #1)} \; }
% \newcommand{\Assert}[2][]{}
% \newcommand{\TODO}[1]{\nonl \textcolor{red}{\# TODO:} #1 \;}

\newcommand{\mycbox}[1]{\tikz{\path[draw=white,fill=white] (0,0) rectangle (8cm,.5cm);}}
\newcommand{\mycboxx}[1]{\tikz{\path[draw=red,fill=white] (0,0) rectangle (.5cm,.8cm);}}

\newcommand{\unfinishedAlg}[2][]{
\begin{minipage}[t]{\textwidth}
\vspace{0pt} 
\begin{algorithm}[H]
#2
\end{algorithm}
\vspace*{#1}
\mycbox{}
\end{minipage}
}

\SetKwFor{GreyWhile}{\transparent{0.4}{// while}}{\transparent{0.4}{do}}{}
\SetKwFor{GreyWhile}{\transparent{0.4}{// while}}{\transparent{0.4}{do}}{}
\SetKwIF{GreyIf}{GreyElseIf}{GreyElseAnother}{\transparent{0.4}{// if}}{\transparent{0.4}{then}}{\transparent{0.4}{// else if}}{\transparent{0.4}{else}}{}
\SetKwIF{GreyElse}{GreyElseIfDummy}{GreyElseAnotherDummy}{\transparent{0.4}{// else}}{}{}{}{}

% \newlength\foo
\newcommand{\continueAlg}[2][]{
	\begin{minipage}[t]{\textwidth}
    \vspace{0pt} 
% \settoheight\foo{ % for some reason doesn't work. leave for now.
    \begin{algorithm}[H]
    #2
		\end{algorithm}
% }
    \vspace*{#1}
    \mycboxx{}
    \vspace*{-#1}
		\end{minipage}
}

\newcommand{\LGBSK}{$L(BS(K_4))$}
\newcommand{\xor}{\veebar}
\newcommand{\nsquare}{\ensurestackMath{\stackinset{c}{}{c}{}{/}{\square}}}
\newcommand{\setComplete}{\ \blacksquare\ }
\newcommand{\setAntiComplete}{\ \nsquare\ }
\newcommand{\graphAppendix}[1]{
  \begin{alg}[#1]
  \end{alg}
  \vspace{-.8cm}
}

\unfinishedAlg[-.45cm]{
	\SetNoFillComment
	\SetKwFunction{colorGoodPartition}{\textsc{Color-Good-Partition}}
	\Indm\nonl\colorGoodPartition{$G, (K_1, K_2, K_3, L, R), c_1, c_2$}\\
	\KwData{$G$ -- square-free, Berge graph \newline
		$(K_1, K_2, K_3, L, R)$ -- good partition \newline
		$c_1$, $c_2$ -- colorings of $G \setminus R$ and $G \setminus L$ (possibly $\NULL$)}
	\KwResult{$\omega(G)$-coloring of $G$}
	\Indp
	$G_1 \gets G\setminus R$\;
	$G_2 \gets G\setminus L$\;
	\If{$c_1, c_2 = \NULL$}{
		$c_1 \gets$ \colorGraph{$G_1$}\;
		$c_2 \gets$ \colorGraph{$G_2$}\;
	}
	\ForEach{$u \in K_1 \cup K_2$ 
		}{relabel $c_2$, so that $c_1(u) = c_2(u)$}
	$B \gets \{ u \in K_3: c_1(u) \neq c_2(u) \}$\;
	\lIf{$B = \emptyset$ }{\Return $c_1 \cup c_2$ } 
	\ForEach{$h \in [2]$, $\textup{distinct colors}$ $i, j$
		}{ $G_h^{i,j} \gets $ subgraph induced on $G_h$ by $\{ v \in G_h : c_h(v) \in \{i,j\} \}$}
	\ForEach{$u \in K_3$}{
		$C_h^{i, j}(u) \gets$ component of $G_h^{i, j}$ containing $u$}
													
	\Assert{$C_h^{c_1(u), c_2(u)}(u) \cap K_2 = \emptyset$}  
													
	\If{$\exists u \in B, h \in [2]: C_h^{c_1(u), c_2(u)}(u) \cap K_1 = \emptyset $
		}{ $c_1' \gets c_1$ with colors $i$ and $j$ swapped in $C_1^{i, j}(u)$\;
		\Assert{$c_1'$ and $c_2$ agree on $K_1 \cup K_2$}
		\Assert{$\forall u \in K_3 \setminus B : c_1'(u) = c_1(u)$}
		\Assert{$c_1'(u) = j = c_2(u)$}
		\Return \colorGoodPartition{$G, K_1, K_2, K_3, L, R, c_1', c_2$}}
	\Else{
		$w \gets$ vertex of $B$ with nost neighbors in $K_1$\;
																							
		\Assert{$\forall u \in B: N(u) \cap K_1 \subset N(w) \cap K_1$}
																							
		relabel $c_1, c_2$, so that $c_1(w) = 1, c_2(w) = 2$\;
																							
		$P \gets$ chordless path $w - p_1 - \ldots - p_k - a$ in $C_1^{1, 2}(w)$ so that\; \nonl
		\pushline $k \geq 1$, $p_1 \in K_3 \cup L$, $p_2 \ldots p_k \in L$, $a \in K$, $c_1(a) \in [2]$ \; \popline
																							
		$Q \gets$ chordless path $w - q_1 - \ldots - q_l - a$ in $C_2^{1, 2}(w)$ so that\; \nonl
		\pushline $l \geq 1$, $q_1 \in K_3 \cup R$, $q_2 \ldots q_l \in R$, $a \in K$, $c_2(a) \in [2]$\; \popline
																							
		$i \gets c_1(a)$\;
		$j \gets 3 - i$\;
																							
		\Assert[Lemma 2.2.(3)]{ exactly one of the colors $1$ and $2$ appears in $K_1$ }
		\Assert{ $|P|$ and $|Q|$ have different parities }
		\Assert[Lemma 2.2.(4)]{ $p_1 \in K_3 \lor p_2 \in K_3$ }
		\Assert[Lemma 2.2.(5)]{ $\nexists y \in K_3: c_1(y) = 2 \land c_2(y) = 1$ }
																							    
		\If{$p_1 \in K_3$}{
			\Assert{$c_2(p_1) \notin [2]$}
			relabel $c_2$, so that $c_2(p_1) = 3$ \;
		}
	}
}

\continueAlg[-10cm]{
	\setcounter{AlgoLine}{23}
	\GreyElse{ \transparent{.4}{ // $\nexists u \in B, h \in [2]: C_h^{c_1(u), c_2(u)}(u) \cap K_1 = \emptyset $}}{
		\GreyIf{\transparent{.4}{$p_1 \in K_3$}}{
			\Assert{color $3$ does not appear in $K_2$}
			\Assert{color $3$ does not appear in $K_1$}
			\Assert{$C_2^{j, 3}(p_1) \cap K_1 = \emptyset$}
			$c_2' \gets c_2$ with colors $j$ and $3$ swapped in $C_2^{j, 3}(p_1)$ \;
			\Assert{$j = 2$} 
			\Return \colorGoodPartition{$G$, $K_1$, $K_2$, $K_3$, $L$, $R$, $c_1$, $c_2'$}
		}
		\Else{
			relabel $c_1$, so that $c_1(q_1) = 3$ \;
			\If{$3$ does not appear in $K_1$}{
				\Assert{$C_1^{j,3}(q_1) \cap K_1 = \emptyset$}
				\Assert{$j = 1$}
				$c_1' \gets c_1$ with colors $j$ and $3$ swapped in $C_1^{j,3}(q_1)$\;
				\Return \colorGoodPartition{$G$, $K_1$, $K_2$, $K_3$, $L$, $R$, $c_1'$, $c_2$}
				} \Else{
				\Assert{$q_1 \ntriangleleft \{a, a_3\}$}
				\Assert{$C_1^{i, 3}(q_1) \cap K_1 = \emptyset$}
				\Assert{$i = 1$}
				$c_1' \gets c_1$ with colors $i$ and $3$ swapped in $C_1^{i,3}(q_1)$\;
				\Return \colorGoodPartition{$G$, $K_1$, $K_2$, $K_3$, $L$, $R$, $c_1'$, $c_2$}
			}
		}
	}
}
\clearpage

\graphAppendix{Grow hyperprism}{alg:growHyperprism}
\noindent\unfinishedAlg[-.70cm]{
	\SetKwFunction{growHyperprism}{\textsc{Grow-Hyperprism}}
	\Indm\nonl\growHyperprism{$G, H, M, F$} \tcp*{Lemma 3.3} \nonl 
	\KwData{$G$ -- square-free, Berge graph \newline
		$H = (A_1, \ldots B_3)$ -- a hyperprism in $G$ \newline
		$M$ -- the set of major neighbors of $H$ in $G$ \newline
		$F$ -- a minimal component of $G \setminus(H \cup M)$ with a set of attachments in $H$ not local.
	}
	\KwResult{$H'$ -- a larger hyperprism, or \newline 
		$L$ -- a \LGBSK }
	\Indp
	$X \gets$ set of attachments of $F$ in $H$\;
	\If{$\exists i : X \cap C_i \neq \emptyset$}{
		relabel strips of $H$, so that $X \cap C_1 \neq \emptyset$\;
		$x_1 :\in X \cap C_1$\;
		\Assert{$X \cap S_2 \neq \emptyset$}
		$x_2 :\in X \cap S_2$\;
		$R_1 \gets 1$-rung of $H$, so that $x_1 \in V(R_1)$\;
		$R_2 \gets 2$-rung of $H$, so that $x_2 \in V(R_2)$\;
		$R_3 \gets$ a $ 3$-rung of $H$\;
		$\forall i \in [3]: a_i, b_i \gets $ends of $R_i$, so that $a_i \in A_i, b_i \in B_i$\;
		$K \gets $ a prism $(R_1, R_2, R_3)$\;
		\Assert[SPGT 10.5]{no vertex in $F$ is major w.r.t. $K$}
		$f_1 - \ldots - f_n \gets$ a minimal path  in $F$, so that \nonl\;
		\pushline { $f_1 \blacktriangleleft \{a_2, a_3\}$, \nonl\;
			$f_n - R \setminus \{a_1\}$ \nonl\;
			there are no other edges between $\{f_1, \ldots f_n\}$ and $V(K) \setminus \{a_1\}$ \;
      }\popline
    \Assert{$F = \{f_1, \ldots, f_n \}$}
    \Assert{$f_1 \blacktriangleleft A_3$ }
    $A_1' \gets A_1 \cup \{f_1\}$\;
    $C_1' \gets C_1 \cup \{f_2, \ldots, f_n\}$\;
    \Return $H' \gets (A_1', A_2, \ldots, B_3, C_1', C_2, C_3)$
  }
  \Else {
    relabel strips of $H$, so that there is $\{x_1 :\in A_1, x_2 :\in A_2\} \subset X$ that is not local\;
    find a path $x - f_1 - \ldots - f_n - x_2$\;
    \Assert{$F = \{f_1, \ldots f_n\}$}
    \If{$n$ \textup{is even and} $H$ \textup{is even, or} $n$ \textup{is odd and} $H$ \textup{is odd}}{
      \Assert{$f_1 - a_3 \xor f_n - b_3$}
      \If{$f_1 - a_3$} {
        $H' \gets $ mirrored $H$ -- every $A_i$ and $B_i$ are swapped\;
        % \TODO{check if $M$ and $F$ are OK}
        \Return \growHyperprism{$G, H', M, F$}\;
      } \Else {
        \If {$f_n \blacktriangleleft B_2 \cup B_3$}{
          $B_1' \gets B_1 \cup \{f_n\}$ \;
        }
      }
    }
  }
}

\unfinishedAlg[-.55cm]{
  \setcounter{AlgoLine}{19}
  \GreyElse{\transparent{.4}{// $\forall_{i\in[3]} X \cap C_i = \emptyset$}}{
    \GreyIf{\transparent{.4}{$n$ \textup{is even and} $H$ \textup{is even, or} $n$ \textup{is odd and} $H$ \textup{is odd}}}{
      \GreyElse{\transparent{.4}{// $f_n - b_3$}}{
        \GreyIf{\transparent{.4}{$f_n \blacktriangleleft B_2 \cup B_3$}}{
          $C_1' \gets C_1 \cup \{f_1, \ldots, f_{n-1}\}$\;
          \Return $H' \gets \begin{pmatrix}
            A_1 & C_1' & B_1'\\
            A_2 & C_2 & B_2\\
            A_3 & C_3 & B_3
            \end{pmatrix}$ \;
        }
        \Else {
          $\forall_{i \in [3]} : A_i' \gets$ neighbors of $f_1$ in $A_i$\;
          $\forall_{i \in [3]} : A_i'' \gets A_i \setminus A_i'$\;
          $\forall_{i \in [3]} : B_i'' \gets$ neighbors of $f_n$ in $B_i$\;
          $\forall_{i \in [3]} : B_i' \gets B_i \setminus B_i''$\;
          \Assert{Every $i$-rung is between $A_i'$ and $B_i'$ or $A_i''$ and $B_i''$}
          $\forall_{i \in [3]} : C_i' \gets $ union of interiors of $i$-rings between $A_i'$ and $B_i'$\;
          $\forall_{i \in [3]} : C_i'' \gets $ union of interiors of $i$-rings between $A_i''$ and $B_i''$\;
          \Assert{$C_i = C_i' \cup C_i''$, $C_i' \cap C_i'' = \emptyset$}
          \Assert{$A_i' \cup C_i' \setAntiComplete C_i'' \cup B_i''$, $A_i'' \cup C_i'' \setAntiComplete C_i \cup B_i$}
          \Assert{$A_i' \setComplete A_i''$, $B_i' \setComplete B_i''$}
          \Assert{$A_1', A_2'', A_3', A_3'' \neq \emptyset$}
          $H' \gets \begin{pmatrix}
            A_1' & C_1' & B_1'\\
            A_2' \cup A_3' & C_2' \cup C_3' & B_2' \cup B_3'\\
            \bigcup_i  A_i'' \cup \{f_1\} & \bigcup_i C_i'' \cup \{f_2, \ldots, f_n\} & \bigcup_i B_i''
            \end{pmatrix}$ \;
          \Return $H'$\;
        }
      }
    }
    \Else {
      $a_1 \gets$ neighbor of $f_1$ in $A_1$\;
      $R_1 \gets 1$-rung with end $a_1$\;
      $b_1 \gets$ the other end of $R_1$\;
      
      $b_2 \gets$ neighbor of $f_2$ in $B_2$\;
      $R_2 \gets 2$-rung with end $b_2$\;
      $a_2 \gets$ the other end of $R_2$\;
      \Assert{$b_1 \in X$, $a_2 \in X$}
      \Assert{$(b_1 - f_1 \land a_2 - f_n) \xor (b_1 - f_n \land a_2 - f_1)$}
      \If{$f_1 - b_1$} {
        \Assert{$H$ is odd}
        $R_3 \gets$ any $3$-rung with ends $a_3, b_3$, such that $\{a_3, b_3\} \setAntiComplete \{f_1, f_n\}$\;
        \Return $V(R_1) \cup V(R_2) \cup V(R_3) \cup \{f_1, \ldots, f_n\}$ - a \LGBSK\;
        \TODO{Is it valid input for part of ALG I?}
      }
    }
  }
}

\continueAlg[-10.6cm]{
  \setcounter{AlgoLine}{42}
  \GreyElse{\transparent{.4}{// $\forall_{i\in[3]} X \cap C_i = \emptyset$}}{ 
    \GreyElse{\transparent{.4}{ // $n$ \textup{is odd and} $H$ \textup{is even, or} $n$ \textup{is even and} $H$ \textup{is odd}}}{
      \Else(\tcp*[h]{$f_1 - a_2$}) {
        $\forall_{i \in [3]} : A_i' \gets A_i \cap X$, $A_i'' \gets A_i \setminus X$\;
        $\forall_{i \in [3]} : B_i' \gets B_i \cap X$, $B_i'' \gets B_i \setminus X$\;
        $\forall_{i \in [3]} : C_i' \gets$ union of $i$-rungs between $A_i'$ and $B_i'$\;
        $\forall_{i \in [3]} : C_i'' \gets$ union of $i$-rungs between $A_i''$ and $B_i''$\;
        \Assert{$C_i = C_i' \cup C_i''$, $C_i' \cap C_i'' = \emptyset$}
        \If {$f_1$ is complete to at least two of $A_i$}{
            relabel strips of $H$, so that $f_1$ is complete to $A_1$ and $A_2$\;
            \Assert{$f_n$ is complete to $B_1$ and $B_2$}
            \Assert[SPGT 10.5]{$n > 1$}
            \Return $\begin{pmatrix}
              A_1 & C_1 & B_1\\
              A_2 & C_2 & B_2\\
              A_3 \cup \{f_1\} & C_3 \cup \{f_2, \ldots, f_{n-1}\} & B_3 \cup \{f_n\}
              \end{pmatrix}$ \;
        }
        \Else {
          \Assert{$A_i' \setComplete A_i''$}
          \Assert{$B_i' \setComplete B_i''$}
          \Return $\begin{pmatrix}
            A_1' & C_1' & B_1'\\
            A_2' \cup A_3' & C_2 \cup C_3' & B_2' \cup C_3'\\
            \bigcup_i A_i'' \cup  \{f_1\} & \bigcup_i C_i'' \cup \{f_2, \ldots, f_{n-1}\} & \bigcup_i B_i'' \cup \{f_n\}
            \end{pmatrix}$ \;
        }
      }
    }
  }
}
\clearpage

\graphAppendix{Good partition from an even hyperprism}{alg:goodPartitionEvenH}
\begin{mynameforalgorithm}
	\SetKwFunction{GoodPartitionEvenH}{\textsc{Good-Partition-From-Even-Hyperprism}}
	\Indm\nonl\GoodPartitionEvenH{$G, H, M$}\\
	\KwData{$G$ -- square-free, Berge graph containing no \LGBSK \newline
		$H = (A_1, \ldots, B_3)$  -- maximal even hyperprism in $G$ \newline
		$M$ -- set of major neighbors of $H$}
	\KwResult{A good partition of $G$}
	\Indp	
	$Z \gets \bigcup\{V(C)$ : $C$ is a component of $G \setminus \{V(H) \cup M \}$ with no attachments in $H \}$  \;

	relabel strips of $H$, so that $M \cup A_1$ and $M \cup B_1$ are cliques \;
	
	$F_1 \gets \bigcup\{ V(C)$ : $C$ is a component of $G \setminus \{H \cup M \cup Z \}$ that
	attaches to $A_1 \cup B_1 \cup C_1$  $\}$ \;

	% $F_A \gets \bigcup \{ V(C) $ : $C$ is a component of $G \setminus \{ H \cup M \cup Z \cup F_1 \cup F_2 \cup F_3 \}$
	% that attaches to $A_1 \cup A_2 \cup A_3$  $\}$ \;

	% $F_B \gets \bigcup \{ V(C) $ : $C$ is a component of $G \setminus \{ H \cup M \cup Z \cup F_1 \cup F_2 \cup F_3 \}$
	% that attaches to $B_1 \cup B_2 \cup B_3$  $\}$ \;

	% \Assert{$F_1, F_2, F_3, F_A, F_B$ are pairwise anticomplete}
	% \Assert{$V(G) = H \cup M \cup Z \cup F_1 \cup F_2 \cup F_3 \cup F_a \cup F_B$}
	\Assert{$M$ is a clique}
	\Assert{$M \cup A_i$ is a clique for at least two values of $i$}
	\Assert{$M \cup B_j$ is a clique for at least two values of $j$}

	$K_1 \gets A_1$, $K_2 \gets M$, $K_3 \gets B_1$ \;
	$R \gets C_1 \cup F_1 \cup Z $ \;
	$L \gets G \setminus \{K_1 \cup K_2 \cup K_3 \cup R\}$ \;
	\Return $(K_1, K_2, K_3, L, R)$\;
\end{mynameforalgorithm}
\clearpage

\graphAppendix{Good partition from an odd hyperprism}{alg:goodPartitionOddH}
\begin{algorithm}
	\SetKwFunction{GoodPartitionOddH}{\textsc{Good-Partition-From-Odd-Hyperprism}}
	\Indm\nonl\GoodPartitionEvenH{$G, H, M$}\\
	\KwData{$G$ -- square-free, Berge graph containing no \LGBSK \newline
		$H = (A_1, \ldots, B_3)$  -- maximal odd hyperprism in $G$ \newline
	$M$ -- set of major neighbors of $H$}
	\KwResult{A good partition of $G$}
	\Indp	
			  
	$Z \gets \bigcup\{V(C)$ : $C$ is a component of $G \setminus \{V(H) \cup M \}$ with no attachments in $H \}$  \;
			  
	relabel strips of $H$, so that $A_1 \setAntiComplete B_1$ and $A_2 \setAntiComplete B_2$\;
	\Assert{$C_1 \neq \emptyset, C_2 \neq \emptyset$}
			
	$\forall_{i \in [3]} F_i \gets \bigcup\{ V(C)$ : $C$ is a component of $G \setminus \{H \cup M \cup Z \}$ that
	attaches to $A_i \cup B_i \cup C_i$  $\}$ \;
	% $F_A \gets \bigcup \{ V(C) $ : $C$ is a component of $G \setminus \{ H \cup M \cup Z \cup F_1 \cup F_2 \cup F_3 \}$
	% that attaches to $A_1 \cup A_2 \cup A_3$  $\}$ \;
			
	$F_B \gets \bigcup \{ V(C) $ : $C$ is a component of $G \setminus \{ H \cup M \cup Z \cup F_1 \cup F_2 \cup F_3 \}$
	that attaches to $B_1 \cup B_2 \cup B_3$  $\}$ \;
			  
	\TODO{$F_i, F_A, F_B$ are from algIV, make sure it is correct}
			  
	\Assert{At least two of $A_i$ and at least two of $B_i$ are cliques}
	\Assert{$M$ is complete to at least two of $A_i$ and at least two of $B_i$}
	\Assert{$M$ is a clique}
	\Assert{For at least two $i$ : $A_i \cup M$ is a clique}
	\Assert{For at least two $j$ : $A_j \cup M$ is a clique}
			
	% pick any $x_1 :\in C_1, x_2 :\in C_2, a_3 :\in A_3$, so that $\tau \gets \{x_1, x_2, a_3\}$ is a triad \;
			
	choose $h$, so that $M \cup A_h$ and $M \cup B_h$ are cliques\;
			
	\If(\tcp*[h]{\TODO{make sure $h=2$ is ok}}){$h = 1 \lor h = 2$}{
		$K_1 \gets A_1$, $K_2 \gets M$, $K_3 \gets B_1$\;
		$R \gets C_1 \cup F_1 \cup Z$\;
		$L \gets G \setminus \{K_1 \cup K_2 \cup K_3 \cup R\}$ \;
		\Return $(K_1, K_2, K_3, L, R)$\;
		} \Else {
		relabel $H$ so that $M \cup A_1$ and $M \cup B_2$ are cliques\;
		$K_1 \gets B_2 \cup B_3$, $K_2 \gets M$, $K_3 \gets A_1 \cup A_3$\;
		$L \gets B_1 \cup C_1 \cup F_1 \cup F_B$\;
		$R \gets G \setminus \{K_1 \cup K_2 \cup K_3 \cup L\}$ \;
		\Return $(K_1, K_2, K_3, L, R)$\;
	}
\end{algorithm}
\clearpage

\graphAppendix{Good partition from a $J$-strip system}{alg:goodPartitionJStrip}
\begin{algorithm}
	\SetKwFunction{GoodPartitionJStrip}{\textsc{Good-Partition-From-J-Strip-System}}
	\Indm\nonl\GoodPartitionJStrip{$G, J, (S, N), M$}\\
	\KwData{$G$ -- square-free, Berge graph \newline
		$J$ -- a maximal $3$-connected graph with appearance in $G$ \newline
		$(S, N)$ -- a maximal $J$-strip system \newline
		$M$ -- a set of major vertices w.r.t. $(S, N)$
	}
	\KwResult{A good partition of $G$}
	\Indp	
	$S^*_{uv} \gets S_{uv} \cup$ $($ components of $G\setminus V(S,N)$ that attach in $S_{uv}$ only $)$\;
	$T_{uv} \gets N_u \cap N_v$\;
	\Assert{$T_{uv} \setComplete N_u \setminus N_{uv}$, $T_{uv} \setComplete N_v \setminus N_{vu}$}
	\Assert{$M \cup T_{uv}$ is a clique}
	\If{$\exists S_{uv}$ -- a rich strip in $(S,N)$} {
		\If{$\exists S_{uv}$ -- a rich strip in $(S, N)$, such that
			$M \cup(N_u \setminus N_{uv})$ and $M \cup(N_v \setminus N_{vu})$ are cliques
			}{
			$K_1 \gets N_u \setminus N_{uv}$, $K_2 \gets M \cup T_{uv}$, $K_3 \gets N_v \setminus N_{vu}$\;
			$L \gets (S^*_{uv} \setminus T_{uv}) \cup ($ components of $G\setminus V(S, N)$ that attach only to $N_u$
			and those that attach only to $N_v$ $)$\;
			$R \gets G \setminus (K_1 \cup K_2 \cup K_3 \cup L)$\;
			\Return $(K_1, K_2, K_3, L, R)$\;
		}
		\Else{
			$S_{uv} \gets$ a rich strip in $(S, N)$, such that
			$M \cup(N_u \setminus N_{uv})$ is not a clique and $M \cup(N_v \setminus N_{vu})$ is a clique \;
			$K_1 \gets N_{uv} \setminus T_{uv}$, $K_2 \gets M \cup T_{uv}$, $K_3 \gets N_v \setminus N_{vu}$\;
			$R \gets (S^*_{uv} \setminus N_u) \cup ($ components of $G\setminus V(S,U)$ that attach only to $N_v$ $)$\;
			$L \gets G \setminus (K_1 \cup K_2 \cup K_3 \cup R)$\;
			\Return $(K_1, K_2, K_3, L, R)$\;
		}
	}
	\Else {
		\Assert{$\forall uv \in E(J): S_{uv} = T_{uv}, S_{uv}$ is a clique}
		$S_{uv} \gets $ any strip\;
		$K_1 \gets N_u \setminus S_{uv}$, $K_2 \gets M$, $K_3 \gets N_v \setminus S_{uv}$\;
    $L \gets S^*_{uv} \cup ($ components of $G\setminus V(S,N)$ that attach only to $N_u$ and only to $N_v$ $)$\;
		$R \gets G \setminus (K_1 \cup K_2 \cup K_3 \cup L)$\;
		\Return $(K_1, K_2, K_3, L, R)$\;
	}
\vspace{-1cm}
\end{algorithm}
\clearpage

\graphAppendix{Good partition from a special $K_4$ strip system}{alg:goodPartitionSpecialK4}
\begin{algorithm}
	\SetKwFunction{GoodPartitionSpecialStrip}{\textsc{Good-Partition-From-Special-Strip-System}}
	\Indm\nonl\GoodPartitionJStrip{$G, J, (S, N), M$}\\
	\KwData{$G$ -- square-free, Berge graph \newline
		$(S, N)$ -- a special $K_4$ strip system \newline
		$M$ -- a set of major vertices w.r.t. $(S, N)$
	}
	\KwResult{A good partition of $G$}
	\Indp	
	$\forall_{i,j \in [4]} O_{ij} \gets$ set of vertices in $V(G) \setminus V(S,N)$ that are
	complete to $(N_i \cup N_j) \setminus S_{ij}$ and 
  anticomplete to $V(S, N) \setminus (N_i \cup N_j \cup S_{ij})$\;

	\If{$(N_1 \setminus N_{12}) \cup M \cup O_{12}$ and $(N_2 \setminus N_{12}) \cup M \cup O_{12}$ are cliques}{
		$K_1 \gets N_1 \setminus N_{12}$, $K_2 \gets O_{12} \cup M$, $K_3 \gets N_2 \setminus N_{12}$\;
		$L \gets$ union of those components of $G \setminus (K_1 \cup K_2 \cup K_3)$ that contain
		vertices of $S_{12}$\;
		$R \gets G \setminus (K_1 \cup K_2 \cup K_3 \cup L)$\;
		\Return $(K_1, K_2, K_3, L, R)$\;
	}
	\Else {
		relabel $G$ so that $(N_1 \setminus N_{12}) \cup M \cup O_{12}$ is not a clique\;
		\Assert{$N_{12} \cup M \cup O_{12}$ is a clique}
		\If{$N_{21} \cup M \cup O_{12}$ is a clique}{
			$X \gets N_{21}$ \;
		}
		\Else {
			$X \gets N_2 \setminus N_{21}$\;
		}
		$K_1 \gets N_{12}$, $K_2 \gets M \cup O_{12}$, $K_3 \gets X$\;
		$L \gets$ component of $G \setminus (K_1 \cup K_2 \cup K_3)$ that contains $N_1 \setminus N_{12}$\;
		$R \gets G \setminus (K_1 \cup K_2 \cup K_3 \cup L)$\;
		\Return $(K_1, K_2, K_3, L, R)$\;
	}
\end{algorithm}
\clearpage

\graphAppendix{Find a special $K_4$ system}{alg:findSpecialK4}
\begin{mynameforalgorithm}
  \SetKwFunction{FindSpecialK}{\textsc{Find-Special-K4-Strip-System}}
	\Indm\nonl\FindSpecialK{$G, J, (S, N), m$}\\
  \KwData{$G$ -- square-free, Berge graph \newline
    $J$ -- a $3$-connected graph with appearance in $G$ \newline
    $(S, N)$ -- a $J$-strip system \newline
    $m \in G \setminus (S, N)$ that is major w.r.t. some choice of rungs of $(S, N)$ but not w.r.t. $(S, N)$
  }
  \KwResult{$(S', N')$ -- A special $K_4$-strip system, or \newline
  $(S'', N'')$ -- a bigger $J$-strip system
  }
  \Indp	
  $X \gets N(m)$\;
  $M \gets$ vertices of $G\setminus V(S,N)$ major w.r.t. $(S,N)$\;
  $M^* \gets$ vertices of $G\setminus V(S,N)$ major w.r.t. some choice of rungs\;
  \Assert[SPGT 8.4]{$J = K_4$}
  $V(J) \gets [4]$\;
  $\forall_{i\neq j \in [4]}$: choose rungs $R_{ij}, R_{ij}$, forming line graphs $L(H)$ and $L(H')$ so that $X$ 
  saturates $L(H)$ but does not saturate $L(H')$\;
  \Assert{$R_{ij} \neq R'_{ij}$ if and only if $\{i, j\} = [2]$}
  $r_{ij}, r_{ji} \gets$ ends of each $R_{ij}$\;
  $r'_{ij}, r'_{ji} \gets$ ends of each $R'_{ij}$\;
  $\forall_{i \in [4]} T_i \gets \{r_{ij}, j \in [4] \setminus \{i\} \}$\;
  $\forall_{i \in [4]} T'_i \gets \{r'_{ij}, j \in [4] \setminus \{i\} \}$\;
  \Assert{$X$ has at least two members in each $T_1, \ldots T_4$}
  \Assert{There is $T'_i$ that contains at most one member of $X$}
  \Assert{$T_3 = T'_3, T_4 = T'_4$}
  relabel 1 and 2 in $J$, so that $|X \cap T_1| = 2$ and $|X \cap T'_1| = 1$\;
  \Assert{$r_{12} \in X, r'_{12} \notin X$}
  \Assert{$r_{13} \in X \xor r_{14} \in X$}
  relabel 3 and 4 in $J$, so that $r_{13} \in X, r_{14} \notin X$\;
  \Assert[6.3.(3)]{$R_{34}$ is even and $[X \cap V(L(H'))] \setminus V(R_{34}) = \{r_{31}, r_{32}, r_{41}, r_{42} \}$}
  \Assert[6.3.(4)]{$R_{14}$ has odd length, $r_{21} \in X$}
  \Assert[6.3.(5)]{$R_{12}$ has length $0$, every $12$-rung has even length}
  \Assert[6.3.(6)]{$R_{24}$ has length $0$ and $R_{23}$ has odd length}
  \Assert[6.3.(7)]{Every $34$-rung has non-zero even length}
  $\forall_{i \neq j \in [4]} O_{ij} \gets$ set of vertices that are not major w.r.t $L(H')$ and
  are complete to $(T'_i \cup T'_j) \setminus R'_{ij}$\;
  \Assert{$r_{12} = r_{21} \in O_{12}$, $m \in O_{34}$}
  \Assert{$O_{34} = M^* \setminus M$}
  $(S', N') \gets$ strip system obtained from $(S, N)$ by replacing $S_{12}$ with $S_{12} \setminus O_{12}$\;
  \If{$\exists_{rung R}: $ adding $R$ to $S'_{12}$ produces enlargement of $(S, N)$}{
    \Return $(S'', N'')$ -- an enlargement of $(S, N)$\;
  }
  \Else{
    \Return $(S', N')$ -- a special $K_4$ strip system\;
  }
  \vspace{-10cm}
\end{mynameforalgorithm}
\clearpage

\graphAppendix{Growing a $J$-strip}{alg:growingJStrip}
\noindent\unfinishedAlg[-.45cm]{
	\SetKwFunction{GrowJStrip}{\textsc{Growing-J-Strip}}
	\Indm\nonl\GrowJStrip{$G, J, (S, N)$}\\
	\KwData{$G$ -- square-free, Berge graph \newline
		$J$ -- a $3$-connected graph with appearance in $G$ \newline
		$(S, N)$ -- a $J$-strip system \newline
		\TODO{make sure def of $J$ is correct}
	}
	\KwResult{$J'$ and a maximal $J'$-strip system, or a special strip system}
	\Indp	
	$M \gets$ vertices of $G \setminus V(S, N)$ that are major on some choice of Rungs of $(S, N)$\;
	\TODO{$M$ like ALGI} 
	\If {$\exists m : m$ is not major on some choice of rungs of $(S, N)$}{
		$OUT \gets$ \FindSpecialK{$G, J, (S, N), m$} \;
		\If{$OUT$ is a special strip system}{
			\Return $OUT$ \;
		}
		\Else{
			\Return \GrowJStrip{$G, OUT$} \;
		} 
	}
	\ElseIf{$\exists F$ : $F$ is a component of $G\setminus(V(S,N) \cup M)$, 
		such that no member of $F$ is major w.r.t. $(S, N)$ 
		and set of attachments of $F$ on $H$ is not local}{
		\Assert[6.2, or actually SPGT 8.5]{}
		$F \gets$ minimal (component?) with this property\;
								
		\If{$\exists v \in V(J): X \subset \bigcup(S_{uv} : uv \in E(J))$}{
			$x :\in X \cap S_{uv} \setminus N_v$, for some $uv \in E(J)$\; \TODO{brackets?}
			$x' :\in X \cap S_{u'v}$, for some $u'v \in E(J), u' \neq u$\;
			\Assert{$\{x, x'\}$ is not local w.r.t. $(S,N)$}
			$L(H) \gets \forall_{i,j \in E(J)}$ choose $ij$-rung $R_{ij}$, so that $x \in V(R_{uv}), x' \in V(R_{u'v})$\;
			\Assert{$\{x, x'\}$ is not local w.r.t. $L(H)$}
			$D \gets$ a branch of $H$ with ends $d, u$: $\delta_H(d) \setminus E(D) = (X \cap E(H)) \setminus E(D)$\;
			$P \gets$ a path with ends $p_1, p_2$, so that: \nonl \;
			\pushline $p_1 \blacktriangleleft N_v \setminus N_{vu}$ and
			no other vertex of $P$ has neighbors in $N_v \setminus N_{uv}$ \nonl \;
			$p_2 - x$ and no other vertex of $P$ has neighbors in $S_{uv} \setminus N_v$ \;
			\popline
			$(S', N') \gets $ add $p_1$ to $N_v$ and $F$ to $S_{uv}$ \;
			\Return \GrowJStrip($G, J, (S',N')$)\;
		}
		\Else {
			$K \gets \{uv \in E(J) : X \cap S_{uv} \neq \emptyset \}$\;
			\Assert[SPGT 8.5.(3)]{There are two disjoint edges in $K$}
			$F$ is a vertex set of a path $\gets f_1 - \ldots - f_n$\;
			\Assert{Every choice of rungs is broad}
		}
	}
}

\continueAlg[-9cm]{
  \setcounter{AlgoLine}{18}
	\GreyIf{\transparent{.4}{$\exists F \ldots$}} {
		\GreyElse{\transparent{.4}{ //  $\nexists v \in V(J): X \subset \bigcup(S_{uv} : uv \in E(J))$}}{
			\Assert{every choice of rungs has the same traversal. (Hard to assert)} 
			$ij \gets$ the traversal edge\;
			$A_1 \gets N_i \setminus S_{ij}$, $A_2 \gets N_j \setminus S_{ij}$\;
			\Assert{$X \cap (V(S,N) \setminus S_{ij}) = A_1 \cup A_2$}
			\If{$n = 1$} {
				$(S', N') \gets$ add $f_1$ to $N_i, N_j, S_{ij}$\;
				\Return \GrowJStrip($G, J, (S',N')$)\;
			}
			\Else {
				$x_1 :\in A_1, x_2 :\in A_2$, so that $x_1$ and $x_2$ are in disjoint strips\;
				\Assert{$x_1 - f_1 \xor x_1 - f_n$}
				\If{$x_1 - f_n$}{relabel $f_1 - \ldots - f_n$ front to back}
				$(S', N') \gets$ add $f_1$ to $N_i$, $f_n$ to $N_j$ and $F$ to $S_{ij}$\;
				\Return \GrowJStrip($G, J, (S',N')$)\;
			}
		}
	}
	\Else{
		\Return $J, (S, N)$ -- a maximal $J$-strip\;
	}
}
  

\clearpage