We begin by recalling the basic notions of graph theory. We use standard definitions, sourced from the book by \citeauthor{BB98} \citetitle*{BB98}, modified and extended as needed.

\begin{defn}[graph]
  A \emph{graph} $G$ is an ordered pair of disjoint sets $(V, E)$ such that $E$ is the subset of the set $V \choose 2$, that is, of unordered pairs of $V$.
\end{defn}

\begin{wrapfigure}{r}{0.4\textwidth}
  \centering\begin{tikzpicture}[scale=.7, simplegraph]
    \node(a) at (0, 0) {$v_1$};
    \node(b) at (2, 0) {$v_2$};
    \node(c) at (4, 0) {$v_3$};
    \node(d) at (6, 0) {$v_4$};

    \draw(a) to (b);
    \draw(b) to (c);
    \draw(c) to (d);

    \draw[dashed](a) to[in=90, out=90] (d);
    \draw[dashed](a) to[in=90, out=90] (c);
    \draw[dashed](b) to[in=90, out=90] (d);
  \end{tikzpicture}
  \caption{An example graph $G_0$}
  \label{fig:examplePath}
\end{wrapfigure}

We will only consider finite graphs, that is, $V$ and $E$ are always finite. If $G$ is a graph, then $V = V(G)$ is the \emph{vertex set} of $G$, and $E = E(G)$ is the \emph{edge set} of $G$. When the context of $G$ is clear we will use $V$ and $E$ to denote its vertex and edge set.

An edge $\{x, y\}$ is said to \emph{join}, or \emph{be between} vertices $x$ and $y$ and is denoted by $xy$. Thus $xy$ and $yx$ denote the same edge (all our graphs are \emph{undirected}). If $xy \in E(G)$ then $x$ and $y$ are \emph{adjacent}, \emph{connected} or \emph{neighboring}. By $N(x)$ we will denote the \emph{neighborhood} of $x$, that is, all vertices $y$ such that $xy$ is an edge. Similarly, for $X \subseteq V(G)$, by $N(X)$ we will denote the \emph{neighborhood} of $X$, meaning all vertices of $v \in V(G) \setminus X$, so that there is a $x \in X$, that $xv$ is an edge in $G$. If $xy \notin E(G)$ then $xy$ is a \emph{nonedge} and $x$ and $y$ are \emph{nonneighbors} or \emph{anticonnected}.

\Cref{fig:examplePath} shows an example graph $G_0 = (V, E)$ with $V = \{v_1, v_2, v_3, v_4\}$ and $E = \{v_1v_2, v_2v_3, v_3v_4\}$. We will mark edges as solid lines on figures. Nonedges significant to the ongoing reasoning will be marked as dashed lines.

\begin{defn}[subgraph]
  A graph $G' = (V', E')$ is a \emph{subgraph} of $G = (V, E)$ if and only if $V' \subseteq V$ and $E' \subseteq E$.
\end{defn}

\begin{defn}[induced subgraph]
  If $G' = (V', E')$ is a subgraph of $G$ and it contains \emph{all edges} of $G$ that join two vertices in $V'$, then $G'$ is said to be an \emph{induced subgraph} of $G$ and is denoted $G[V']$.
\end{defn}

Given a graph $G = (V, E)$ and a set $X \subseteq V$ by $G\setminus X$ we will denote the induced subgraph $G[V\setminus X]$.

For example $(\{v_1, v_2, v_3\}, \{v_1v_2\})$ is \emph{not} an induced subgraph of the example graph $G_0$, while $(\{v_1, v_2, v_3\}, \{v_1v_2, v_2v_3\}) = G_0[\{v_1, v_2, v_3\}] = G_0 \setminus \{v_0\}$ is.

\begin{defn}[$X$-completeness]
  Given a graph $G = (V, E)$ and a set $X \subseteq V$, vertex $v \in V(G) \setminus X$ is \emph{$X$-complete} if and only if it is adjacent to every vertex $x \in X$. A set $Y \subseteq V$ is $X$-complete if and only if $X \cap Y = \emptyset$ and every vertex $y \in Y$ is $X$-complete.
\end{defn}

For example, for $X = \{v_2\}$, the set $\{v_1, v_3\}$ is $X$-complete in $G$, while the set $\{v_3, v_4\}$ is not.

\begin{defn}[path]
  A \emph{path} is a graph $P$ of the form
  \[ V(P) = \{x_1, x_2, \ldots, x_l\},\quad E(P) = \{x_1x_2, x_2x_3, \ldots, x_{l-1}x_l\} \]
\end{defn}
This path $P$ is usually denoted by $x_1x_2\ldots x_l$ or $x_1$-$x_2$-$\ldots$-$x_l$. The vertices $x_1$ and $x_l$ are the \emph{endvertices} and ${l-1} = |E(P)|$ is the \emph{length} of the path P. $\{x_2, \ldots x_{l-1}\}$ is the \emph{inside} of the path $P$, denoted as $P^*$. Notice that we don't allow any edges other than the ones between consecutive vertices for a graph to be called a path.

Our graph $G_0$ is a path of length 3, with the inside $G_0^* = \{v_2, v_3\}$. If we added any edge to the $G_0$ it would stop being a path.


\begin{defn}[connected graph, subset]
  A graph $G$ is \emph{connected} if and only if for every pair $\{x, y\} \subseteq V(G)$ of distinct vertices, there is a path from $x$ to $y$.
  A subset $X \subseteq V(G)$ is connected if and only if the graph $G[X]$ is connected.
\end{defn}

\begin{defn}[component]
  A \emph{component} of a graph $G$ is its maximal connected induced subgraph.
\end{defn}


\begin{defn}[cycle]
  A \emph{cycle} is a graph $C$ of the form
  \[ V(C) = \{x_1, x_2, \ldots, x_l\},\quad E(C) = \{x_1x_2, x_2x_3, \ldots, x_{l-1}x_l, x_lx_1\} \]
\end{defn}

This cycle $C$ is usually denoted by $x_1x_2\ldots x_lx_1$ or $x_1$-$x_2$-$\ldots$-$x_l$-$x_1$. $l = |E(C)|$ is the \emph{length} of the cycle $C$. Sometimes we will denote the cycle of length $l$ as $C_l$.


\begin{defn}[hole]
  A \emph{hole} is a cycle of length at least four.
\end{defn}

If a path, a cycle or a hole has an odd length, it will be called \emph{odd}. Otherwise, it will be called \emph{even}. Notice that if we add an edge $v_1v_4$ to the path $G_0$ it becomes an even cycle $C_4$.
\begin{defn}[complement]
  A \emph{complement} of a graph $G = (V, E)$ is a graph $\overline{G} = (V, {V \choose 2} \setminus E)$, that is, two vertices $x, y$ are adjacent in $\overline{G}$ if and only if they are not adjacent in $G$.
\end{defn}

% Sometimes, we will call a complement of a member of a class $\Gamma$ an \emph{anti-
% $\Gamma$}, e.g. graph $G = (\{v_1, v_2, v_3, v_4\}, \{v_1v_3, v_2v_4\})$ is an anticycle.

\begin{defn}[anticonnected graph, subset]
  A graph $G$ is \emph{anticonnected} if and only if $\overline{G}$ is connected.
  A subset $X$ is \emph{anticonnected} if and only if $\overline{G}[X]$ is connected.
\end{defn}

\begin{defn}[anticomponent]
  An \emph{anticomponent} of a graph $G$ is an induced subgraph whose complement is a component in $\overline{G}$.
\end{defn}

\begin{defn}[antipath]
  An \emph{antipath} is a graph $G$ such that $\overline{G}$ is a path.
\end{defn}

\begin{defn}[antihole]
  An \emph{antihole} is a graph $G$ such that $\overline{G}$ is a hole.
\end{defn}

\begin{defn}[clique]
  A \emph{complete graph} or a \emph{clique} is a graph of the form $G = (V, {V \choose 2})$, that is, every two vertices are connected.
\end{defn}

We will denote the clique on $n$ vertices as $K_n$.

\begin{defn}[clique number]
  The \emph{clique number} of a graph $G$, denoted as $\omega(G)$, is a cardinality of its largest induced clique.
\end{defn}

\begin{defn}[anticlique]
  An \emph{independent set} or an \emph{anticlique} is a graph of the form $G = (V, \emptyset)$, that is no vertices are connected.
\end{defn}
In a similar fashion, given a graph $G = (V, E)$, a subset of its vertices $V' \subseteq V$ will be called \emph{independent} (in the context of $G$) if and only if $G[V']$ is an anticlique.

\begin{defn}[stability number]
  The \emph{stability number} of a graph $G$, denoted as $\alpha(G)$, is a cardinality of its largest induced stable set.
\end{defn}

\begin{defn}[coloring]
  Given a graph $G$, its \emph{coloring} is a function $c: V(G) \rightarrow \mathbb{N}^+$, such that for every edge $xy \in E(G)$, $c(x) \neq c(y)$ . A $k$-coloring of $G$ (if exists) is a coloring, such that $c(x) \leq k$ for all vertices $x \in V(G)$.
\end{defn}

\begin{defn}[chromatic number]
  The \emph{chromatic number} of a graph $G$, denoted as $\chi(G)$, is a smallest natural number $k$, for which there exists a $k$-coloring of $G$.
\end{defn}

\begin{defn}[line graph]
  The \emph{line graph} of a graph $G = (V, E)$ is the graph $L(G)$ with $V(L(G)) = E$ and $E(L(G)) = \{e_1 e_2: e_1, e_2 \in E, e_1 \cap e_2 \neq \emptyset\}$, that is, $e_1, e_2 \in E$ are adjacent if and only if they share an endpoint in $G$.
\end{defn}


