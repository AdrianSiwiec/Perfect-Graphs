We use standard definitions, sourced from the book by \citeauthor{BB98} \citetitle*{BB98} and extended as needed.

\begin{defn}[graph]
  A \emph{graph} $G$ is an ordered pair of disjoint sets $(V, E)$ such that $E$ is the subset of the set $V^2$ that is of unordered pairs of $V$.
\end{defn}

\begin{wrapfigure}{r}{0.4\textwidth}
  \label{fig:examplePath}
  \centering\begin{tikzpicture}[scale=.7, simplegraph]
    \node(a) at (0, 0) {$v_1$};
    \node(b) at (2, 0) {$v_2$};
    \node(c) at (4, 0) {$v_3$};
    \node(d) at (6, 0) {$v_4$};

    \draw(a) to (b);
    \draw(b) to (c);
    \draw(c) to (d);

    \draw[dashed](a) to[in=90, out=90] (d);
    \draw[dashed](a) to[in=90, out=90] (c);
    \draw[dashed](b) to[in=90, out=90] (d);
  \end{tikzpicture}
  \caption{An example graph $G_0$}
\end{wrapfigure}

We will only consider finite graphs, that is $V$ and $E$ are always finite. If $G$ is a graph, then $V = V(G)$ is the \emph{vertex set} of $G$, and $E = E(G)$ is the \emph{edge set}. An edge $\{x, y\}$ is said to \emph{join}, or be between vertices $x$ and $y$ and is denoted by $xy$. Thus $xy$ and $yx$ mean the same edge (all our graphs are \emph{unordered}). If $xy \in E(G)$ then $x$ and $y$ are adjacent, connected or neighboring. If $xy \notin E(G)$ then $xy$ is a \emph{nonedge} and $x$ and $y$ are \emph{anticonnected}.

Figure \ref{fig:examplePath} shows an example of a graph $G_0 = (V, E)$ with $V = \{v_1, v_2, v_3, v_4\}$ and $E = \{v_1v_2, v_2v_3, v_3v_4\}$. We will mark edges as solid lines on figures. Nonedges significant to the ongoing reasoning will be marked as dashed lines.

\begin{defn}[subgraph]
  $G' = (V', E')$ is a \emph{subgraph} of $G = (V, E)$ if $V' \subset V$ and $E' \subset E$.
\end{defn}

\begin{defn}[induced subgraph]
  If $G' = (V', E')$ is a subgraph of $G$ and it contains \emph{all edges} of $G$ that join two vertices in $V'$, then $G'$ is said to be \emph{induced subgraph} of $G$ and is denoted $G[V']$.
\end{defn}
For example $(\{v_1, v_2, v_3\}, \{v_1v_2\})$ is \emph{not} an induced subgraph of the example graph $G_0$, while $(\{v_1, v_2, v_3\}, \{v_1v_2, v_2v_3\}) = G_0[\{v_1, v_2, v_3\}]$ is.

\begin{defn}[path]
  A \emph{path} is a graph $P$ of the form
  \[ V(P) = \{x_1, x_2, \ldots, x_l\},\quad E(P) = \{x_1x_2, x_2x_3, \ldots, x_{l-1}x_l\} \]
\end{defn}
This path $P$ is usually denoted by $x_1x_2\ldots x_l$ or $x_1 - x_2 - \ldots - x_l$. The vertices $x_1$ and $x_l$ are the \emph{endvertices} and ${l-1} = |E(P)|$ is the \emph{length} of the path P. $\{x_1, \ldots x_{l-1}\}$ is the \emph{inside} of the path $P$, denoted as $P^*$.

Graph $G_0$ is a path of length 3, with the inside $G_0^* = \{v_2, v_3\}$. If we would add any edge to $G_0$ it would stop being a path (sometimes we call such an edge a \emph{chord}).

\begin{defn}[cycle]
  A \emph{cycle} is a graph $C$ of the form
  \[ V(C) = \{x_1, x_2, \ldots, x_l\},\quad E(C) = \{x_1x_2, x_2x_3, \ldots, x_{l-1}x_l, x_lx_1\} \]
\end{defn}

This cycle $C$ is usually denoted by $x_1x_2\ldots x_lx_1$ or $x_1 - x_2 - \ldots - x_l - x_1$. $l = |E(C)|$ is the \emph{length} of the cycle $C$.

If a path or a cycle has an odd length, it will be called \emph{odd}. Otherwise, it will be called \emph{even}.

Notice, that a cycle is not a path (nor is a path a cycle). If we add an edge $v_1v_4$ to the path $G_0$ it becomes an even cycle.