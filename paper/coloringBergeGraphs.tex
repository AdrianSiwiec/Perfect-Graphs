A natural problem for perfect graphs is a problem of coloring them. In 1988 Gr\"otschel et al. published an ellipsoid-method-based polynomial algorithm for coloring perfect graphs \cite{Grtschel1993}. We consider it in \Cref{sec:coloringEllipsoid}. However due to its use of the ellipsoid method, this algorithm has been usually considered unpractical \cite{coloringSquareFree,Chudnovsky2003, coloringArtemis}.

There has been much progress on the quest of finding an algorithm coloring perfect graphs, without the use of the ellipsoid method (see \Cref{sec:classicalColoring}), however there is still no known polynomial combinatorial algorithm to do this.

\section{Information theory background}
\label{sec:InformationTheory}

\Cref{sec:InformationTheory} and \Cref{sec:computingTheta} are based on lecture notes by Lovász \cite{Lovasz95}.

The only polynomial technique of coloring perfect graphs known so far arose in the field of semidefinite programming. Semidefinite programs are linear programs over the cones of semidefinite matrices. The connection of coloring graphs and the cones of semidefinite matrices might be surprising, so let us take a brief digression into the field of information theory, where we will see the connection more clearly. Also, this was exactly the background which motivated Berge to introduce perfect graphs \cite{Chudnovsky2003}.

\subsection{Shannon capacity of a graph}
\label{sec:ShannonCapacity}

Suppose we have a noisy communication channel in which certain signal values can be confused with others. For instance, suppose our channel has five discrete signal values, represented as 0, 1, 2, 3, 4. However, each value of $i$ when sent across the channel can be confused with value $(i \pm 1)$ mod $5$. This situation can be modeled by a graph $C_5$ (\cref{fig:c5}) in which vertices correspond to signal values and two vertices are connected if and only if the values they represent can be confused.

\begin{wrapfigure}{r}{0.35\textwidth}
  \centering\begin{tikzpicture}[scale=.7,simplegraph]
  \def\ngon{5}
  \node[regular polygon,regular polygon sides=\ngon,minimum size=3cm, draw=none] (p) {};
  \foreach\x in {1,...,\ngon}{\node[] (p\x) at (p.corner \x){$\the\numexpr\intcalcMod{\x}{5}$};}

  \draw (p1) to (p2);
  \draw (p2) to (p3);
  \draw (p3) to (p4);
  \draw (p4) to (p5);
  \draw (p5) to (p1);

\end{tikzpicture}%
  \caption{An example of a noisy channel}%
  \label{fig:c5}
  % \vspace{-0.5cm}
\end{wrapfigure}

We are interested in transmission without possibility of confusion. For this example it is possible for two values to be transmitted without ambiguity e.g. values 1 and 4, which allows us to send $2^n$ non-confoundable messages in $n$ steps. But we could do better, for example we could communicate five two-step codewords e.g. "00", "12", "24", "43", "31". Each pair of these codewords includes at least one position where its values differ by two or more modulo 5, which allows the recipient to distinguish them without confusion.  This allows us to send $5^{n / 2}$ non-confoundable messages in $n$ steps.

Let us be more precise. Given a graph $G$ modeling a communication channel and a number $k \geq 1$ we say that two messages $v_1v_2\ldots v_k$, $w_1w_2\ldots w_k \in V(G)^k$ of length $k$ are non-confoundable if and only if there is $1 \leq i \leq k$ such that $v_i$, $w_i$ are non-confoundable. We are interested in the maximum rate at which we can reliably transmit information, that is, the \emph{Shannon capacity} of the channel defined by $G$.

For $k = 1$, maximum number of messages we can send without confusion in a single step is equal to $\alpha(G)$. To describe longer messages we use \emph{strong product} $G \cdot H$ of two graphs $G = (V, E)$, $H = (W, F)$ as the graph with $V(G \cdot H) = V \times W$, with $(i, u)(j, v) \in E(G \cdot H)$ if and only if $ij \in E$ and $uv \in F$, or $ij \in E$ and $u = v$, or $i = j$ and $uv \in F$. Given channel modeled by $G$ it is easy to see that the maximum number of distinguishable words of length 2 is equal to $\alpha(G \cdot G)$, and in general the number of distinguishable words of length $k$ is equal to $\alpha(G^k)$ which gives us $\sqrt[k]{\alpha(G^k)}$ as the number of distinguishable signals per single transmission. So, we can define the Shannon capacity of the channel defined by $G$ as $\Theta(G) = \sup\limits_k \sqrt[k]{\alpha(G^k)}$.

Unfortunately, it is not known whether $\Theta(G)$ can be computed for all graphs in finite time. If we could calculate $\alpha(G^k)$ for a first few values of $k$ (we will show how to do it in \Cref{alg:maxStableSet}) we could have a lower bound on $\Theta(G)$. Let us now turn into search for some usable upper bound.

\subsection{Lovász number}

For a channel defined by a graph $C_5$, using five messages of length 2 to communicate gives us a lower bound on $\Theta(G)$ equal $\sqrt{5}$ (as does calculating $\sqrt{\alpha(C_5^2)}$).

Consider an "umbrella" in $\mathbb{R}^3$ with the unit vector $e_1 = (1, 0, 0)$ as its "handle" and 5 "ribs" of unit length.  Open it up to the point where non-consecutive ribs are orthogonal, i.e., form an angle of 90$^\circ$. This way we get a representation of $C_5$ by 5 unit vectors $u_1, \ldots, u_5$ so that each $u_i$ forms the same angle with $e_1$ and any two non-adjacent vertices are represented with orthogonal vectors. See \Cref{fig:umbrella}. We can calculate $e_1^\intercal u_i = 5 ^ {-1/4}$.


\begin{wrapfigure}{r}{0.4\textwidth}
  \centering\begin{tikzpicture}[scale=0.7,simplegraph]
  \node(z) at (6.07,- 6.07) {};
  % \node(zi)[draw=none] at (10.94, -6.07){};
  % \node(zii)[draw=none] at (10.94, -5.07){};
  \node(p1) at (9.23, -3.22) {$c_1$};
  \node(p2) at (11.37, -3.44) {$c_2$};
  \node(p3) at (11.72, -7.29) {$c_3$};
  \node(p4) at (9.91, -9.99) {$c_4$};
  \node(p5) at (8.57, -7.19) {$c_5$};

  \draw[dashed] (p1) to (p2);
  \draw[dashed] (p2) to (p3);
  \draw[dashed] (p3) to (p4);
  \draw[dashed] (p4) to (p5);
  \draw[dashed] (p5) to (p1);

  \draw[-{Triangle[width=7pt,length=6pt]}, line width=2pt] (z) to[edge node={node [draw=none,above] {$u_1$}}] (p1);
  \draw[-{Triangle[width=7pt,length=6pt]}, line width=2pt] (z) to[edge node={node [pos=0.7, draw=none,below] {$u_2$}}] (p2);
  \draw[-{Triangle[width=7pt,length=6pt]}, line width=2pt] (z) to[edge node={node [pos=0.7, draw=none, below right] {$u_3$}}] (p3);
  \draw[-{Triangle[width=7pt,length=6pt]}, line width=2pt] (z) to[edge node={node [draw=none,below] {$u_4$}}] (p4);
  \draw[-{Triangle[width=7pt,length=6pt]}, line width=2pt] (z) to[edge node={node [pos=0.7,draw=none,below] {$u_5$}}] (p5);
  \draw[dotted, line width = 1.5pt] (z) to (10.64, -6.07);
  \draw[dotted, line width = 1.5pt] (10.64, -5.27) to[out=0, in=0, looseness=1.9] (10.64, -6.07);
\end{tikzpicture}%
  \caption{Lovász's umbrella}
  \label{fig:umbrella}
\end{wrapfigure}

It turns out, that we can obtain a similar representation of the vertices of $C_5^k$ by unit vectors $v_i \in \mathbb{R}^{3k}$, so that any two non-adjacent vertices are labeled with orthogonal vectors (this representation is sometimes called the \emph{orthogonal representation} \cite{Lovsz1989Orthogonal}). Moreover, we still get $e_1^\intercal v_i = 5^{-k/4}$ for every $i \in V(C_5^k)$ (the proof is quite technical and we omit it here).

If $S$ is any stable set in $C_5^k$, then $\{v_i: i \in S\}$ is a set of mutually orthogonal unit vectors, so we could extend $S$ to $S'$, so that $\{v_i: i \in S'\}$ is a basis of $\mathbb{R}^{3k}$. This gives us
$$\sum\limits_{i\in S}(e_1^\intercal v_i)^2 \leq \sum\limits_{i \in S'}(e_1^\intercal v_i)^2 \leq |e_1|^2 = 1$$
Each term on the left hand side is equal to $5^{-k/4}$, so the left hand side is equal to $|S|5^{-k/2}$, and so $|S| \leq 5^{k/2}$. Since $|S|$ was an arbitrary stable set, we get $\alpha(C_5^k) \leq 5 ^{k/2}$ and $\Theta(C_5) = \sqrt{5}$.

It turns out that this method extends to any graph $G$ in place of $C_5$. All we have to do is to find a orthogonal representation that will give us the best bound. So, we can define the \emph{Lovász number} of a graph $G$ as:
$$\vartheta(G) = \min\limits_{c,U} \max\limits_{i\in V} \frac{1}{(c^\intercal u_i)^2},$$

\noindent where $c$ is a unit vector in $\mathbb{R}^{|V(G)|}$ and $U$ is an orthogonal representation of $G$.

Contrary to Lovász's first hope \cite{Lovasz1979} $\vartheta(G)$ does not always equal $\Theta(G)$, but it is only an upper bound on it. However, these two are equal for some graphs, including all perfect graphs, as is demonstrated in the Lovász "Sandwich theorem".

\begin{theorem}[Lovász "Sandwich theorem" \cite{Knuth1994}]
  \label{thm:sandwich}
  For any graph $G$:
  $$ \omega(G) \leq \vartheta(\overline{G}) \leq \chi(G).$$
\end{theorem}

Because in perfect graphs $\omega(G) = \chi(G)$, we get $\omega(G) = \vartheta(\overline{G}) = \chi(G)$. Therefore, if for any perfect graph $G$, we could calculate $\vartheta(G)$ and $\vartheta(\overline{G})$, we would get $\omega(G)$, $\chi(G)$ and $\alpha(G)$.

But how can we construct an optimum (or even sufficiently good) orthogonal representation? It turns out that it can be computed in polynomial time using semidefinite optimization.

\section{Computing \boldmath$\vartheta$}
\label{sec:computingTheta}

First, let us recall some definitions, with \cite{gilbertstrang2020} as a reference for the linear algebra.

\begin{defn}[eigenvector, eigenvalue]
  Let $A$ be an $n \times n$ real matrix. An \emph{eigenvector} of $A$ is a vector $x$ such that $Ax$ is parallel to $x$. In other words, there is a real or a complex number $\lambda$, such that $Ax = \lambda x$. This $\lambda$ is called the \emph{eigenvalue} of $A$ belonging to eigenvector $x$.
\end{defn}

If a matrix $A$ is symmetric\footnote{Matrix $A$ is symmetric if and only if $A = A^\intercal$}, all the eigenvalues are real.

\begin{defn}[positive semidefinite matrix]
  Let $A$ be an $n \times n$ symmetric matrix. $A$ is \emph{positive semidefinite} if and only if all of its eigenvalues are nonnegative. We denote it by $A \succeq 0$.
\end{defn}

We have equivalent definitions of semidefinite matrices.
\begin{theorem}[Proposition 2.2 of \cite{Lovasz95}]
  For a real symmetric $n \times n$ matrix $A$, the following are equivalent:
  \begin{enumerate}[(i)]
    \item $A$ is positive semidefinite,
    \item \label{en:ei2} for every $x \in \mathbb{R}^n$, $x^\intercal Ax$ is nonnegative,
    \item for some matrix $U$, $A = U^\intercal U$,
    \item $A$ is a nonnegative linear combination of matrices of the type $xx^\intercal$.
  \end{enumerate}
\end{theorem}

From (\ref{en:ei2}) it follows that diagonal entries of any positive semidefinite matrix are nonnegative and that the sum of two positive semidefinite matrices is positive semidefinite.

We may think equivalently of $n \times n$ matrices as vectors with $n^2$ coordinates.
\begin{defn}[convex cone]
  A subset $C$ of $\mathbb{R}^n$ is a \emph{convex cone}, if and only if for any positive scalars $\alpha, \beta$ and for any $x, y \in C$, $\alpha x + \beta y \in C$.
\end{defn}

The fact that the sum of two positive semidefinite matrices is again positive semidefinite, with the fact that every positive scalar multiple of a positive semidefinite matrix is positive semidefinite, translates into the geometric statement that the set of all positive semidefinite matrices forms a convex closed cone $\mathcal{P}_n$ in $\mathbb{R}^{n \times n}$ with its apex at 0. This cone $\mathcal{P}_n$ is important but its structure is not trivial.

\paragraph{Semidefinite programs.}

\begin{defn}[semidefinite program]
  Let $A_1, \ldots, A_n, B$ be given symmetric $m \times m$ matrices and be $c \in \mathbb{R}^n$ a given vector. A \emph{semidefinite program} is an optimization problem of the following form:
  \begin{equation*}
    \begin{array}{ll@{}ll}
      \text{minimize}   & c^\intercal x                          & \\
      \text{subject to} & x_1A_1 + \ldots + x_nA_n - B \succeq 0   \\
    \end{array}
  \end{equation*}
  Any choice of the values $x_i$ that satisfies the given constraint is called a \emph{feasible solution}.
\end{defn}


The special case when $A_1, \ldots A_n, B$ are diagonal matrices is a ''generic'' linear program, in fact we can think of semidefinite programs as generalizations of linear programs. Not all properties of linear programs are carried over to semidefinite programs, but the intuition is helpful.

Solving semidefinite programs is a complex topic, we refer to \cite{grotschel1993} for details. All we need to know is that we can solve semidefinite programs up to an arbitrarily small error in polynomial time. One of the methods to do this is called the \emph{ellipsoid method}, hence the name for the coloring algorithm.

\paragraph{Calculating \boldmath$\vartheta$.}

Let us recall, that an orthogonal representation of a graph $G = (V, E)$ is a labeling $u: V \rightarrow \mathbb{R}^d$ for some $d$, such that $u_i^\intercal u_j = 0$ for all anticonnected vertices $i, j$. The \emph{angle} of an orthogonal representation is the smallest half-angle of a rotational cone containing the representing vectors.

\begin{theorem}[Proposition 5.1 of \cite{Lovasz95}]
  The minimum angle $\phi$ of any orthogonal representation of $G$ is given by $\cos^2\phi = 1/\vartheta(G)$.
\end{theorem}

This leads us to definition of $\vartheta(G)$ in terms of semidefinite programming.

\begin{theorem}[Proposition 5.3 of \cite{Lovasz95}]
  $\vartheta(G)$ is the optimum of the following semidefinite program:

  \begin{equation*}
    \begin{array}{l@{}rlll}
      \text{minimize~~}   & t      &                                              \\
      \text{subject to~~} & Y      & \succeq & 0                                    \\
                        & Y_{ij} & = & -1      & (\forall~ij \in E(\overline{G})) \\
                        & Y_{ii} & = & t - 1                                      \\
    \end{array}
  \end{equation*}

  It is also the optimum of the dual program

  \begin{equation*}
    \begin{array}{l@{}rlll}
      \text{maximize~~}   & \sum_{i \in V} \sum_{j \in V}  Z_{ij} &                                     \\
      \text{subject to~~} & Z                                     & \succeq & 0                         \\
                          & Z_{ij}                                & =       & 0 & (\forall~ij \in E(G)) \\
                          & tr(Z)                                 & =       & 1                         \\
    \end{array}
  \end{equation*}
\end{theorem}

Any stable set $S$ of $G$ provides a feasible solution of the dual program, by choosing $Z_{ij} = \frac{1}{S}$ if $i, j \in S$ and 0 otherwise. Similarly, any $k$-coloring of $\overline{G}$ provides a feasible solution of the former semidefinite program, by choosing $Y_{ij} = -1$ if $i$ and $j$ have different colors, $Y_{ii} = k-1$, and $Y_{ij} = 0$ otherwise.

Now, that we know how to calculate $\vartheta(G)$, let us describe the algorithm to calculate the coloring of $G$.

\section{Coloring perfect graphs using an ellipsoid method}
\label{sec:coloringEllipsoid}
The following part is based on \citetitle{Laurent2005} by \citeauthor{Laurent2005} \cite{Laurent2005}.

\subsection{Maximum cardinality stable set}

Given graph $G$, recall that the stability number of $G$ is equal the clique number of the complement of $G$. This gives us a way to compute $\alpha(G)$ for any perfect graph $G$.

In fact, to calculate $\chi(\overline{G})$ and $\alpha(G)$ we only need an approximated value of $\vartheta(G)$ with precision smaller than $1/2$, as the former values are always integral.

We will now show how to find a stable set in $G$ of size $\alpha(G)$. We will construct a sequence of induced subgraphs $G = G_0 \supseteq G_1 \supseteq \ldots \supseteq G_{|V|}$, so that $G_{|V|}$ is a required stable set.

\begin{alg}[Maximum cardinality stable set in a perfect graph]
  \label{alg:maxStableSet}
  Input: A perfect graph $G = (V, E)$.

  \noindent Output: A maximum cardinality stable set in $G$.
\end{alg}

\begin{algorithmic}[1]
  \mProcedure{Get-Max-Stable-Set}{$G$}
  \ls $v_1, \ldots, v_n \gets$ vertices of $G$
  \ls $G_0 \leftarrow G$
  \ls $\alpha_G = \alpha(G)$
  \mForEach{$1 \leq i \leq |V|$}
  \mIf{$\alpha(G_{i-1} \setminus v_i) = \alpha_G$}
  \ls $G_i \leftarrow G_{i-1} \setminus \{v_i\}$
  \mElse
  \ls $G_i \leftarrow G_{i-1}$
  \mEndIf
  \mEndFor
  \ls \RETURN $G_{|V|}$
  \mEndProcedure
\end{algorithmic}

\begin{theorem}
  \Cref{alg:maxStableSet} works in polynomial time and for an input of $G$ returns a maximal cardinality stable set of $G$.
\end{theorem}
\begin{proof}
  Let us prove that $G_{|V|}$ is indeed a stable set. Suppose otherwise and let $v_iv_j$ be an edge in $G_{|V|}$ with $i < j$ and $i$ minimal. But then $\alpha(G_{i-1} \setminus v_i) = \alpha(G_{i-1}) = \alpha(G)$ so by our construction $v_i$ is not in $G_i$ and $v_iv_j$ is not an edge of $G_n$. Therefore there are no edges in $G_{|V|}$.

  Because at every step we have $\alpha(G_i) = \alpha(G_{i-1})$, therefore $\alpha(G_{|V|}) = \alpha(G)$, so $G_{|V|}$ is required maximum cardinality stable set.

  The running time of \Cref{alg:maxStableSet} is polynomial, because we construct $|V|$ auxiliary graphs, each requiring calculating $\alpha$ once with additional $O(|V|^2)$ time for constructing the graph.
\end{proof}

Given a weight function $w : V \rightarrow \mathbb{N}$ we could calculate the maximum weighted stable set in $G$ using the following modification:

\begin{alg}[Maximum weighted stable set in a perfect graph]
  \label{alg:maxWeightedStableSet}
  Input: A perfect graph $G = (V, E)$ and a weight function $w : V \rightarrow \mathbb{N}$.

  \noindent Output: A maximum weighted stable set in $G$.
\end{alg}

\begin{algorithmic}[1]
  \mProcedure{Get-Max-Weighted-Stable-Set}{$G, w$}
  \mForEach{$v \in V(G)$}
  \ls $W_v \gets$ a set of nonadjacent vertices of size $w(v)$
  \mEndFor
  \ls $V' \gets \bigcup W_v$, for $v \in V(G)$ \Comment{$|V'| = \sum_{v\in V(G)} w(v)$ \vspace{-.1cm}}
  \ls $E' \gets \{xy$ for $x, y \in V'$, such that for some $u, v \in V, u \neq v$:
  \lsx~~~~~~~~ $x \in W_u$, $y \in W_v$, and $uv \in E(G)\}$
  \ls $G' \gets (V', E')$
  \ls $S' \gets$ \textsc{Get-Max-Stable-Set$(G')$}
  \ls $S \gets \{v \in V(G)$, such that for some $x\in W_v$, $x \in S'\}$
  \ls \RETURN $S$
  \mEndProcedure
\end{algorithmic}

We create graph $G'$ by replacing every vertex $v$ by a set $W_v$ of $w(v)$ nonadjacent vertices, making two vertices $x \in W_v$, $y \in W_u$ adjacent in $G'$ if and only if the vertices $v$, $u$ are adjacent in $G$. Then we can calculate a maximum cardinality stable set in $G'$ (we remark that $G'$ is still perfect because every new introduced hole is even) and return a result of those vertices in $G$ whose any (and therefore all) copies were chosen.

\subsection{Stable set intersecting all maximum cardinality cliques}
Next, let us show how to find a stable set intersecting all the maximum cardinality cliques of $G$. We will create a list $Q_1, \ldots Q_t$ of all maximum cardinality cliques of $G$ and return a stable set intersecting them all.

Recall that to calculate a maximum cardinality clique of $G$, we can calculate a maximum cardinality stable set of $\overline{G}$.
\begin{alg}[Stable set intersecting all maximum cardinality cliques]
  \label{alg:ssIntersectingCliques}
  Input: A perfect graph $G = (V, E)$.

  \noindent Output: A stable set which intersects all the maximum cardinality cliques of $G$.
\end{alg}

\begin{algorithmic}[1]
  \mProcedure{Get-Stable-Set-Intersecting-Max-Cliques}{$G$}
  \ls $\omega_G \gets \omega(G)$
  \ls $Q_1 \gets$ \textsc{Max-Stable-Set}$(\overline{G})$
  \ls $t \gets 1$
  \mWhile{true}
  \ls $w \gets$ a weight function $V \rightarrow \mathbb{N}$, so that for $v \in V$, $w(v)$ is equal to
  \lsx the number of cliques $Q_1, \ldots Q_t$ that contain $v$.
  \ls $S \gets$ \textsc{Max-Weighted-Stable-Set}$(G, w)$ \label{line:ssGetS}
  \mIf{$\omega(G \setminus S) < \omega_G$}
  \ls \RETURN $S$
  \mElse
  \ls $t \gets t+1$ \vspace{-.1cm}
  \ls $Q_t \gets$ \textsc{Max-Stable-Set}$(\overline{G \setminus S})$
  \mEndIf
  \mEndWhile
  \mEndProcedure
\end{algorithmic}

To understand \Cref{alg:ssIntersectingCliques}, let us note that $S$ calculated in line \ref{line:ssGetS} has a weight equal $t$, which means that $S$ meets each of $Q_1, \ldots Q_t$. Then, if $\omega(G \setminus S) < \omega(G)$, then $S$ meets all the maximum cardinality cliques in G so we return $S$.

There are at most $|V|$ maximum cardinality cliques in $G$. Adding a single clique to the list of maximum cardinality cliques requires constructing auxiliary graph for weighted maximum stable set, which is of size $O(|V|^2)$ and running \Cref{alg:maxWeightedStableSet} on it. Therefore total running time is polynomial.

\subsection{Minimum coloring}

\begin{alg}[Color perfect graph]
  \label{alg:minColoring}
  Input: A perfect graph $G = (V, E)$.

  \noindent Output: $C$ -- a coloring of $G$ using $\chi(G)$ colors.
\end{alg}

\begin{algorithmic}[1]
  \mProcedure{Color-Perfect-Graph}{$G$}
  \mIf{$G =$~\textsc{Get-Max-Stable-Set$(G)$}}
  \ls $C \gets$ coloring all vertices of $G$ with one color
  \ls \RETURN $C$
  \mElse
  \ls $S \gets$ \textsc{Get-Stable-Set-Intersecting-Max-Cliques}$(G)$
  \ls $C' \gets$ \textsc{Color-Perfect-Graph}$(G \setminus S)$
  \ls $C \gets C'$ with vertices of $S$ colored with an additional color
  \ls \RETURN $C$
  \mEndIf
  \mEndProcedure
\end{algorithmic}

Note that coloring $C'$ uses $\chi(G \setminus S) = \omega(G \setminus S) = \omega(G) -1$ colors, so we will call the recursion $O(|V|)$ times, each step of recursion requires polynomial time. Therefore the total running time of \Cref{alg:minColoring} is polynomial.

\section{Classical algorithms}
\label{sec:classicalColoring}

Ever since Grötschel et al. published an ellipsoid-method-based polynomial algorithm for coloring perfect graphs, a combinatorial algorithm for doing the same has been sought. As of yet, there isn't one, although there is much progress in the field.

A \emph{prism} is a graph consisting of two disjoint triangles and two disjoint paths between them. Notice that for a graph to contain no odd holes, all three paths in a prism must have the same parity. A prism with all three paths odd is called an \emph{odd prism} and a prism with all three paths even is called an \emph{even prism}.

In 2005 Maffray and Trotignon showed a coloring algorithm that colors graphs containing no odd holes, no antiholes and no prisms (sometimes called Artemis graphs) in $O(|V|^4|E|)$ time \cite{Maffray2006}. They later improved the time complexity to $O(|V|^2|E|)$ \cite{coloringArtemis}. In 2015 Maffray showed an algorithm for coloring Berge graphs with no squares\footnote{sometimes called square-free Berge graphs} (a square is a $C_4$) and no odd prisms \cite{Maff2015}.

In 2016 Chudnovsky et al. published an algorithm that given a perfect graph $G$ with $\omega(G) = k$ colors it optimally in a time polynomial for a fixed $k$ \cite{Chudnovsky2017}. In general case this algorithm has a time complexity of $O(|V|^{(k+1)^2})$.

A most recent advancement (2018) is an algorithm by Chudnovsky et al. that colors any square-free Berge graphs in time of $O(|V|^9)$ \cite{coloringSquareFree}. Before proving strong perfect graph conjecture a similar conjecture for square-free Berge graphs has been proven by Conforti et al. \cite{Conforti2004}. During one of her lectures, Maria Chudnovsky expressed hope that discovery of a full algorithm for coloring Berge graphs might follow a similar pattern.

The algorithm for coloring square-free Berge graphs is much more complex than CCLSV algorithm for Berge graphs recognition and would be much harder to implement. The main idea is that, if square-free Berge graph $G$ is not a member of class that is simple to recognize and color (Artemis graphs), then it must have an induced prism. We find it and grow it until it is maximal and then we can create a partition of $G$, that we color recursively. The paper in which the algorithm is presented \cite{coloringSquareFree} is of much more mathematical ''flavor'' than \cite{MC05} -- there is a proof that a $O(|V|^9)$ algorithm exists, but actually deducing it from the paper isn't straightforward. We provide its pseudocode in the Appendix A.