\section{Coloring Perfect Graphs}

A natural problem for perfect graphs is a problem of coloring them. In 1988 Martin Grötschel et al. published an ellipsoid-method-based polynomial algorithm for coloring perfect graphs \cite{Grtschel1993}. We consider it in section \cref{sec:ellipsoidMethod}. Hovewer due to its use of the ellipsoid method this algorithm has been usually considered unpractical \cite{coloringSquareFree,Chudnovsky2003, coloringArtemis}.

There has been much progress on the quest of finding a more classical algorithm coloring perfect graphs, without the use of ellipsoid method (see \cref{sec:classicalColoring}), however there is still no known polynomial combinatorial algorithm to do this. \todo{better wording of this paragraph}

\subsection{Ellipsoid method}
\label{sec:ellipsoidMethod}

\subsubsection{Background}

\begin{wrapfigure}{r}{0.35\textwidth}
  \label{fig:c5}
	\centering\begin{tikzpicture}[scale=.7,simplegraph]
  \def\ngon{5}
  \node[regular polygon,regular polygon sides=\ngon,minimum size=3cm, draw=none] (p) {};
  \foreach\x in {1,...,\ngon}{\node[] (p\x) at (p.corner \x){$\the\numexpr\intcalcMod{\x}{5}$};}

  \draw (p1) to (p2);
  \draw (p2) to (p3);
  \draw (p3) to (p4);
  \draw (p4) to (p5);
  \draw (p5) to (p1);

\end{tikzpicture}%
	\caption{Example noisy channel}%
	% \vspace{-0.5cm}
	% \end{floatingfigure}
\end{wrapfigure}

Let us first discuss the background in which perfect graphs first arose. Suppose we have a noisy communication channel in which certain signal values can be confused with others. For instance, suppose our channel has five discrete signal values, represented as 0, 1, 2, 3, 4. However, each value of $i$ when sent across the channel can be confused with value $(i \pm 1)$ mod $5$. This situation can be modeled by a graph $C_5$ (\cref{fig:c5}) with in which vertices correspond to signal values and two vertices are connected iff values they represent can be confused.

We are interested in transission without possibility of confusion. In other words, we are interested in the Shannon capacity of the channel modeled by a graph \cite{Lovasz1979}. For this example it is possible for two values to be transmited without ambiguity e.g. values 1 and 4, which allows us to communicate $2^n$ bits of information in $n$ steps. But we could do better, for example we could communicate five two-step codewords e.g. "00", "12", "24", "43", "31". Each pair of these codewords includes at least one position where its values differ by two or more modulo 5, which allows the recipent to distinguish them without confusion.  This allows us to send $5^{n / 2}$ bits of information in $n$ steps, or effectively send $(5^{n/2})^{1/n} = \sqrt{5}$ bits per step.

Such methods provide a lower bound on the Shannon capacity of a graph

\subsubsection{\Lovasz number}
\TODO{background, definition}

\begin{theorem}[\Lovasz "sandwich theorem"]
  For any graph $G$:
  $$ \omega(G) \leq \vartheta(\overline{G}) \leq \chi(G) $$
\end{theorem}

Because in perfect graphs $\omega(G) = \chi(G)$, we get $\omega(G) = \vartheta(G) = \chi(G)$.

\subsubsection{Computing \boldmath$\vartheta$}
\TODO{TODO}

\subsubsection{Maximum cardinality stable set}

\TODO{cite the paper on using theta to color.}

Given graph $G$, recall that stability number of $G$ is equal clique number of the complement of $G$. This gives us a way to compute $\alpha(G)$ for any perfect graph $G$.

In fact, to calculate $\chi(\overline{G})$ and $\alpha(G)$ we only need an approximated value of $\vartheta(G)$ with precision $< \frac{1}{2}$.

We will now show how to find a stable set in $G$ of size $\alpha(G)$.

\begin{alg}[maximum cardinality stable set in a perfect graph]
  \label{alg:maxStableSet}
  Input: A perfect graph $G = (V, E)$.

  \noindent Output: A maximum cardinality stable set in $G$.
\end{alg}
\begin{algtext}
  Let $v_1, \ldots v_n$ be an ordering of vertices of $G$. We will construct a sequence of induced subgraphs $G = G_0 \supseteq G_1 \supseteq \ldots \supseteq G_n$, so that $G_n$ is a required stable set.

  Let $G_0 \leftarrow G$. Then, for each $i \geq 1$, compute $\alpha(G_{i-1} \setminus v_i)$. If $\alpha(G_{i-1} \setminus v_i) = \alpha(G)$, then set $G_i \leftarrow G_{i-1} \setminus \{v_i\}$, else set $G_i \leftarrow G_{i-1}$.

  Return $G_n$.
\end{algtext}

Let us proove that $G_n$ is indeed a stable set. Suppose otherwise and let $v_iv_j$ be an edge in $G_n$ with $i < j$ and $i$ minimal. But then $\alpha(G_{i-1} \setminus v_i) = \alpha(G_{i-1}) = \alpha(G)$ so by our construction $v_i$ is not in $G_i$ and $v_iv_j$ is not an edge of $G_n$. Therefore there are no edges in $G_n$.

Because at every step we have $\alpha(G_i) = \alpha(G_{i-1})$, therefore $\alpha(G_n) = \alpha(G)$, so $G_n$ is required maximum cardinality stable set.

The running time of \cref{alg:maxStableSet} is polynomial, because we construct $n$ auxiliary graphs, each requiring calculating $\alpha$ once plus additional $O(|V|^2)$ time for constructing the graph.

Given a weight function $w : V \rightarrow \mathbb{N}$ we could calculate the maximum weighted stable set in $G$ in the following manner. Create graph $G'$ by replacing every node $v$ by a set $W_v$ of $w(v)$ nonadjacent nodes, making two nodes $x \in W_v$, $y \in W_u$ adjacent in $G'$ iff the nodes $v$, $u$ are adjacent in $G$. Then calculate a maximum cardinality stable set in $G'$ (we remark that $G'$ is still perfect because every new introduced hole is even) and return a result of those vertices in $G$ whose any (and therefore all) copies were chosen. We will use this technique later on.

\subsubsection{Stable set intersecting all maximum cardinality cliques}
Next, let us show how to find a stable set intersecting all the maximum cardinality cliques of $G$.

\begin{alg}
  \label{alg:ssIntersectingCliques}
  Input: A perfect graph $G = (V, E)$.

  \noindent Output: A stable set which intersects all the maximum cardinality cliques of $G$.
\end{alg}
\begin{algtext}
  We will create a list $Q_1, \ldots Q_t$ of all maximum cardinality cliques of $G$.

  Let $Q_1 \leftarrow$ a maximum cardinality clique of $G$. We calculate this by running \cref{alg:maxStableSet} on $\overline{G}$.

  Now suppose $Q_1, \ldots, Q_t$ have been found. We show how to calculate $Q_{t+1}$ or see that we are done.

  Let us define a weight function $w : V \rightarrow \mathbb{N}$, so that for $v \in V$, $w(v)$ is equal to the number of cliques $Q_1, \ldots Q_t$ that contain $v$.

  Assign $S \leftarrow$ the maximum $w$-weighted stable set, as described in a remark to \cref{alg:maxStableSet}. It is easy to see that $S$ has weight $t$, which means that $S$ meets each of $Q_1, \ldots Q_t$.

  If $\omega(G \setminus S) < \omega(G)$, then $S$ meets all the maximum cardinality cliques in G so we return $S$. Otherwise we find a maximum cardinality clique in $G \setminus S$ (it will be of size $\omega(G)$, because $\omega(G \setminus S) = \omega(G)$), add it to our list as $Q_{t+1}$ and continue with longer list.

\end{algtext}

There are at most $|V|$ maximum cardinality cliques in $G$. Adding a single clique to the list of maximum cardinality cliques requires constructing auxiliary graph for weighted maximum stable set, which is of size $O(V^2)$ and running \cref{alg:maxStableSet} on it. Therefore total running time is polynomial.

\subsubsection{Minimum coloring}

\begin{alg}
  \label{alg:minColoring}
  Input: A perfect graph $G = (V, E)$.

  \noindent Output: A coloring of $G$ using $\chi(G)$ colors.
\end{alg}
\begin{algtext}
  If $G$ is equal to its maximum cardinality stable set, color all verces one color and return.

  Else find $S$ intersecting all maximum cardinality cliques of $G$ (\cref{alg:ssIntersectingCliques}). Color recursively all vertices of $G \setminus S$ with $\chi(G \setminus S) = \omega(G \setminus S) = \omega(G) -1$ colors and all vertices of $S$ with one additional color.
\end{algtext}

We will call recursion at most $O(|V|)$ times, each step of recursion is polynomial in time. Therefore the running time of \cref{alg:minColoring} is polynomial.

\subsection{Classical algorithms}
\label{sec:classicalColoring}