\documentclass{beamer}
\usetheme{Madrid}
\usepackage[utf8]{inputenc}
\usepackage{graphicx}
\usepackage{tikz}
\usepackage{bm}
\usepackage{wasysym}
\usepackage{centernot}
\usepackage{mathtools}
\usepackage{xcolor}
\usetikzlibrary{arrows}
\graphicspath{ {./img/} }

\DeclarePairedDelimiter\ceil{\lceil}{\rceil}
\DeclarePairedDelimiter\floor{\lfloor}{\rfloor}

\title{Perfect Graph Recognition and Coloring}
\subtitle{Dni Magistranta}
\author{Adrian Siwiec}


\date{December 05, 2019}

\definecolor{c1}{RGB}{51,51,178}
\definecolor{c2}{RGB}{255,173,5}
\definecolor{c3}{RGB}{20,17,21}
\definecolor{c4}{RGB}{255,34,12}
\definecolor{c5}{RGB}{184,184,243}
\definecolor{ofaded}{RGB}{255,173,5}
\definecolor{tfaded}{RGB}{51,51,178}
\definecolor{ududff}{rgb}{0,0,0}

\tikzstyle{taken}=[color=c1]
\tikzstyle{optimal}=[color=c2]
\tikzstyle{od}=[color=c2, dashed]
\tikzstyle{l}=[above, color=black]
\tikzstyle{lr}=[left, color=black]
\tikzstyle{sm}=[scale=0.6]

\newcommand{\smm}{\mathsf{SMM}}
\newcommand{\smmmove}{\textsc{SMMMove}}
\newcommand{\tqbf}{\textsc{TQBF}}
\newcommand{\mpmove}{\textsc{MPMove}}
\newcommand{\dechex}{\textsc{DecisionHex}}
\newcommand{\decpos}{\textsc{DecisionPoset}}
\newcommand{\PSPACE}{\textsc{PSPACE}}
\newcommand{\PSPC}{\textsc{PSPACE}\emph{-complete}}
\newcommand{\PSPH}{\textsc{PSPACE}\emph{-hard}}


\tikzstyle{vertex}=[vertex/.style = {shape=circle,draw, fill=black, scale=0.5}]
\tikzstyle{edge}=[edge/.style = {-,> = latex', line width=1mm}]

\begin{document}

\begin{frame}
\titlepage
\end{frame}

\begin{frame}{References}
	\textbf{"The strong perfect graph theorem"}
	\\ \emph{Maria Chudnovsky, Neil Robertson, Paul Seymour, Robin Thomas}
	\\ https://arxiv.org/abs/math/0212070
	\vspace{0.3cm}\\
	\textbf{"A polynomial algorithm for recognizing perfect graphs"}
	\\ \emph{Gérard Cornuéjols, Xinming Liu, Kristina Vušković}
	\\ https://ieeexplore.ieee.org/document/1238177
	\vspace{0.3cm}\\
	\textbf{"Recognizing Berge Graphs"}
	\\ \emph{Maria Chudnovsky, Gérard Cornuéjols, Xinming Liu, Paul Seymour, Kristina Vušković}
	\\ https://link.springer.com/article/10.1007/s00493-005-0012-8
	\vspace{0.3cm}\\
	\textbf{"Colouring perfect graphs with bounded clique number"}
	\\ \emph{Maria Chudnovsky, Aurélie Lagoutte, Paul Seymour, Sophie Spirkl}
	\\ https://arxiv.org/abs/1707.03747
	
\end{frame}

\begin{frame}{Perfect Graphs}
\begin{itemize}
\item<1->[]\begin{block}{Perfect Graphs}
A graph is \emph{perfect} if the chromatic number of every induced subgraph equals the size of its largest clique.
\end{block}
\item<2->[]\vspace{-0.55cm}\begin{block}{Perfect Graphs are interesting}
In all perfect graphs, the \emph{graph coloring problem}, \emph{maximum clique problem}, and \emph{maximum independent set problem} can all be solved in polynomial time. (Grötschel, Lovász, Schrijver 1988)
\end{block}
\item<3->[]\vspace{-0.55cm}\begin{block}{Families of graphs that are perfect}
\begin{itemize}
\item Bipartite graphs
\vspace{-0.1cm}\item Line graphs of bipartite graphs
\vspace{-0.1cm}\item Chordal graphs
\vspace{-0.1cm}\item Comparability graphs
\vspace{-0.1cm}\item ...
\end{itemize}
\end{block}

\end{itemize}
\end{frame}




\begin{frame}{Strong Perfect Graph Theorem}
\begin{itemize}
\item<1->[]\begin{block}{Perfect Graphs}
A graph is \emph{perfect} if the chromatic number of every induced subgraph equals the size of its largest clique.
\end{block}
\begin{block}{Berge Graphs}
A graph is \emph{Berge} if no induced subgraph of G is an odd cycle of length at least five or the
complement of one.
\end{block}
\item<2->[]\begin{block}{Strong Perfect Graph Theorem}
A graph $G$ is Perfect if and only if it is Berge.
\end{block}
\end{itemize}
\end{frame}

\begin{frame}{Recognizing Berge Graphs - an overview of an overview}
\begin{block}{An odd hole}
An odd hole in G is an induced subgraph of G that is a cycle of odd length at least five.
\end{block}
A graph G is Berge if G and its complement both have no odd hole.

\begin{block}{Algorithm}
The idea of our algorithm is to decompose the input graph $G$ into a polynomial number of simpler graphs
$G_1, ..., G_m$ so that:
\begin{itemize}
\item $G$ is odd-hole-free if and only if every $G_i$ is odd-hole-free.
\item $G$ it is easy to check if $G_i$ is odd-hole-free.
\end{itemize}
\end{block}

\end{frame}

\begin{frame}{Decompositions}

\begin{block}{2-join}
A graph $G$ has a \emph{2-join} $V_1|V_2$ with special sets $(A_1, A_2, B_1, B_2)$ if $A_i, B_i \subset V_i$, every vertex of $A_1$ is adjacent to every vertex of $A_2$, every vertex of $B_1$ is adjacent to every vertex of $B_2$ and there are no other adjacencies between $V_1$ and $V_2$.
\end{block}

\begin{block}{Double Star}
A set $S$ of vertices is a double star if $S$ contains two adjacent vertices $u$ and $v$ such that
$S \subseteq N(u) \cup N(v)$.
\end{block}
Double-star cutsets pose a problem.

\end{frame}


\begin{frame}{Perfect Graph Colouring}
\begin{block}{Skew Partitions}
A \emph{skew partition} in $G$ is a partition $(A, B)$ of $V(G)$, such that $G[A]$ is not connected and $\overline{G}[B]$ is not connected.
\end{block}

\begin{block}{Decomposition Theorem}
Every Berge graph either admits a balanced skew partition, or admits one of two other decompositions, or it belongs to one of five well-understood classes.
\end{block}

\end{frame}

\begin{frame}{Goals}
\begin{block}{Goals}
\begin{itemize}
\item Implement Berge Graph recognition algorithm.
\begin{itemize}
\item Compare with existing \emph{Java} implementation.
\item Are there any other implementations to compare to?
\end{itemize}
\item Implement Perfect Graph colouring algorithm.
\begin{itemize}
\item Compare with existing linear programming solutions.
\end{itemize}
\end{itemize}
\end{block}
\end{frame}

%
%\begin{frame}{Further Definitions}
%\begin{block}{Competetive Ratio}
%	An online algorithm $ALG$ is \emph{c-competetive} if $ALG(\sigma) \geq OPT(\sigma)/c$ for all request sequences $\sigma$, where $OPT$ is the value of optimal solution.
%\end{block}		
%\begin{block}{Edge Type}
%	An edge is of type $i \in [0,k]$ if it has udergone i decision flips by the algorithm. The default state of an edge is \emph{rejected}.
%\end{block}
%\begin{block}{Path Type}
%	The type of a path $P$ is defined as a sequence of types of its edges.
%\end{block}
%	\centering\emph{Example:} $01010 \longrightarrow 12121$
%\end{frame}
%
%\begin{frame}{The algorithm \emph{Greedy}}
%\begin{block}{\emph{Greedy}}
%	The algorithm \emph{Greedy} repeatedly applies an orbitrary augmenting path whenever possible.
%\end{block}
%\begin{block}{Theorem 1}
%	The algorithm \emph{Greedy} achieves the competetive ratio of $3/2$ for every even $k$ and $2$ for every odd $k$.
%\end{block}
%\end{frame}
%
%\begin{frame}{Proof of Theorem 1 (Upper Bound)}
%\begin{block}{Even alternating paths (symmetric difference)}
%\begin{minipage}{.6\textwidth}
%	\centering\begin{tikzpicture}[baseline=(current bounding box.center)]
%		\tikzset{vertex}
%		\tikzset{edge}
%		\node[vertex] (1) at  (0,0) {};
%		\node[vertex] (2) at  (1,0) {};
%		\node[vertex] (3) at  (2,0) {};
%		\node[vertex] (4) at  (3,0) {};
%		\node[vertex] (5) at  (4,0) {};
%
%		\draw[edge, taken] (1) to (2);
%		\draw[edge, optimal] (2) to (3);
%		\draw[edge, taken] (3) to (4);
%		\draw[edge, optimal] (4) to (5);
%	\end{tikzpicture}
%\end{minipage}%
%\begin{minipage}{.25\textwidth}
%	\qquad
%	\centering\begin{tikzpicture}
%		\tikzset{vertex}
%		\tikzset{edge}
%		\node[vertex] (1) at  (0,0) {};
%		\node[vertex] (2) at  (1,0) {};
%		\node[vertex] (3) at  (1.5,0.87) {};
%		\node[vertex] (4) at  (1,1.74) {};
%		\node[vertex] (5) at  (0,1.74) {};
%		\node[vertex] (6) at  (-0.5,0.87) {};
%		\draw[edge, taken] (1) to (2);
%		\draw[edge, optimal] (2) to (3);
%		\draw[edge, taken] (3) to (4);
%		\draw[edge, optimal] (4) to (5);
%		\draw[edge, taken] (5) to (6);
%		\draw[edge, optimal] (6) to (1);
%	\end{tikzpicture}
%\end{minipage}
%\end{block}
%\begin{block}{Odd alternating paths (symmetric difference)}
%\begin{minipage}{.6\textwidth}
%	\centering\begin{tikzpicture}[baseline=(current bounding box.center)]
%		\tikzset{vertex}
%		\tikzset{edge}
%		\node[vertex, fill=black!50, draw=black!50] (1) at  (0,0) {};
%		\node[vertex, fill=black!50, draw=black!50] (2) at  (1,0) {};
%		\draw[edge, ofaded!50] (1) to (2);
%	\end{tikzpicture}
%\end{minipage}%
%\begin{minipage}{.25\textwidth}
%	\qquad
%	\centering\begin{tikzpicture}
%		\tikzset{vertex}
%		\tikzset{edge}
%		\node[vertex, fill=black!50, draw=black!50] (1) at  (0,0) {};
%		\node[vertex, fill=black!50, draw=black!50] (2) at  (1,0) {};
%		\node[vertex, fill=black!50, draw=black!50] (3) at  (2,0) {};
%		\node[vertex, fill=black!50, draw=black!50] (4) at  (3,0) {};
%		\draw[edge, ofaded!50] (1) to (2);
%		\draw[edge, tfaded!50] (2) to node[l]  {$i \centernot| 2$} (3);
%		\draw[edge, ofaded!50] (3) to  (4);
%	\end{tikzpicture}
%\end{minipage}%
%\vspace{0.9cm}
%\begin{minipage}{.6\textwidth}
%	\centering\begin{tikzpicture}[baseline=(current bounding box.center)]
%		\tikzset{vertex}
%		\tikzset{edge}
%		\node[vertex] (1) at  (0,0) {};
%		\node[vertex] (2) at  (1,0) {};
%		\node[vertex] (3) at  (2,0) {};
%		\node[vertex] (4) at  (3,0) {};
%		\node[vertex] (5) at  (4,0) {};
%		\node[vertex] (6) at  (5,0) {};
%		\draw[edge, optimal] (1) to (2);
%		\draw[edge, optimal] (3) to (4);
%		\draw[edge, optimal] (5) to (6);
%		\draw[edge, taken] (2) to (3);
%		\draw[edge, taken] (4) to (5);
%	\end{tikzpicture}
%\end{minipage}%
%\begin{minipage}{.18\textwidth}
%	\qquad
%	\centering\begin{tikzpicture}
%		\tikzset{vertex}
%		\tikzset{edge}
%		$\dots$
%	\end{tikzpicture}
%\end{minipage}
%\vspace{0.4cm}
%\end{block}
%\end{frame}
%
%\begin{frame}{Proof of Theorem 1 (Lower Bound)}
%$k = 4, n = 2$
%\begin{center}
%\begin{itemize}
%	\item<1->[]\hspace*{2.5cm}\begin{tikzpicture}[baseline=(current bounding box.center), sm]
%		\tikzset{vertex}
%		\tikzset{edge}
%		\node[vertex] (1) at  (0,0) {};
%		\node[vertex] (2) at  (1,0) {};
%		\node[vertex] (3) at  (2,0) {};
%		\node[vertex] (4) at  (3,0) {};
%		\node[vertex] (5) at  (4,0) {};
%		\node[vertex] (6) at  (5,0) {};
%		\node[vertex] (7) at  (6,0) {};
%		\node[vertex] (8) at  (7,0) {};
%		\node[vertex] (9) at  (8,0) {};
%		\node[vertex] (10) at  (9,0) {};
%		\draw[edge, taken] (1) to node[l] {$1$} (2);
%		\draw[edge, taken] (3) to node[l] {$1$} (4);
%		\draw[edge, taken] (5) to node[l] {$1$} (6);
%		\draw[edge, taken] (7) to node[l] {$1$} (8);
%		\draw[edge, taken] (9) to node[l] {$1$} (10);
%		\draw[edge, od] (1) to node[l] {$1$} (2);
%		\draw[edge, od] (3) to node[l] {$1$} (4);
%		\draw[edge, od] (5) to node[l] {$1$} (6);
%		\draw[edge, od] (7) to node[l] {$1$} (8);
%		\draw[edge, od] (9) to node[l] {$1$} (10);
%	\end{tikzpicture}%
%
%	\vspace{0.4cm}
%	\item<2->[]\hspace*{1.9cm}\begin{tikzpicture}[baseline=(current bounding box.center), sm]
%		\tikzset{vertex}
%		\tikzset{edge}
%		\node[vertex] (1) at  (0,0) {};
%		\node[vertex] (2) at  (1,0) {};
%		\node[vertex] (3) at  (2,0) {};
%		\node[vertex] (4) at  (3,0) {};
%		\node[vertex] (5) at  (4,0) {};
%		\node[vertex] (6) at  (5,0) {};
%		\node[vertex] (7) at  (6,0) {};
%		\node[vertex] (8) at  (7,0) {};
%		\node[vertex] (9) at  (8,0) {};
%		\node[vertex] (10) at  (9,0) {};
%		\node[vertex] (11) at  (10,0) {};
%		\node[vertex] (12) at  (11,0) {};
%		\draw[edge, taken] (1) to node[l] {$1$} (2);
%		\draw[edge, taken] (3) to node[l] {$1$} (4);
%		\draw[edge, taken] (5) to node[l] {$1$} (6);
%		\draw[edge, taken] (7) to node[l] {$1$} (8);
%		\draw[edge, taken] (9) to node[l] {$1$} (10);
%		\draw[edge, taken] (11) to node[l] {$1$} (12);
%		\draw[edge, od] (1) to node[l] {$1$} (2);
%		\draw[edge, od] (3) to node[l] {$1$} (4);
%		\draw[edge, od] (5) to node[l] {$1$} (6);
%		\draw[edge, od] (7) to node[l] {$1$} (8);
%		\draw[edge, od] (9) to node[l] {$1$} (10);
%		\draw[edge, od] (11) to node[l] {$1$} (12);
%		\draw[dashed] (10) to node[l] {$2$} (11);
%		\draw[dashed] (8) to node[l] {$2$} (9);
%		\draw[dashed] (6) to node[l] {$2$} (7);
%		\draw[dashed] (4) to node[l] {$2$} (5);
%		\draw[dashed] (2) to node[l] {$2$} (3);
%	\end{tikzpicture}%
%	\vspace{0.4cm}
%	\onslide<3->\hspace*{1.3cm}\begin{tikzpicture}[baseline=(current bounding box.center), sm]
%		\tikzset{vertex}
%		\tikzset{edge}
%		\node[vertex] (1) at  (0,0) {};
%		\node[vertex] (2) at  (1,0) {};
%		\node[vertex] (3) at  (2,0) {};
%		\node[vertex] (4) at  (3,0) {};
%		\node[vertex] (5) at  (4,0) {};
%		\node[vertex] (6) at  (5,0) {};
%		\node[vertex] (7) at  (6,0) {};
%		\node[vertex] (8) at  (7,0) {};
%		\node[vertex] (9) at  (8,0) {};
%		\node[vertex] (10) at  (9,0) {};
%		\node[vertex] (11) at  (10,0) {};
%		\node[vertex] (12) at  (11,0) {};
%		\node[vertex] (13) at  (12,0) {};
%		\node[vertex] (14) at  (13,0) {};
%		\draw[edge, taken] (1) to node[l] {$1$} (2);
%		\draw[edge, taken] (3) to node[l] {$3$} (4);
%		\draw[edge, taken] (5) to node[l] {$3$} (6);
%		\draw[edge, taken] (7) to node[l] {$3$} (8);
%		\draw[edge, taken] (9) to node[l] {$3$} (10);
%		\draw[edge, taken] (11) to node[l] {$3$} (12);
%		\draw[edge, taken] (13) to node[l] {$1$} (14);
%		\draw[edge, od] (1) to node[l] {$1$} (2);
%		\draw[edge, od] (3) to node[l] {$3$} (4);
%		\draw[edge, od] (5) to node[l] {$3$} (6);
%		\draw[edge, od] (7) to node[l] {$3$} (8);
%		\draw[edge, od] (9) to node[l] {$3$} (10);
%		\draw[edge, od] (11) to node[l] {$3$} (12);
%		\draw[edge, od] (13) to node[l] {$1$} (14);
%		\draw[dashed] (12) to node[l] {$2$} (13);
%		\draw[dashed] (10) to node[l] {$2$} (11);
%		\draw[dashed] (8) to node[l] {$2$} (9);
%		\draw[dashed] (6) to node[l] {$2$} (7);
%		\draw[dashed] (4) to node[l] {$2$} (5);
%		\draw[dashed] (2) to node[l] {$2$} (3);
%	\end{tikzpicture}%
%	\vspace{0.4cm}
%	\onslide<4->\hspace*{0.7cm}\begin{tikzpicture}[baseline=(current bounding box.center), sm]
%		\tikzset{vertex}
%		\tikzset{edge}
%		\node[vertex] (1) at  (0,0) {};
%		\node[vertex] (2) at  (1,0) {};
%		\node[vertex] (3) at  (2,0) {};
%		\node[vertex] (4) at  (3,0) {};
%		\node[vertex] (5) at  (4,0) {};
%		\node[vertex] (6) at  (5,0) {};
%		\node[vertex] (7) at  (6,0) {};
%		\node[vertex] (8) at  (7,0) {};
%		\node[vertex] (9) at  (8,0) {};
%		\node[vertex] (10) at  (9,0) {};
%		\node[vertex] (11) at  (10,0) {};
%		\node[vertex] (12) at  (11,0) {};
%		\node[vertex] (13) at  (12,0) {};
%		\node[vertex] (14) at  (13,0) {};
%		\node[vertex] (15) at  (14,0) {};
%		\node[vertex] (16) at  (15,0) {};
%		\draw[edge, taken] (1) to node[l] {$1$} (2);
%		\draw[edge, taken] (3) to node[l] {$3$} (4);
%		\draw[edge, taken] (5) to node[l] {$3$} (6);
%		\draw[edge, taken] (7) to node[l] {$3$} (8);
%		\draw[edge, taken] (9) to node[l] {$3$} (10);
%		\draw[edge, taken] (11) to node[l] {$3$} (12);
%		\draw[edge, taken] (13) to node[l] {$1$} (14);
%		\draw[edge, taken] (15) to node[l] {$1$} (16);
%		\draw[edge, od] (1) to node[l] {$1$} (2);
%		\draw[edge, od] (3) to node[l] {$3$} (4);
%		\draw[edge, od] (5) to node[l] {$3$} (6);
%		\draw[edge, od] (7) to node[l] {$3$} (8);
%		\draw[edge, od] (9) to node[l] {$3$} (10);
%		\draw[edge, od] (11) to node[l] {$3$} (12);
%		\draw[edge, od] (13) to node[l] {$1$} (14);
%		\draw[edge, od] (15) to node[l] {$1$} (16);
%		\draw[dashed] (14) to node[l] {$2$} (15);
%		\draw[dashed] (12) to node[l] {$4$} (13);
%		\draw[dashed] (10) to node[l] {$4$} (11);
%		\draw[dashed] (8) to node[l] {$4$} (9);
%		\draw[dashed] (6) to node[l] {$4$} (7);
%		\draw[dashed] (4) to node[l] {$4$} (5);
%		\draw[dashed] (2) to node[l] {$2$} (3);
%	\end{tikzpicture}%
%	\vspace{0.4cm}
%	\onslide<5->\hspace*{0.7cm}\begin{tikzpicture}[baseline=(current bounding box.center), sm]
%		\tikzset{vertex}
%		\tikzset{edge}
%		\node[vertex] (1) at  (0,0) {};
%		\node[vertex] (2) at  (1,0) {};
%		\node[vertex] (3) at  (2,0) {};
%		\node[vertex] (4) at  (3,0) {};
%		\node[vertex] (5) at  (4,0) {};
%		\node[vertex] (6) at  (5,0) {};
%		\node[vertex] (7) at  (6,0) {};
%		\node[vertex] (8) at  (7,0) {};
%		\node[vertex] (9) at  (8,0) {};
%		\node[vertex] (10) at  (9,0) {};
%		\node[vertex] (11) at  (10,0) {};
%		\node[vertex] (12) at  (11,0) {};
%		\node[vertex] (13) at  (12,0) {};
%		\node[vertex] (14) at  (13,0) {};
%		\node[vertex] (15) at  (14,0) {};
%		\node[vertex] (16) at  (15,0) {};
%		\node[vertex] (24) at  (3,-1) {};
%		\node[vertex] (25) at  (4,-1) {};
%		\node[vertex] (28) at  (7,-1) {};
%		\node[vertex] (29) at  (8,-1) {};
%		\node[vertex] (32) at  (11,-1) {};
%		\node[vertex] (33) at  (12,-1) {};
%		\draw[edge, taken] (1) to node[l] {$1$} (2);
%		\draw[edge, taken] (3) to node[l] {$3$} (4);
%		\draw[edge, taken] (5) to node[l] {$3$} (6);
%		\draw[edge, taken] (7) to node[l] {$3$} (8);
%		\draw[edge, taken] (9) to node[l] {$3$} (10);
%		\draw[edge, taken] (11) to node[l] {$3$} (12);
%		\draw[edge, taken] (13) to node[l] {$1$} (14);
%		\draw[edge, taken] (15) to node[l] {$1$} (16);
%		\draw[edge, optimal] (14) to node[l] {$2$} (15);
%		\draw[dashed] (12) to node[l] {$4$} (13);
%		\draw[edge, optimal] (10) to node[l] {$4$} (11);
%		\draw[dashed] (8) to node[l] {$4$} (9);
%		\draw[edge, optimal] (6) to node[l] {$4$} (7);
%		\draw[dashed] (4) to node[l] {$4$} (5);
%		\draw[edge, optimal] (2) to node[l] {$2$} (3);
%		\draw[edge, optimal] (4) to (24);
%		\draw[edge, optimal] (5) to (25);
%		\draw[edge, optimal] (8) to (28);
%		\draw[edge, optimal] (9) to (29);
%		\draw[edge, optimal] (12) to (32);
%		\draw[edge, optimal] (13) to (33);
%	\end{tikzpicture}
%
%\end{itemize}
%\end{center}
%\end{frame}
%
%\begin{frame}{The \lga{} }
%\begin{block}{The $L$-Greedy Algorithm}
%	We define the \lga{} for some parameter $L$ as a greedy algorithm that applies only augmenting paths of lengths at most $2L+1$ that are in the symmetric difference between the current matching and some particular matching $OPT$. The $OPT$ is then updated using some augmenting path.
%\end{block}
%\end{frame}
%
%\begin{frame}{The \lga{} }
%\begin{block}{The competetive ratio}
%	The competetive ratio of \lga{} with $L = \floor*{\sqrt{k-1}}$ for even $k \geq 6$ is at most:\\
%	\centering\scalebox{1.3}{\centering${{k(L+2)-2}\over{(L+1)(k-1)}} = 1 + O({{1}\over{\sqrt{k}}})$}
%\end{block}
%\end{frame}
%
%\begin{frame}{Lower bound for deterministic algorithms}
%\begin{block}{Theorem 2}
%	The deterministic competetive ratio for $k \geq 3$ is at least $1 + {{1}\over{k-1}}$
%\end{block}
%\end{frame}
%
%\begin{frame}{Lower bound for $k=3$ is $3/2$}
%\begin{center}
%\begin{overprint}
%\onslide<1>\hspace*{1cm}\begin{tikzpicture}[baseline=(current bounding box.center)]
%		\tikzset{vertex}
%		\tikzset{edge}
%		\node[vertex] (1) at  (0,0) {};
%		\node[vertex] (2) at  (1,0) {};
%		\node[draw=none] (20) at  (6, 1) {};
%		\node[draw=none] (21) at  (-4.5,-4) {};
%		\draw[edge, optimal] (1) to (2);
%\end{tikzpicture}
%
%\onslide<2>\hspace*{1cm}\begin{tikzpicture}[baseline=(current bounding box.center)]
%		\tikzset{vertex}
%		\tikzset{edge}
%		\node[vertex] (1) at  (0,0) {};
%		\node[vertex] (2) at  (1,0) {};
%		\node[draw=none] (20) at  (6, 1) {};
%		\node[draw=none] (21) at  (-4.5,-4) {};
%		\draw[edge, taken] (1) to node[l] {$1$} (2);
%		\draw[edge, od] (1) to (2);
%\end{tikzpicture}
%
%\onslide<3>\hspace*{1cm}\begin{tikzpicture}[baseline=(current bounding box.center)]
%		\tikzset{vertex}
%		\tikzset{edge}
%		\node[draw=none] (20) at  (0, 1) {};
%		\node[vertex] (1) at  (0,0) {};
%		\node[vertex] (2) at  (1,0) {};
%		\node[vertex] (3) at  (2,0) {};
%		\node[vertex] (4) at  (0,-1) {};
%		\node[draw=none] (20) at  (6, 1) {};
%		\node[draw=none] (21) at  (-4.5,-4) {};
%		\draw[edge, taken] (1) to node[l] {$1$} (2);
%		\draw[edge, optimal] (3) to (2);
%		\draw[edge, optimal] (1) to (4);
%\end{tikzpicture}
%
%\onslide<4>\hspace*{1cm}\begin{tikzpicture}[baseline=(current bounding box.center)]
%		\tikzset{vertex}
%		\tikzset{edge}
%		\node[vertex] (1) at  (0,0) {};
%		\node[vertex] (2) at  (1,0) {};
%		\node[vertex] (3) at  (2,0) {};
%		\node[vertex] (4) at  (0,-1) {};
%		\node[draw=none] (20) at  (6, 1) {};
%		\node[draw=none] (21) at  (-4.5,-4) {};
%		\draw[edge, taken] (1) to node[l] {$1$} (2);
%		\draw[edge, optimal] (3) to (2);
%		\draw[edge, optimal] (1) to (4);
%		\pocket{}
%\end{tikzpicture}
%
%\onslide<5>\hspace*{1cm}\begin{tikzpicture}[baseline=(current bounding box.center)]
%		\tikzset{vertex}
%		\tikzset{edge}
%		\node[vertex] (1) at  (0,0) {};
%		\node[vertex] (2) at  (1,0) {};
%		\node[vertex] (3) at  (2,0) {};
%		\node[vertex] (4) at  (0,-1) {};
%		\node[draw=none] (20) at  (6, 1) {};
%		\node[draw=none] (21) at  (-4.5,-4) {};
%		\draw[dashed] (1) to node[l] {$2$} (2);
%		\draw[edge, taken] (3) to node[l] {$1$} (2);
%		\draw[edge, taken] (1) to node[lr] {$1$} (4);
%		\draw[edge, od] (3) to (2);
%		\draw[edge, od] (1) to (4);
%		\pocket{}
%\end{tikzpicture}
%
%\onslide<6>\hspace*{1cm}\begin{tikzpicture}[baseline=(current bounding box.center)]
%		\tikzset{vertex}
%		\tikzset{edge}
%		\node[vertex] (1) at  (0,0) {};
%		\node[vertex] (2) at  (1,0) {};
%		\node[vertex] (3) at  (2,0) {};
%		\node[vertex] (4) at  (0,-1) {};
%		\node[vertex] (5) at  (3,0) {};
%		\node[vertex] (6) at  (1,-1) {};
%		\node[draw=none] (20) at  (6, 1) {};
%		\node[draw=none] (21) at  (-4.5,-4) {};
%		\draw[edge, optimal] (1) to node[l] {$2$} (2);
%		\draw[edge, taken] (3) to node[l] {$1$} (2);
%		\draw[edge, taken] (1) to node[lr] {$1$} (4);
%		\draw[edge, optimal] (6) to (4);
%		\draw[edge, optimal] (3) to (5);
%		\pocket{}
%\end{tikzpicture}
%
%\onslide<7>\hspace*{1cm}\begin{tikzpicture}[baseline=(current bounding box.center)]
%		\tikzset{vertex}
%		\tikzset{edge}
%		\node[vertex] (1) at  (0,0) {};
%		\node[vertex] (2) at  (1,0) {};
%		\node[vertex] (3) at  (2,0) {};
%		\node[vertex] (4) at  (0,-1) {};
%		\node[vertex] (5) at  (3,0) {};
%		\node[vertex] (6) at  (1,-1) {};
%		\node[draw=none] (20) at  (6, 1) {};
%		\node[draw=none] (21) at  (-4.5,-4) {};
%		\draw[edge, taken] (1) to (2);
%		\draw[edge, od] (1) to node[l] {$3$} (2);
%		\draw[dashed] (3) to node[l] {$2$} (2);
%		\draw[dashed] (1) to node[lr] {$2$} (4);
%		\draw[edge, taken] (6) to node[l] {$1$} (4);
%		\draw[edge, od] (6) to (4);
%		\draw[edge, taken] (3) to node[l] {$1$} (5);
%		\draw[edge, od] (3) to (5);
%		\pocket{}
%\end{tikzpicture}
%
%\onslide<8>\hspace*{1cm}\begin{tikzpicture}[baseline=(current bounding box.center)]		
%		\tikzset{vertex}
%		\tikzset{edge}
%		\node[vertex] (1) at  (0,0) {};
%		\node[vertex] (2) at  (1,0) {};
%		\node[vertex] (3) at  (2,0) {};
%		\node[vertex] (4) at  (0,-1) {};
%		\node[vertex] (5) at  (3,0) {};
%		\node[vertex] (6) at  (1,-1) {};
%		\node[vertex] (7) at  (4,0) {};
%		\node[vertex] (8) at  (2,-1) {};
%		\node[vertex] (9) at  (0,1) {};
%		\node[vertex] (10) at  (1,1) {};
%		\node[draw=none] (20) at  (6, 1) {};
%		\node[draw=none] (21) at  (-4.5,-4) {};
%		\draw[edge, taken] (1) to node[l] {$3$} (2);
%		\draw[dashed] (3) to node[l] {$2$} (2);
%		\draw[dashed] (1) to node[lr] {$2$} (4);
%		\draw[edge, taken] (6) to node[l] {$1$} (4);
%		\draw[edge, od] (6) to (4);
%		\draw[edge, taken] (3) to node[l] {$1$} (5);
%		\draw[edge, optimal] (3) to (8);
%		\draw[edge, optimal] (7) to (5);
%		\draw[edge, optimal] (9) to (1);
%		\draw[edge, optimal] (10) to (2);
%		\pocket{}
%\end{tikzpicture}
%
%\onslide<9>\hspace*{1cm}\begin{tikzpicture}[baseline=(current bounding box.center)]
%		\filldraw[draw=none, fill=lightgray, rounded corners] (1.7,0.6) rectangle (4.3,-1.3);
%		\filldraw[draw=none, fill=lightgray, rounded corners] (-4.3,-2.4) rectangle (-1.7,-4.3);		
%		\tikzset{vertex}
%		\tikzset{edge}
%		\node[vertex] (1) at  (0,0) {};
%		\node[vertex] (2) at  (1,0) {};
%		\node[vertex] (3) at  (2,0) {};
%		\node[vertex] (4) at  (0,-1) {};
%		\node[vertex] (5) at  (3,0) {};
%		\node[vertex] (6) at  (1,-1) {};
%		\node[vertex] (7) at  (4,0) {};
%		\node[vertex] (8) at  (2,-1) {};
%		\node[vertex] (9) at  (0,1) {};
%		\node[vertex] (10) at  (1,1) {};
%		\node[draw=none] (20) at  (6, 1) {};
%		\node[draw=none] (21) at  (-4.5,-4) {};
%		\draw[edge, taken] (1) to node[l] {$3$} (2);
%		\draw[dashed] (3) to node[l] {$2$} (2);
%		\draw[dashed] (1) to node[lr] {$2$} (4);
%		\draw[edge, taken] (6) to node[l] {$1$} (4);
%		\draw[edge, od] (6) to (4);
%		\draw[edge, taken] (3) to node[l] {$1$} (5);
%		\draw[edge, optimal] (3) to (8);
%		\draw[edge, optimal] (7) to (5);
%		\draw[edge, optimal] (9) to (1);
%		\draw[edge, optimal] (10) to (2);
%		\pocket{}
%\end{tikzpicture}
%
%\onslide<10>\hspace*{1cm}\begin{tikzpicture}[baseline=(current bounding box.center)]
%		\filldraw[draw=none, fill=lightgray, rounded corners] (3.7,0.6) rectangle (6.3,-1.3);
%		\filldraw[draw=none, fill=lightgray, rounded corners] (-4.3,-2.4) rectangle (-1.7,-4.3);		
%		\tikzset{vertex}
%		\tikzset{edge}
%		\node[vertex] (1) at  (0,0) {};
%		\node[vertex] (2) at  (1,0) {};
%		\node[vertex] (3) at  (2,0) {};
%		\node[vertex] (4) at  (0,-1) {};
%		\node[vertex] (5) at  (3,0) {};
%		\node[vertex] (6) at  (1,-1) {};
%		\node[vertex] (7) at  (4,0) {};
%		\node[vertex] (8) at  (2,-1) {};
%		\node[vertex] (9) at  (0,1) {};
%		\node[vertex] (10) at  (1,1) {};
%		\node[vertex] (11) at  (6,0) {};
%		\node[vertex] (12) at  (5,0) {};
%		\node[vertex] (13) at  (3,-1) {};
%		\node[vertex] (14) at  (2,1) {};
%		\node[vertex] (15) at  (3,1) {};
%		\node[vertex] (16) at  (4,-1) {};
%		\node[draw=none] (20) at  (6, 1) {};
%		\node[draw=none] (21) at  (-4.5,-4) {};
%		\node[vertex, draw=none] (20) at  (0,-1) {};
%		\draw[edge, taken] (1) to node[l] {$3$} (2);
%		\draw[dashed] (3) to node[l] {$2$} (2);
%		\draw[dashed] (1) to node[lr] {$2$} (4);
%		\draw[edge, taken] (6) to node[l] {$1$} (4);
%		\draw[edge, od] (6) to (4);
%		\draw[edge, taken] (3) to node[l] {$3$} (5);
%		\draw[dashed] (7) to node[l] {$2$} (5);
%		\draw[edge, optimal] (9) to (1);
%		\draw[edge, optimal] (10) to (2);
%		\draw[edge, taken] (13) to node[l] {$1$} (8);
%		\draw[edge, od] (13) to (8);
%		\draw[dashed] (3) to node[lr] {$2$} (8);
%		\draw[edge, optimal] (14) to (3);
%		\draw[edge, optimal] (15) to (5);
%		\draw[edge, optimal] (16) to (7);
%		\draw[edge, optimal] (11) to (12);
%		\draw[edge, taken] (12) to node[l] {$1$} (7);
%		\pocket{}
%\end{tikzpicture}
%
%
%\end{overprint}
%\end{center}
%\end{frame}
%
%\begin{frame}{Lower bound for $k > 3$}
%	\centering\includegraphics[height=7.5 
%	cm]{lowerBound}
%\end{frame}
%
%\begin{frame}{The edge arrival/departure model}
%\begin{block}{Limited departure model}
%	Edges arrive and depart. An edge cannot be removed while it is matched by the online matching.
%\end{block}
%\begin{block}{Full departure model}
%	An edge can be removed at any time.
%\end{block}
%\end{frame}
%
%\begin{frame}{Bounds in the full departure model}
%\begin{block}{Lower bound of 2}
%\centering\begin{tikzpicture}[baseline=(current bounding box.center)]
%		\tikzset{vertex}
%		\tikzset{edge}
%		\node[draw=none] (1) at  (0,0) {};
%		\node[vertex] (2) at  (1,0) {};
%		\node[vertex] (3) at  (2,0) {};
%		\node[draw=none] (4) at  (3,0) {};
%		\draw[edge, taken] (2) to node[l] {$1$} (3);
%		\draw[edge, od] (2) to (3);
%		
%		\node[vertex] (11) at  (0,-1) {};
%		\node[vertex] (12) at  (1,-1) {};
%		\node[vertex] (13) at  (2,-1) {};
%		\node[vertex] (14) at  (3,-1) {};
%		\draw[dashed] (12) to node[l] {$2$} (13);
%		\draw[edge, taken] (11) to node[l] {$1$} (12);
%		\draw[edge, od] (11) to node[l] {$1$} (12);
%		\draw[edge, taken] (13) to node[l] {$1$} (14);
%		\draw[edge, od] (13) to node[l] {$1$} (14);
%		
%		\node[draw=none] (21) at  (0,-2) {};
%		\node[vertex] (22) at  (1,-2) {};
%		\node[vertex] (23) at  (2,-2) {};
%		\node[draw=none] (24) at  (3,-2) {};
%		\draw[edge, taken] (22) to node[l] {$3$} (23);
%		\draw[edge, od] (22) to (23);
%		
%		\node[vertex] (31) at  (0,-3) {};
%		\node[vertex] (32) at  (1,-3) {};
%		\node[vertex] (33) at  (2,-3) {};
%		\node[vertex] (34) at  (3,-3) {};
%		\draw[dashed] (32) to node[l] {$4$} (33);
%		\draw[edge, taken] (31) to node[l] {$1$} (32);
%		\draw[edge, od] (31) to node[l] {$1$} (32);
%		\draw[edge, taken] (33) to node[l] {$1$} (34);
%		\draw[edge, od] (33) to node[l] {$1$} (34);
%		
%		\node[draw=none] (42) at  (1,-4) {};
%		\node[draw=none] (43) at  (2,-4) {};
%		\draw[draw=none] (42) to node[l] {$\dots$} (43);
%	\end{tikzpicture}
%\end{block}
%\end{frame}
%
%\begin{frame}{Lower bound for limited departure model}
%\begin{block}{The adversary}
%		The adversary maintains a graph such that the symmetric difference between $AL$ and $OPT$ consists only of augmenting paths, has no alternating cycles or alternating paths of even length.
%	\begin{itemize}
%		\item We consider the game as played on collection of alternating strings.\\
%		\item Augmenting a path translates to incrementing type of each edge. E.g. $01210 \rightarrow 12321$.\\		
%		\item The moves available to adversary are:
%	\begin{itemize}		
%		\item Splitting string: \ $12321 \rightarrow 123,1$ or $12321 \rightarrow 1,3,1$.
%		\item Merge, with $0$: \ $1,1 \rightarrow 101$.
%		\item Append $0$s when needed: \ $101 \rightarrow 01010$.
%	\end{itemize}
%	\end{itemize}
%\end{block}
%\end{frame}
%
%\begin{frame}{The game}
%	$k = 4$, competetive ratio of $10/7$
%	\centering\includegraphics[height=4cm]{adversary}
%\end{frame}
%
%\begin{frame}{Real, hard numbers}
%	\centering\includegraphics[height=5cm]{theWorld}
%\end{frame}



\end{document}
